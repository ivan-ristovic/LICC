\documentclass{beamer}
\usepackage[utf8]{inputenc} % make weird characters work
\usepackage[english,serbian]{babel}
%\usetheme{Pittsburgh}
%\usecolortheme{beetle}
\setbeamertemplate{navigation symbols}{}

\usepackage{listings}
\definecolor{codegreen}{rgb}{0,0.6,0}
\definecolor{codegray}{rgb}{0.5,0.5,0.5}
\definecolor{codeblue}{rgb}{0.0,0,0.82}
\lstdefinestyle{mystyle}{
    numbers=left,
    numberstyle=\scriptsize,
    numbersep=8pt,
    commentstyle=\color{codegray},
    keywordstyle=\color{codegreen},
    numberstyle=\tiny\color{codeblue},
    stringstyle=\color{codegreen},
    basicstyle=\ttfamily\footnotesize,
    breakatwhitespace=false,
    breaklines=true,
    captionpos=b,
    keepspaces=true,
    showspaces=false,
    showstringspaces=false,
    showtabs=false,
    tabsize=4,
    xleftmargin=3em,
    framexleftmargin=1.5em
}
\lstset{style=mystyle}

\usepackage[font=scriptsize,labelfont=bf]{caption}

\title{Neograničena provera modela softvera kroz inkrementalno SAT rešavanje}
\author{\href{mailto:ivan_ristovic@math.rs}{Ivan Ristovi\'c}}
\date{}


\begin{document}
\begin{frame}
    \titlepage
\end{frame}

\begin{frame}{Uvod}
    \begin{itemize}
        \item Ograničena provera modela softvera
        \item Inspiracija za neograničenu proveru modela softvera
        \item Nov način kodiranja stanja programa omogućava uklanjanje granice koja postoji kod ograničene provere modela softvera
        \item Kodiranje stanja programa u \emph{DimSpec} formulu korišćenjem alata \emph{LLUMC}
        \item Korišćenje inkrementalnih SAT rešavača za nalaženje modela
    \end{itemize}
\end{frame}

\begin{frame}{Pitanja}
    \centering
    ???
\end{frame}


\end{document}
