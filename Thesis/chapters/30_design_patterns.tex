\section{Obrasci za projektovanje}
\label{sec:DesignPatterns}

\emph{Obrasci za projektovanje} (engl. \emph{design patterns} \cite{DesignPatterns}, drugačije nazvani i projektni šabloni, uzorci) predstavljaju opšte i ponovno upotrebljivo rešenje čestog problema, često implementirani koristeći koncepte objektno-orijentisanog programiranja. Svaki obrazac ima četiri osnovna elementa:
\begin{itemize}
    \item ime - ukratko opisuje problem, rešenje i posledice
    \item problem - opisuje slučaj u kome se obrazac koristi
    \item rešenje - opisuje elemente dizajna kao i odnos tih elemenata
    \item posledice - obuhvataju rezultate i ocene primena obrasca
\end{itemize}

Obrasce za projektovanje je moguće grupisati po situaciji u kojoj se mogu iskoristiti ili načinu na koji rešavaju zadati problem. Stoga je opšte prihvaćena podela na tri grupe:
\begin{itemize}
    \item \emph{gradivni obrasci} (engl. \emph{creational patterns})
    \item \emph{strukturni obrasci} (engl. \emph{structural patterns})
    \item \emph{obrasci ponašanja} (engl. \emph{behavioral patterns})
\end{itemize}

Gradivni obrasci apstrahuju proces pravljenja objekata i važni su kada sistemi više zavise od sastavljanja objekata nego od nasleđivanja. Neki od najvažnijih gradivnih obrazaca su \emph{apstraktna fabrika} (engl. \emph{abstract factory}), \emph{graditelj} (engl. \emph{builder}), \emph{proizvodni metod} (engl. \emph{factory method}), \emph{prototip} (engl. \emph{prototype}) i \emph{unikat} (engl. \emph{singleton}). Strukturni obrasci se bave načinom na koji se klase i objekti sastavljaju u veće strukture. Neki od najvažnijih strukturnih obrazaca su \emph{adapter} (engl. \emph{adapter}), \emph{most} (engl. \emph{bridge}), \emph{sastav} (engl. \emph{composite}), \emph{dekorater} (engl. \emph{decorator}), \emph{fasada} (engl. \emph{facade}), \emph{muva} (engl. \emph{flyweight}) i \emph{proksi} (engl. \emph{proxy}). Obrasci ponašanja se bave načinom na koji se klase i objekti sastavljaju u veće strukture. Neki od najvažnijih strukturnih obrazaca su \emph{lanac odgovornosti} (engl. \emph{chain of responsibility}), \emph{komanda} (engl. \emph{command}), \emph{interpretator} (engl. \emph{interpreter}), \emph{iterator} (engl. \emph{iterator}), \emph{posmatrač} (engl. \emph{observer}), \emph{strategija} (engl. \emph{strategy}) i \emph{posetilac} (engl. \emph{visitor}). Za potrebe ovog rada, biće neophodno poznavanje nekoliko obrazaca, stoga će u nastavku biti opisani samo tih par obrazaca, zainteresovani čitalac može pročitati više u \cite{DesignPatternsBook}.


\subsection{Obrazac "Posmatrač"}
\label{subsec:DesignPatternsObserver}




\subsection{Obrazac "Posetilac"}
\label{subsec:DesignPatternsListener}
