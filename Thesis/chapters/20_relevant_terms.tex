\chapter{Pregled relevantnih pojmova}
\label{chp:RelevantTerms}

U ovom poglavlju će biti opisani koncepti i alati čije je razumevanje
potrebno kako bi se razumeo opis dalje apstrakcije i implementacije 
samog programa. Umesto analize samog sadržaja izvornog koda analizira
se \emph{apstraktno sintaksno stablo} (eng. \emph{Abstract Syntax Tree}, u 
daljem tekstu \emph{AST}), opisano u odeljku \ref{sec:AST}.
Alat koji je korišćen za generisanje parsera za proizvoljnu gramatiku jezika 
se zove \emph{Another Tool For Language Recognition} \cite{ANTLR}, 
u daljem tekstu \emph{ANTLR}, opisan u odeljku \ref{sec:ANTLR}.
Parser generiše AST specifičan za datu gramatiku i nema sličnosti u 
dobijenim apstrakcijama za različite jezike. Kako bismo poredili stabla 
različitih jezika, kreiramo reprezentaciju na višem nivou i specifični AST 
podižemo na taj nivo. Ta reprezentacija će biti opisana u narednim 
poglavljima, kao i načini kako se ona može analizirati. Takođe, pojmovi 
specifični za implementaciju će takođe biti opisani u ovom poglavlju.

\section{Apstraktna sintaksna stabla - AST}
\label{sec:AST}

Kako bi se od datoteke na fajl sistemu koja sadrži izvorni kod programa 
došlo do izvršivog programa, potrebno je izvršiti više koraka 
\cite{CompilerConstruction}:
\begin{itemize}
    \item pretprocesiranje
    \item prevođenje
    \item asembliranje
    \item linkovanje
\end{itemize}

Ovi koraci će biti opisani na jednom primeru. Pretpostavimo da želimo 
da kompajliramo kod pisan u programskom jeziku C prikazan na slici 
\ref{fig:CompilationProcessInit}. Primetimo, da postoji greška u datom
kodu - simbol \texttt{c} koji se koristi će biti prepoznat kao 
identifikator koji ne odgovara nijednoj promenljivoj. Ovo, doduše, nije
sintaksna greška - izraz \texttt{a + c} je sasvim validan u programskom
jeziku C bez analize konteksta u kom se javlja. Problem će postati 
očigledan tek nakon parsiranja izvornog koda i provere ispunjenosti 
sintaksih pravila. Ovakve greške se nazivaju \emph{semantičke greške}.

\begin{figure}[h!]
    \begin{lstlisting}
    #include<stdio.h>

    #define T int

    int main()
    {
        T a, b;
        a = a + c;        // c nije deklarisano
        printf("%d", a);
        return 0;
    }
    \end{lstlisting}
    \caption{Primer izvornog koda pisanog u programskom jeziku C.}
    \label{fig:CompilationProcessInit}
\end{figure}

U fazi pretprocesiranja se vrše samo tekstualne operacije kao što su
brisanje komentara ili zamena makroa u jezicima kao što je C. Prvo 
mesto gde se vrši analiza sadržaja izvornog fajla je faza prevođenja.
Tu analizu vrši program koji se naziva \emph{pretprocesor}. Rezultat 
rada pretprocesora za kod sa slike \ref{fig:CompilationProcessInit} 
bi izgledao kao na slici \ref{fig:CompilationProcessPrep} \footnote{
U nekim implementacijama C standardne biblioteke, moguće je da se 
poziv funckije \texttt{printf} zameni pozivom funkcije \texttt{fprintf}
sa ispisom na \texttt{stdout}. U standardu se propisuje da funkcije 
kao što je \texttt{printf} mogu biti implementirane kao makroi. Izlaz 
na slici \ref{fig:CompilationProcessPrep} je generisan od strane 
\texttt{GCC 7.4.0} po C11 standardu.} i ovo nije slučaj u datom 
okruženju.

\begin{figure}[h!]
    \begin{lstlisting}
    int main()
    {
        int a, b;
        a = a + c;
        printf("%d", a);
        return 0;
    }
    \end{lstlisting}
    \caption{Rezultat rada pretprocesora za kod sa slike 
             \ref{fig:CompilationProcessInit}.}
    \label{fig:CompilationProcessPrep}
\end{figure}

Prilikom faze prevođenja, kako prevodilac ne bi radio nad sirovim 
karakterima izvornog koda, potrebno je izvršiti pripremu istog. 
Prevodilac ima u vidu moguće elemente programskog jezika, tzv. 
\emph{tokene}, koje treba prepoznati u datom fajlu - ključne reči, 
operatore, promenljive itd. Program koji radi \emph{tokenizaciju} -
prepoznavanje tokena u izvornom fajlu - se naziva \emph{lekser}. 
Pojednostavljen primer tokena koje lekser pokušava da prepozna 
se može videti na slici \ref{fig:CLexerExample}. Primer izlaza
leksera za izlaz pretprocesora sa slike \ref{fig:CompilationProcessPrep}
se može videti na slici \ref{fig:CompilationProcessLex}.

\begin{figure}[h!]
    \begin{lstlisting}
    Identifier : IdentifierNondigit 
                 (IdentifierNondigit | Digit)*
               ;

    IdentifierNondigit : Nondigit
                       | UniversalCharacterName
                       ;

    Nondigit : [a-zA-Z_]
             ;

    Digit : [0-9]
          ;
    \end{lstlisting}
    \caption{Primer delimične definicije tokena za ime promenljive po C11 standardu.}
    \label{fig:CLexerExample}
\end{figure}

\begin{figure}[h!]
    \begin{lstlisting}
    identifier 'main'	 [LeadingSpace]	Loc=<sample.c:3:5>
    l_paren '('		Loc=<sample.c:3:9>
    r_paren ')'		Loc=<sample.c:3:10>
    l_brace '{'	 [StartOfLine]	Loc=<sample.c:4:1>
    int 'int'	 [StartOfLine] [LeadingSpace]	Loc=<sample.c:5:5>
    identifier 'a'	 [LeadingSpace]	Loc=<sample.c:5:9>
    comma ','		Loc=<sample.c:5:10>
    identifier 'b'	 [LeadingSpace]	Loc=<sample.c:5:12>
    semi ';'		Loc=<sample.c:5:13>
    identifier 'a'	 [StartOfLine] [LeadingSpace]	Loc=<sample.c:6:5>
    equal '='	 [LeadingSpace]	Loc=<sample.c:6:7>
    identifier 'a'	 [LeadingSpace]	Loc=<sample.c:6:9>
    plus '+'	 [LeadingSpace]	Loc=<sample.c:6:11>
    identifier 'c'	 [LeadingSpace]	Loc=<sample.c:6:13>
    semi ';'		Loc=<sample.c:6:14>
    identifier 'printf'	 [StartOfLine] [LeadingSpace]	Loc=<sample.c:7:5>
    l_paren '('		Loc=<sample.c:7:11>
    string_literal '"%d"'		Loc=<sample.c:7:12>
    comma ','		Loc=<sample.c:7:16>
    identifier 'a'	 [LeadingSpace]	Loc=<sample.c:7:18>
    r_paren ')'		Loc=<sample.c:7:19>
    semi ';'		Loc=<sample.c:7:20>
    return 'return'	 [StartOfLine] [LeadingSpace]	Loc=<sample.c:8:5>
    numeric_constant '0'	 [LeadingSpace]	Loc=<sample.c:8:12>
    semi ';'		Loc=<sample.c:8:13>
    r_brace '}'	 [StartOfLine]	Loc=<sample.c:9:1>
    eof ''		Loc=<sample.c:9:2>
    \end{lstlisting}
    \caption{Primer delimične definicije tokena za ime promenljive po standardu C11.}
    \label{fig:CompilationProcessLex}
\end{figure}

Nakon završetka rada leksera potrebno je parsirati dobijene tokene.
Parsiranje vrši program koji se naziva \emph{parser}. Parser, slično
kao što lekser ima definicije tokena jezika, mora imati informacije 
o gramatici jezika. Gramatika programskog jezika se najčešće definiše
putem kontekstno-slobodnih gramatika \cite{ContextFreeGrammars}, 
čiji je primer dat na slici \ref{fig:CompilationProcessGram}.

\begin{figure}[h!]
    \begin{lstlisting}
    functionDefinition
        :   declarationSpecifiers? declarator declarationList? compoundStatement
        ;

    declarationList
        :   declaration
        |   declarationList declaration
        ;

    declaration
        :   declarationSpecifiers initDeclaratorList ';'
        | 	declarationSpecifiers ';'
        |   staticAssertDeclaration
        ;
    \end{lstlisting}
    \caption{Isečak gramatike programskog jezika C po standardu C11.}
    \label{fig:CompilationProcessLex}
\end{figure}

Izlaz rada parsera je \emph{stablo parsiranja} (eng. \emph{parse tree} 
ili \emph{derivation tree}). Takvo stablo i dalje sadrži sve relevantne
informacije o izvornom kodu. Vizuelni prikaz rada parsera za gramatiku
sa slike C11 i izvonog koda sa slike \ref{fig:CompilationProcessPrep} je
dat na slici \ref{fig:CompilationProcessPars}.

\begin{figure}[h!]
    \includegraphics{images/}
    \caption{Isečak gramatike programskog jezika C po standardu C11.}
    \label{fig:CompilationProcessLex}
\end{figure}
\section{Obrasci za projektovanje}
\label{sec:DesignPatterns}

\emph{Obrasci za projektovanje} (engl. \emph{design patterns} \cite{DesignPatterns}, drugačije nazvani i \emph{projektni šabloni, uzorci}) predstavljaju opšte i ponovno upotrebljivo rešenje čestog problema, obi;no implementirani kroz koncepte objektno-orijentisanog programiranja. Svaki obrazac za projektovanje ima četiri osnovna elementa:
\begin{itemize}
    \item ime - ukratko opisuje problem, rešenje i posledice
    \item problem - opisuje slučaj u kome se obrazac koristi
    \item rešenje - opisuje elemente dizajna i odnos tih elemenata
    \item posledice - obuhvataju rezultate i ocene primena obrasca
\end{itemize}

Obrasce za projektovanje je moguće grupisati po situaciji u kojoj se mogu iskoristiti ili načinu na koji rešavaju zadati problem. Stoga je opšte prihvaćena podela na tri grupe:
\begin{itemize}
    \item \emph{gradivni obrasci} (engl. \emph{creational patterns})
    \item \emph{strukturni obrasci} (engl. \emph{structural patterns})
    \item \emph{obrasci ponašanja} (engl. \emph{behavioral patterns})
\end{itemize}

Gradivni obrasci apstrahuju proces pravljenja objekata i važni su kada sistemi više zavise od sastavljanja objekata nego od nasleđivanja. Neki od najvažnijih gradivnih obrazaca su \emph{apstraktna fabrika} (engl. \emph{abstract factory}), \emph{graditelj} (engl. \emph{builder}), \emph{proizvodni metod} (engl. \emph{factory method}), \emph{prototip} (engl. \emph{prototype}) i \emph{unikat} (engl. \emph{singleton}). Strukturni obrasci se bave načinom na koji se klase i objekti sastavljaju u veće strukture. Neki od najvažnijih strukturnih obrazaca su \emph{adapter} (engl. \emph{adapter}), \emph{most} (engl. \emph{bridge}), \emph{sastav} (engl. \emph{composite}), \emph{dekorater} (engl. \emph{decorator}), \emph{fasada} (engl. \emph{facade}), \emph{muva} (engl. \emph{flyweight}) i \emph{proksi} (engl. \emph{proxy}). Obrasci ponašanja se bave načinom na koji se klase i objekti sastavljaju u veće strukture. Neki od najvažnijih strukturnih obrazaca su \emph{lanac odgovornosti} (engl. \emph{chain of responsibility}), \emph{komanda} (engl. \emph{command}), \emph{interpretator} (engl. \emph{interpreter}), \emph{iterator} (engl. \emph{iterator}), \emph{posmatrač} (engl. \emph{observer}), \emph{strategija} (engl. \emph{strategy}) i \emph{posetilac} (engl. \emph{visitor}).

Za potrebe ovog rada, obrasci za projektovanje će se koristiti kao opšte prihvaćeno i programerski intuitivno rešenje određenih problema. Takođe, u kontekstu stabala parsiranja i AST-ova opisanih u poglavlju \ref{sec:AST}, obrasci \emph{Posmatrač} i \emph{Posetilac} su od velikog značaja jer pružaju interfejs za obilazak takvih stabala. Ovi obrasci se koriste od strane alata korišćenih u ovom radu kao što je ANTLR, opisan u poglavlju \ref{subsec:ANTLR}. Takođe, s obzirom da su ovi obrasci opšte-prihvaćeno rešenje za pružanje interfejsa obilaska stabala, biće korišćeni i u implementaciji opšte apstrakcije. U nastavku će zbog opisanih razloga biti opisani samo obrasci posmatrač i posetilac, dok zainteresovani čitalac može pročitati više u \cite{DesignPatternsBook}.

\subsection{Obrazac "Posmatrač"}
\label{subsec:DesignPatternsObserver}

Obrazac za projektovanje \emph{Posmatrač} je strukturni obrazac za projektovanje koji se koristi kada je potrebno definisati jedan-ka-više vezu između objekata tako da ukoliko jedan objekat promeni stanje (subjekat) svi zavisni objekti su obavešteni o izmeni i shodno ažurirani. Posmatrač predstavlja \emph{pogled} (engl. \emph{View}) u MVC (engl. \emph{Model-View-Controller}) arhitekturi. Na slici \ref{fig:UMLObserver} se može videti UML dijagram \cite{UML} ovog obrasca. 

\begin{figure}[h!]
\centering
\includegraphics[scale=0.7]{images/observer.jpg}
\caption{UML dijagram obrasca za projektovanje "Posmatrač".} 
\label{fig:UMLObserver}
\end{figure}

Primer upotrebe ovog obrasca može biti aukcija gde je aukcionar subjekat i započinje aukciju, dok učesnici aukcije (objekti) posmatraju aukcionera i reaguju na podizanje cene. Prihvatanje promene cene menja trenutnu cenu i aukcioner oglašava promenu iste, a svi učesnici aukcije dobijaju informaciju da se izmena izvršila. Za potrebe ovog rada, primer upotrebe može biti obilazak stablolike kolekcije (recimo stabla parsiranja) i obaveštavanje o nailasku na čvorove određenih tipova. Te informacije se dalje mogu iskoristiti za izračunavanja nad pomenutom strukturom ili generisanje novih struktura (recimo AST). 

\subsection{Obrazac "Posetilac"}
\label{subsec:DesignPatternsListener}

Obrazac za projektovanje \emph{Posetilac} je strukturni obrazac za projektovanje koji predstavlja operaciju koju je potrebno izvesti nad elementima objektne strukture. Posetilac omogućava definisanje nove operacije bez izmena klasa elemenata nad kojima operiše. Operacija koja će se izvesti zavisi od imena zahteva, tipa posetioca i tipa elementa kog posećuje. Na slici \ref{fig:UMLVisitor} se može videti UML dijagram \cite{UML} ovog obrasca. 

\begin{figure}[h!]
    \centering
    \includegraphics[scale=0.6]{images/visitor.jpg}
    \caption{UML dijagram obrasca za projektovanje "Posetilac".} 
    \label{fig:UMLVisitor}
\end{figure}

Primer upotrebe ovog obrasca može biti operisanje taksi kompanija. Kada osoba pozove taksi kompaniju (prihvatanje posetioca), kompanija šalje vozilo osobi koja je pozvala kompaniju. Nakon ulaska u vozilo (posetilac), mušterija ne kontroliše svoj transport već je to u rukama taksiste (posetioca). Za potrebe ovog rada, primer upotrebe može biti prikupljanje informacija o kolekciji stablolike strukture (recimo stablo parsiranja) i korišćenje istih za neko izračunavanje ili generisanje novih struktura (recimo AST). 

\input{chapters/23_antlr.tex}
\input{chapters/24_symbolics.tex}

