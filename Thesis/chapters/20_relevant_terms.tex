\chapter{Pregled relevantnih pojmova}
\label{chp:RelevantTerms}

U ovom poglavlju će biti opisani koncepti i alati čije je razumevanje
potrebno kako bi se razumeo opis dalje apstrakcije i implementacije 
samog programa. Umesto analize samog sadržaja izvornog koda analizira
se \emph{apstraktno sintaksno stablo} (eng. \emph{Abstract Syntax Tree}, u 
daljem tekstu \emph{AST}), opisano u odeljku \ref{sec:AST}.
Alat koji je korišćen za generisanje parsera za proizvoljnu gramatiku jezika 
se zove \emph{Another Tool For Language Recognition} \cite{ANTLR}, 
u daljem tekstu \emph{ANTLR}, opisan u odeljku \ref{sec:ANTLR}.
Parser generiše AST specifičan za datu gramatiku i nema sličnosti u 
dobijenim apstrakcijama za različite jezike \footnote{Zanimljivo je da, 
čak iako intuitivno pomislimo da su im gramatike slične, C i C++ gramatike
nemaju skoro nijedno istoimeno pravilo. Jedina sličnost je da obe gramatike 
koriste rekurzivna pravila da reše problem prioriteta operatora.}.
Kako bismo poredili stabla različitih jezika, kreiramo reprezentaciju na 
višem nivou i specifični AST podižemo na taj nivo. Ta reprezentacija će
biti opisana u narednim poglavljima, kao i načini kako se ona može
analizirati. Takođe, pojmovi specifični za implementaciju će takođe biti
opisani u ovom poglavlju.

\section{Apstraktna sintaksna stabla - AST}
\label{sec:AST}

Kako bi se od datoteke na fajl sistemu koja sadrži izvorni kod programa 
došlo do izvršivog programa, potrebno je izvršiti više koraka 
\cite{CompilerConstruction}:
\begin{itemize}
    \item pretprocesiranje
    \item prevođenje
    \item asembliranje
    \item linkovanje
\end{itemize}

Ovi koraci će biti opisani na jednom primeru. Pretpostavimo da želimo 
da kompajliramo kod pisan u programskom jeziku C prikazan na slici 
\ref{fig:CompilationProcessInit}. Primetimo, da postoji greška u datom
kodu - simbol \texttt{c} koji se koristi će biti prepoznat kao 
identifikator koji ne odgovara nijednoj promenljivoj. Ovo, doduše, nije
sintaksna greška - izraz \texttt{a + c} je sasvim validan u programskom
jeziku C bez analize konteksta u kom se javlja. Problem će postati 
očigledan tek nakon parsiranja izvornog koda i provere ispunjenosti 
sintaksih pravila. Ovakve greške se nazivaju \emph{semantičke greške}.

\begin{figure}[h!]
    \begin{lstlisting}
    #include<stdio.h>

    #define T int

    int main()
    {
        T a, b;
        a = a + c;        // c nije deklarisano
        printf("%d", a);
        return 0;
    }
    \end{lstlisting}
    \caption{Primer izvornog koda pisanog u programskom jeziku C.}
    \label{fig:CompilationProcessInit}
\end{figure}

U fazi pretprocesiranja se vrše samo tekstualne operacije kao što su
brisanje komentara ili zamena makroa u jezicima kao što je C. Prvo 
mesto gde se vrši analiza sadržaja izvornog fajla je faza prevođenja.
Tu analizu vrši program koji se naziva \emph{pretprocesor}. Rezultat 
rada pretprocesora za kod sa slike \ref{fig:CompilationProcessInit} 
bi izgledao kao na slici \ref{fig:CompilationProcessPrep} \footnote{
U nekim implementacijama C standardne biblioteke, moguće je da se 
poziv funckije \texttt{printf} zameni pozivom funkcije \texttt{fprintf}
sa ispisom na \texttt{stdout}. U standardu se propisuje da funkcije 
kao što je \texttt{printf} mogu biti implementirane kao makroi. Izlaz 
na slici \ref{fig:CompilationProcessPrep} je generisan od strane 
\texttt{GCC 7.4.0} po C11 standardu.} i ovo nije slučaj u datom 
okruženju.

\begin{figure}[h!]
    \begin{lstlisting}
    int main()
    {
        int a, b;
        a = a + c;
        printf("%d", a);
        return 0;
    }
    \end{lstlisting}
    \caption{Rezultat rada pretprocesora za kod sa slike 
             \ref{fig:CompilationProcessInit}.}
    \label{fig:CompilationProcessPrep}
\end{figure}

Prilikom faze prevođenja, kako prevodilac ne bi radio nad sirovim 
karakterima izvornog koda, potrebno je izvršiti pripremu istog. 
Prevodilac ima u vidu moguće elemente programskog jezika, tzv. 
\emph{tokene}, koje treba prepoznati u datom fajlu - ključne reči, 
operatore, promenljive itd. Program koji radi \emph{tokenizaciju} -
prepoznavanje tokena u izvornom fajlu - se naziva \emph{lekser}. 
Pojednostavljen primer tokena koje lekser pokušava da prepozna 
se može videti na slici \ref{fig:CLexerExample}. Primer izlaza
leksera za izlaz pretprocesora sa slike \ref{fig:CompilationProcessPrep}
se može videti na slici \ref{fig:CompilationProcessLex}.

\begin{figure}[h!]
    \begin{lstlisting}
    Identifier : IdentifierNondigit 
                 (IdentifierNondigit | Digit)*
               ;

    IdentifierNondigit : Nondigit
                       | UniversalCharacterName
                       ;

    Nondigit : [a-zA-Z_]
             ;

    Digit : [0-9]
          ;
    \end{lstlisting}
    \caption{Primer delimične definicije tokena za ime promenljive po C11 standardu.}
    \label{fig:CLexerExample}
\end{figure}

\begin{figure}[h!]
    \begin{lstlisting}
    identifier 'main'	 [LeadingSpace]	Loc=<sample.c:3:5>
    l_paren '('		Loc=<sample.c:3:9>
    r_paren ')'		Loc=<sample.c:3:10>
    l_brace '{'	 [StartOfLine]	Loc=<sample.c:4:1>
    int 'int'	 [StartOfLine] [LeadingSpace]	Loc=<sample.c:5:5>
    identifier 'a'	 [LeadingSpace]	Loc=<sample.c:5:9>
    comma ','		Loc=<sample.c:5:10>
    identifier 'b'	 [LeadingSpace]	Loc=<sample.c:5:12>
    semi ';'		Loc=<sample.c:5:13>
    identifier 'a'	 [StartOfLine] [LeadingSpace]	Loc=<sample.c:6:5>
    equal '='	 [LeadingSpace]	Loc=<sample.c:6:7>
    identifier 'a'	 [LeadingSpace]	Loc=<sample.c:6:9>
    plus '+'	 [LeadingSpace]	Loc=<sample.c:6:11>
    identifier 'c'	 [LeadingSpace]	Loc=<sample.c:6:13>
    semi ';'		Loc=<sample.c:6:14>
    identifier 'printf'	 [StartOfLine] [LeadingSpace]	Loc=<sample.c:7:5>
    l_paren '('		Loc=<sample.c:7:11>
    string_literal '"%d"'		Loc=<sample.c:7:12>
    comma ','		Loc=<sample.c:7:16>
    identifier 'a'	 [LeadingSpace]	Loc=<sample.c:7:18>
    r_paren ')'		Loc=<sample.c:7:19>
    semi ';'		Loc=<sample.c:7:20>
    return 'return'	 [StartOfLine] [LeadingSpace]	Loc=<sample.c:8:5>
    numeric_constant '0'	 [LeadingSpace]	Loc=<sample.c:8:12>
    semi ';'		Loc=<sample.c:8:13>
    r_brace '}'	 [StartOfLine]	Loc=<sample.c:9:1>
    eof ''		Loc=<sample.c:9:2>
    \end{lstlisting}
    \caption{Primer delimične definicije tokena za ime promenljive po standardu C11.}
    \label{fig:CompilationProcessLex}
\end{figure}

Nakon završetka rada leksera potrebno je parsirati dobijene tokene.
Parsiranje vrši program koji se naziva \emph{parser}. Parser, slično
kao što lekser ima definicije tokena jezika, mora imati informacije 
o gramatici jezika. Gramatika programskog jezika se najčešće definiše
putem kontekstno-slobodnih gramatika \cite{ContextFreeGrammars}, 
čiji je primer dat na slici \ref{fig:CompilationProcessGram}.

\begin{figure}[h!]
    \begin{lstlisting}
    functionDefinition
        :   declarationSpecifiers? declarator declarationList? compoundStatement
        ;

    declarationList
        :   declaration
        |   declarationList declaration
        ;

    declaration
        :   declarationSpecifiers initDeclaratorList ';'
        | 	declarationSpecifiers ';'
        |   staticAssertDeclaration
        ;
    \end{lstlisting}
    \caption{Isečak gramatike programskog jezika C po standardu C11.}
    \label{fig:CompilationProcessLex}
\end{figure}

Izlaz rada parsera je \emph{stablo parsiranja} (eng. \emph{parse tree} 
ili \emph{derivation tree}). Takvo stablo i dalje sadrži sve relevantne
informacije o izvornom kodu. Vizuelni prikaz rada parsera za gramatiku
sa slike C11 i izvonog koda sa slike \ref{fig:CompilationProcessPrep} je
dat na slici \ref{fig:CompilationProcessPars}.

\begin{figure}[h!]
    \includegraphics{images/}
    \caption{Isečak gramatike programskog jezika C po standardu C11.}
    \label{fig:CompilationProcessLex}
\end{figure}
\section{ANTLR}
\label{sec:ANTLR}

Pretpostavljajući da imamo gramatiku proizvoljnog programskog jezika, postavlja se pitanje: \emph{Da li je moguće definisati postupak i zatim napraviti program koji će generisati kodove leksera i parsera napisane u određenom programskom jeziku za proizvoljnu gramatiku datu na ulazu?}. Odgovor je potvrdan i postoji veliki broj alata koji se mogu koristiti u ove svrhe, od kojih je navedeno par njih: 
\begin{itemize}
    \item \emph{GNU Bison} \cite{GNUBison}\\
        GNU Bison je generator parsera i deo GNU projekta \cite{GNUProject}, često referisan samo kao \emph{Bison}. Bison generiše parser na osnovu korisnički definisanih kontekstno slobodnih gramatika \cite{ContextFreeGrammars}, upozoravajući pritom na dvosmislenosti prilikom parsiranja ili nemogućnost primena gramatičkih pravila. Generisani parser je najčešće C a ređe C++ kod, mada se u vreme pisanja ovog rada eksperimentiše sa Java podrškom. Generisani kodovi su u potpunosti prenosivi i ne zahtevaju specifične kompajlere. Bison može da, osim podrazumevanih \emph{LALR(1)} \cite{LALR1} parsera, generiše i kanoničke \emph{LR} \cite{LR}, \emph{IELR(1)} \cite{IELR1} i \emph{GLR} \cite{GLR} parsere.
    \item \emph{Flex} \cite{Flex}\\
        Kreiran kao alternativa \emph{lex}-u \cite{LexYacc}, Flex generiše samo leksere pa se stoga najčešće koristi u kombinaciji sa drugim alatima koji mogu da generišu parsere, kao što je \emph{BYACC}, opisan u nastavku.
    \item \emph{BYACC} \cite{BYACC}\\
        \emph{Berkeley YACC}, skraćeno \emph{BYACC}, pisan po ANSI C standardu i otvorenog koda, se smatra od strane mnogih kao \textit{najbolja varijanta YACC-a} \cite{LexYacc}. BYACC dozvoljava tzv. \emph{reentrant} kod - memorija je deljenja između poziva pa je bezbedno konkurentno izvršavanje koda - na način kompatibilan sa Bison-om i to je delom razlog njegove popularnosti.
    \item \emph{ANTLR} \cite{ANTLR}\\
        \emph{Another Tool for Language Recognition}, ili kraće \emph{ANTLR}, je generator \emph{LL(*)} \cite{LLStar} leksera i parsera pisan u programskom jeziku Java sa intuitivnim interfejsom za obilazak stabla parsiranja. Verzija $3$ podržava generisanje parsera u jezicima Ada95, ActionScript, C, C\#, Java, JavaScript, Objective-C, Perl, Python, Ruby, i Standard ML, dok verzija $4$ u vreme pisanja ovog rada samo generiše parsere u narednim jezicima: Java, C\#, C++, JavaScript, Python, Swift i Go.
\end{itemize}
ANTLR, verzije $4$, je izabran u ovom radu zbog svoje jednostavnosti, intuitivnosti i podrške za mnoge moderne programske jezike. Verzija $4$ je izabrana umesto verzije $3$ po preporuci autora, na osnovu eksperimentalne analize brzine i pouzdanosti verzije $4$ u odnosu na verziju $3$. Lekseri i parseri za ulazne gramatike će u implementaciji biti generisani u programskom jeziku C\#.

Parseri generisani koristeći ANTLR koriste novu tehnologiju koja se naziva \emph{Prilagodljiv LL(*)} (engl. \emph{Adaptive LL(*)}) ili \emph{ALL(*)} \cite{ANTLRReference}, dizajniranu od strane Terensa Para, autora ANTLR-a, i Sema Harvela. \emph{ALL(*)} vrši \emph{dinamičku analizu} gramatike u fazi izvršavanja, dok su starije verzije radile analizu pre pokretanja parsera, tzv. \emph{statičku analizu}. Ovaj pristup je takođe efikasniji zbog značajno manjeg prostora ulaznih sekvenci.

Najbolji aspekt ANTLR-a je lakoća definisanja gramatičkih pravila koji opisuju sintaksne konstrukte nalik na aritmetičkim izrazima u programskim jezicima. Primer jednostavnog pravila za definisanje aritmetičkog izraza je dat na slici \ref{fig:ANTLRExpressions}. Pravilo \texttt{exp} je levo rekurzivno jer barem jedna od njegovih alternativnih definicija referiše na baš pravilo \texttt{exp}. ANTLR4 automatski zamenjuje levo rekurzivna pravila u nerekurzivne ekvivalente. Jedini zahtev koji mora biti ispunjen je da levo rekurzivna pravila moraju biti \emph{direktna} - da pravila odmah referišu sama sebe. Pravila ne smeju referisati drugo pravilo sa leve strane definicije takvo da se eventualno kroz rekurziju stigne nazad do pravila od kog se krenulo bez poklapanja sa nekim tokenom.

\begin{figure}[h!]
    \begin{lstlisting}[language={}]
    exp : (exp)
        | exp '*' exp
        | exp '+' exp
        | INT
        ;
    \end{lstlisting}
    \caption{Definicija uprošćenog aritmetičkog izraza u ANTLR4 gramatici.}
    \label{fig:ANTLRExpressions}
\end{figure}


\subsection{Preduslovi za pokretanje ANTLR4}
\label{subsec:ANTLRInstallation}

Kako bi ANTLR generisao parser u proizvoljnom programskom jeziku, potrebno je instalirati ANTLR i imati \emph{Java Runtime Environment} (skr. \emph{JRE}) instaliran na sistemu i dostupan globalno pokretanjem putem komande \texttt{java}. Instalacija se sastoji od preuzimanja najnovijeg \emph{.jar} fajla
\footnote{Takođe je moguće kompajlirati izvorni kod dostupan na servisu GitHub \url{https://github.com/antlr/antlr4}}, sa zvanične stranice \cite{ANTLR} ili recimo korišćenjem \emph{curl} alata: 
\begin{lstlisting}[language={}]
$ curl -O http://www.antlr.org/download/antlr-4-complete.jar
\end{lstlisting}

Na UNIX sistemima moguće je kreirati alias \texttt{antlr4} ili \emph{shell} skript unutar direktorijuma \texttt{/usr/local/bin} sa imenom \texttt{antlr4} koji će pokrenuti \emph{.jar} fajl na sledeći način (pretpostavljajući da se \emph{.jar} fajl nalazi u direktorijumu \texttt{/usr/local/lib}):
\begin{lstlisting}[language={}]
#!/bin/sh
java -cp "/usr/local/lib/antlr4-complete.jar:$CLASSPATH" org.antlr.v4.Tool $*
\end{lstlisting}

Na Windows sistemima moguće je kreirati \emph{batch} skript sa imenom \texttt{antlr4.bat} koji će pokrenuti ANTLR4, na sledeći način (pretpostavljajući da se \emph{.jar} fajl nalazi u direktorijumu \texttt{C:\textbackslash{}lib}):
\begin{lstlisting}[language={}]
java -cp C:\lib\antlr-4-complete.jar;%CLASSPATH% org.antlr.v4.Tool %*
\end{lstlisting}

Ukoliko su aliasi ili skript fajlovi imenovani kao iznad, moguće je iz komandne linije pojednostavljeno pokretati ANTLR4:  
\begin{lstlisting}[language={}]
$ antlr4
ANTLR Parser Generator Version 4.0
-o ___    specify output directory where all output is generated
-lib ___  specify location of .tokens files
...
\end{lstlisting}

Dodatno, za Unix sisteme \footnote{Za Windows operativni sistem je moguće kreirati \emph{batch} skript po opisu na \url{https://github.com/antlr/antlr4/blob/master/doc/getting-started.md}.}, moguće je kreirati dodatni alias \texttt{grun} (ili alternativno, kreirati \texttt{shell script}) za biblioteku \texttt{TestRig}. Biblioteka TestRig se može koristiti za brzo testiranje parsera - moguće je pokrenuti parser od bilo kog pravila i dobiti izlaz parsera u raznim formatima. TestRig dolazi uz ANTLR \texttt{.jar} fajl i moguće je napraviti prečicu za brzo pokretanje (nalik na ANTLR alias):
\begin{lstlisting}[language={}]
$ alias grun='java -cp "/usr/local/lib/antlr-4-complete.jar:$CLASSPATH" org.antlr.v4.gui.TestRig'
\end{lstlisting}


\subsection{Generisanje parsera koristeći ANTLR4}
\label{subsec:ANTLRParserGeneration}

Prvi korak u izradi aplikacije koja u sebi koristi parsiranje nekog jezika je definisanje gramatike jezika i kreiranje leksera i parsera za isti. U nastavku će biti opisan proces kreiranja interfejsa za parsiranje programa pisanih u pseudo-programskom jeziku (u nastavku \emph{pseudo-jezik}), nalik na pseudokod. Ovako dobijeni interfejs će moći da se koristi u opšte svrhe, za potrebe ovog rada će se koristiti za generisanje AST stabla za program pisan u pseudo-jeziku.

Definišimo gramatiku pseudo-jezika prateći ANTLR pravila za definisanje gramatika. Kao i za svaki drugi programski jezik, treba obezbediti da postoje određeni koncepti koji se pojavljuju u programskim jezicima: \emph{identifikatori}, \emph{izrazi}, \emph{naredbe}, \emph{funkcije} i slično. Za sada se fokusirajmo na naredbe, kao samostalne izvršive jedinice koda. Stoga program možemo smatrati kao niz naredbi. U nekim slučajevima će biti potrebno definisanje kompleksnih naredbi koje se sastoje od više drugih naredbi, i ovakve složene naredbe ćemo zvati \emph{blok} ili \emph{blok naredbi}. Stoga, radi konzistentnosti, program će biti blok naredbi. Kako bismo označili da su naredbe deo bloka naredbi, koristićemo reči \texttt{begin} i \texttt{end}, osim ukoliko je reč o samo jednoj naredbi. Ovakve situacije rešavamo definisanjem \emph{alternativa} u definiciji pravila - više definicija razdvojenih simbolom \texttt{|}. Specijalne reči kao što su \texttt{begin} i \texttt{end} će biti rezervisane reči našeg pseudo-jezika, tzv. \emph{ključne reči}. Na slici \ref{fig:PseudoDef1} se može videti definicija programa \footnote{Drugim rečima, jedan program u pseudo-jeziku je jedinica prevođenja, pa je zato pravilo nazvano \emph{unit}.} i bloka naredbi pseudo-jezika, pri čemu se ključne reči u pravilima navode između apostrofa. ANTLR dozvoljava jednostavne definicije pravila u kojima figuriše promenljiv broj drugih pravila, pri čemu se koriste simboli kao u regularnim izrazima \footnote{U regularnim izrazima, simbol \texttt{a?} označava opciono pojavljivanje simbola \texttt{a}, simbol \texttt{a+} označava jedno ili više pojavljivanja simbola \texttt{a}, a simbol \texttt{a*} označava proizvoljan broj pojavljivanja simbola \texttt{a} - kombinacija simbola \texttt{?} i \texttt{+}.}, što je iskorišćeno za definiciju pravila bloka naredbi. \texttt{NAME} predstavlja ime programa, što je zapravo identifikator. Identifikatore ćemo definisati kasnije, za sada možemo posmatrati identifikator kao nisku karaktera s tim što će postojati restrikcije vezane za to koji karakteri se mogu naći unutar identifikatora ali o tome će biti reči kasnije.

\begin{figure}[h!]
    \begin{lstlisting}[language={}]
    unit
        : 'algorithm' NAME block EOF
        ;
    
    block
        : 'begin' statement+ 'end'
        | statement
        ;
    \end{lstlisting}
    \caption{Definicija jedinice prevođenja i bloka naredbi pseudo-jezika.}
    \label{fig:PseudoDef1}
\end{figure}

Sledeći korak je definisanje naredbe pseudo-jezika. Slično kao i u drugim programskim jezicima, potrebno je podržati koncept deklaracije promenljive, dodele vrednosti izraza promenljivoj, naredbe kontrole toka - grananje i petlje. Na slici \ref{fig:PseudoDef2} je definisano šta se sve smatra jednom naredbom. Naredbe mogu biti i prazne, što je označeno ključnom rečju \texttt{pass}. Iz definicije naredbe sa slike se jasno vidi šta sve može biti naredba (prateći redosled alternativa pravila):
\begin{itemize}
    \item deklaracija
    \item dodela
    \item poziv funkcije (označen kao \texttt{cexp}, skraćeno od \emph{function call expression}) \footnote{Funkcije mogu vratiti vrednosti pa se stoga njihovi pozivi mogu naći u izrazima - dakle poziv funkcije je validan izraz (stoga \texttt{expression} u imenu \texttt{function call expression}). Naravno, ta vrednost se može ignorisati ili pak sama funkcija može biti takva da nema povratnu vrednost već je samo neophodno izvršiti je zbog sporednih efekata.}
    \item vraćanje vrednosti izraza (ključna vrednost \texttt{return}) iz funkcije
    \item prekidanje izvršavanja davanjem poruke o grešci
    \item naredba grananja
    \item \emph{while} petlja
    \item \emph{repeat-until} petlja
    \item inkrementiranje/dekrementiranje vrednosti promenljive
\end{itemize}
    
\begin{figure}[h!]
    \begin{lstlisting}[language={}]
    statement
        : 'pass'
        | declaration
        | assignment
        | cexp
        | 'return' exp
        | 'error' STRING
        | 'if' exp 'then' block ('else' block)? 
        | 'while' exp 'do' block 
        | 'repeat' block 'until' exp
        | ('increment' | 'decrement') var	
        ;
    \end{lstlisting}
    \caption{Definicija naredbe pseudo-jezika.}
    \label{fig:PseudoDef2}
\end{figure}

Deklaracija, prikazana na slici \ref{fig:PseudoDef3}, uvodi pojavljivanje simbola datog preko identifikatora \texttt{NAME} kao oznaku za promenljivu, funkciju ili proceduru - funkciju bez povratne vrednosti. Svaka promenljiva mora biti određenog tipa, što se postiže pravilom \texttt{type}. Promenljivoj se, opciono, može pridružiti početna vrednost, drugim rečima promenljiva se može \emph{inicijalizovati} tako da joj se pridruži vrednost nekog izraza. Procedure i funkcije imaju opcione parametre, vrednosti izraza koje im se prosleđuju kasnije u pozivu kao argumenti. Lista parametara, takođe prikazana na slici \ref{fig:PseudoDef3}, se navodi kao lista proizvoljno mnogo parova \texttt{NAME : type}, što se vidi iz definicije pravila \texttt{parlist}.

\begin{figure}[h!]
    \begin{lstlisting}[language={}]
    declaration
        : 'declare' type NAME ('=' exp)? 
        | 'procedure' NAME '(' parlist? ')' block 
        | 'function' NAME '(' parlist? ')' 'returning' type block 
        ;

    parlist
        : NAME ':' type (',' NAME ':' type)*
        ;
    \end{lstlisting}
    \caption{Definicija deklaracije u pseudo-jeziku.}
    \label{fig:PseudoDef3}
\end{figure}

Identifikatori su niske karaktera koje predstavljaju ime koje odgovara određenoj memorijskoj adresi. Identifikatori se koriste umesto sirovih vrednosti adresa kako bi kod bio čitljiviji i lakši za pisanje - na nivou asemblera se većinom koriste adrese ili automatski generisane oznake. Na slici \ref{fig:PseudoDef4} se može videti definicija identifikatora. Identifikator se sastoji od slova, cifara i simbola \texttt{\_}, s tim što ne sme početi cifrom. Ovo je konvencija koju prati dosta jezika, uključujući programski jezik C. Primetimo da je identifikator nešto što bi lekser trebalo da prepozna tokom tokenizacije. Međutim, kada definišemo gramatiku od koje će ANTLR praviti lekser i parser, možemo i tokene definisati preko gramatičih pravila dajući regularni izraz za njihovo poklapanje. Listovi stabla parsiranja su uvek tokeni, drugim rečima se nazivaju i \emph{terminalni simboli}. Tokeni se, naravno, mogu naći bilo gde u stablu parsiranja.

\begin{figure}[h!]
    \begin{lstlisting}[language={}]
    NAME
        : [a-zA-Z_][a-zA-Z_0-9]*
        ;
    \end{lstlisting}
    \caption{Definicija identifikatora u pseudu-jeziku.}
    \label{fig:PseudoDef4}
\end{figure}

Pošto želimo da pseudo-jezik bude strogo tipiziran, potreban je koncept tipa (što smo videli u deklaracijama), čija je definicija data na slici \ref{fig:PseudoDef4}. Tip može biti \emph{primitivan} (drugim rečima \emph{prost}) ili \emph{složen}. Primitivni tipovi su podržani u samoj sintaksi jezika - u našem slučaju brojevi i niske. Brojevi mogu biti celi ili realni. U složene tipove spadaju korisnički definisani tipovi (sa imenom \texttt{NAME}, u četvrtoj alternativi pravila \texttt{typename} sa slike \ref{fig:PseudoDef4}) i kolekcije. Od kolekcija su podržani nizovi, liste i skupovi. Prilikom definicije kolekcije mora se navesti tip elemenata kolekcije i taj tip mora biti uniforman - isti za sve elemente kolekcije. 

\begin{figure}[h!]
    \begin{lstlisting}[language={}]
    type 
        : typename 'array'?
        | typename 'list'?
        | typename 'set'?
        ;

    typename 
        : 'integer' 
        | 'real' 
        | 'string' 
        | NAME 
        ;
    \end{lstlisting}
    \caption{Definicija tipa u pseudu-jeziku.}
    \label{fig:PseudoDef4}
\end{figure}

Izrazi, iako se definišu rekurzivno, se mogu posmatrati kao kombinacija promenljivih, operatora i poziva funkcija sa odlikom da se mogu \emph{evaluirati}, tj. moguće je izračunati njegovu vrednost. Iz definicije pravila \texttt{exp} na slici \ref{fig:PseudoDef5}, mogu se uočiti tipovi izraza, pri čemu nije vođeno računa o matematičkom prioritetu operatora, radi jednostavnosti. Izraz može biti \emph{literal}, koji predstavlja konstantu, bilo brojevnu, logičku ili nisku karaktera. Promenljive, definisane pravilom \texttt{var} su takođe izrazi, jer se trenutna vrednost promenljive posmatra kao vrednosti izraza. Primetimo da promenljiva može biti kolekcijskog tipa, u kom slučaju se navodi redni broj elementa nakon identifikatora promenljive - taj redni broj može biti rezultat evaluacije drugog izraza, ali ne bilo kakvog, stoga se u pravilu \texttt{iexp} definiše šta sve može biti korišćeno da se indeksira element kolekcije. Izrazima se može dati prioritet pomoću zagrada, što se vidi u trećoj alternativi pravila \texttt{exp}. U naredne tri alternative su opisani tipovi izraza: aritmetički, relacioni i logički. Aritmetički izrazi su vezani aritmetičkim operatorima definisanim preko pravila \texttt{aop}, slično važi i za ostala dva tipa. Svi tipovi izraza navedeni iznad su binarni, što znači da operatori zahtevaju dva argumenta. Postoje i unarni izrazi, od kojih su podržane promena znaka i logička negacija, što se vidi iz pravila \texttt{uop}.

\begin{figure}[h!]
    \begin{lstlisting}[language={}]
    exp
        : literal 
        | var
        | '(' exp ')'
        | exp aop exp
        | exp rop exp
        | exp lop exp
        | uop exp
        | cexp
        ;

    var 
        : NAME ('[' iexp ']')?
        ;
    
    iexp 
        : literal
        | var
        | aexp
        ;

    cexp
        : 'call' NAME '(' explist? ')'
        ;

    explist
        : exp (',' exp)*
        ;

    aop : '+' | '-' | '*' | '/' | 'div' | 'mod' ;
    rop : '>' | '>=' | '<' | '<=' | '==' | '=/=' ;
    lop : 'and' | 'or' ;
    uop : '-' | 'not' ;
    \end{lstlisting}
    \caption{Definicija izraza u pseudu-jeziku.}
    \label{fig:PseudoDef5}
\end{figure}

Definicija literala je prikazana na slici \ref{fig:PseudoDef6}. Literali mogu bili istinitosne konstante \texttt{True} i \texttt{False}, brojevne konstante ili niske karaktera. Brojevne konstante mogu bili celobrojni ili realni dekadni brojevi. Realne konstatne je moguće definisati u fiksnom ili pokretnom zarezu. Niske se mogu definisati između navodnika ili apostrofa. Pritom, kao i u modernim programskim jezicima, moguće je navesti sekvence koje predstavljaju specijalne karaktere kao što su novi red, tabulator itd. Oznaka \texttt{fragment} označava optimizaciju, naime za nije potrebno da postoji na primer pravilo \texttt{Digit}, već samo dajemo simbol za regularni izraz koji će se koristiti u više drugih pravila i poklapati jednu dekadnu cifru.

\begin{figure}[h!]
    \begin{lstlisting}[language={}]
    literal : 'True' | 'False' | INT | FLOAT | STRING ;

    STRING
        : '"' ( EscapeSequence | ~('\\'|'"') )* '"' 
        ;

    fragment
    EscapeSequence
        : '\\' [abfnrtvz"'\\]
        | '\\' '\r'? '\n'
        ;

    INT
        : Digit+
        ;

    FLOAT
        : Digit+ '.' Digit* ExponentPart?
        | '.' Digit+ ExponentPart?
        | Digit+ ExponentPart
        ;

    fragment
    ExponentPart
        : [eE] [+-]? Digit+
        ;

    fragment
    Digit
        : [0-9]
        ;
    \end{lstlisting}
    \caption{Definicija konstanti u pseudu-jeziku.}
    \label{fig:PseudoDef6}
\end{figure}

Poslednje što treba definisati je sve ono što lekser treba da preskoči tokom prolaska kroz izvorni kod programa. To su beline (nevidljivi karakteri kao što su razmaci, tabulatori i novi redovi) i komentari. Definicije ovih pravila se mogu videti na slici \ref{fig:PseudoDef7}. Vidimo da se u njima koristi posebna oznaka \texttt{-> skip}, koja predstavlja instrukcije lekseru da preskoči sve ono što ovo pravilo poklopi. Komentari su u stilu kao u programskom jeziku C (ali naravno, isti stil se koristi i u mnogim jezicima) i mogu biti jednolinijski ili višelinijski. Beline koje treba preskočiti su definisane u pravilu \texttt{WS}, skraćeno od \emph{whitespace}, što u prevodu sa engleskog znači \emph{beli prostor, belina}.

\begin{figure}[h!]
    \begin{lstlisting}[language={}]
    BlockComment
        :   '/*' .*? '*/'  -> skip
        ;
    LineComment
        :   '//' ~[\r\n]*  -> skip
        ;
    WS  
        : [ \t\u000C\r\n]+ -> skip
        ;
    \end{lstlisting}
    \caption{Definicija komentara i belina u pseudu-jeziku.}
    \label{fig:PseudoDef7}
\end{figure}

\begin{figure}[h!]
    \begin{lstlisting}[language={}]
    grammar Pseudo;
    \end{lstlisting}
    \caption{Definicija imena gramatike.}
    \label{fig:PseudoDef8}
\end{figure}

Ovako definisanu gramatiku možemo sačuvati u fajl sa imenom \texttt{Pseudo.g4}, potrebno je samo navesti ime gramatike na početku fajla, kao na slici \ref{fig:PseudoDef8}. Naredni korak je kreiranje leksera i parsera koristeći ANTLR4, predpostavljajući da je instaliran na način opisan u \ref{subsec:ANTLRInstallation}. Pokretanjem ANTLR-a generišemo lekser i parser za gramatiku pseudo-jezika:
\begin{lstlisting}[language={}]
$ antlr4 Pseudo.g4
\end{lstlisting}

ANTLR4 će generisati lekser i parser podrazumevano napisane u programskom jeziku Java. Ukoliko želimo to da promenimo, možemo koristiti opciju \texttt{-Dlanguage=...}. Kako bismo testirali generisani lekser i parser, možemo koristiti ANTLR TestRig da vizualno prikažemo stablo parsiranja, s tim što moramo prvo kompajlirati generisane Java kodove. TestRig pozivamo navođenjem ime gramatike (koje se poklapa sa imenom leksera i parsera) i imenom pravila od koga će parser krenuti. Opcija \texttt{-gui} pokreće vizualni prikaz stabla parsiranja pokazan na slici \ref{fig:PseudoTreeGui} (vizualni prikaz je moguće preskočiti i samo ispisati stablo u LISP formi koristeći opciju \texttt{-tree}), mada je moguće i ispisati samo tokene koristeći opciju \texttt{-tokens}. Ulaz se prosleđuje programu dok se ne naiđe na simbol \texttt{EOF}, ili alternativno se može preneti ulaz putem UNIX pipeline-a:
\begin{lstlisting}[language={}]
$ javac *.java
$ echo "declare integer x = 5" | grun Pseudo declaration -tokens
[@0,0:6='declare',<'declare'>,1:0]
[@1,8:14='integer',<'integer'>,1:8]
[@2,16:16='x',<NAME>,1:16]
[@3,18:18='=',<'='>,1:18]
[@4,20:20='5',<INT>,1:20]
[@5,22:21='<EOF>',<EOF>,2:0]
$ echo "declare integer x = 5" | grun Pseudo declaration -tree
(declaration declare (type (typename integer)) x = (exp (literal 5)))
$ echo "declare integer x = 5" | grun Pseudo declaration -gui
\end{lstlisting}    

\begin{figure}[h!]
    \centering
        \includegraphics[scale=0.8]{images/pseudo_parse_tree.png}
    \caption{Prikaz dela stabla parsiranja koje generiše parser kreiran od strane alata ANTLR4 za kod sa slike \ref{fig:CompilationProcessPrep}.}
    \label{fig:PseudoTreeGui}
\end{figure}


\subsection{Integrisanje parsera u program}
\label{subsec:ANTLRParserIntegration}

ANTLR, osim leksera i parsera za unetu gramatiku, može da kreira interfejse koji prate \emph{dizajn šablone} (takođe nazivani i \emph{projektni uzorci}) posetilac (engl. \emph{visitor}) i posmatrač (engl. \emph{observer}) opisane u \ref{sec:DesignPatterns}. Tako kreirani interfejsi imaju metode koje je potrebno implementirati koje se u slučaju posmatrača pozivaju automatski tokom obilaska stabla parsiranja onda kada se naiđe na pravilo koje odgovara tom metodu a u slučaju posetioca odluku o obilasku dece u stablu donosi programer.



