\section{Parsiranje gramatika programskih jezika}
\label{sec:ParsingGrammars}

Ukoliko imamo gramatiku proizvoljnog programskog jezika, postavlja se pitanje: 
\begin{quote}
    Da li je moguće definisati postupak i zatim napraviti program koji će generisati kodove leksera i parsera napisane u nekom specifičnom programskom jeziku?
\end{quote}
Odgovor je potvrdan i postoji veliki broj alata koji se mogu koristiti u ove svrhe, od kojih je navedeno par njih u odeljcima ispod.

\subsection{Lex i Flex}
\label{subsec:LexFlex}
\emph{Lex} \cite{LexYacc} je program koji generiše leksere. Danas se više koristi \emph{flex} \cite{Flex}, kreiran kao alternativa \emph{lex}-u, s obzirom da je i do dva puta brži od \emph{lex}-a, koristi manje memorije nego \emph{lex}, i vreme kompilacije leksera koje \emph{flex} generiše je i do tri puta kraće nego kompilacija leksera koje generiše \emph{lex}. Pošto \emph{flex}, isto kao i \emph{lex}, generiše samo leksere, najčešće se koristi u kombinaciji sa drugim alatkama koje mogu da generišu parsere, kao što su npr. \emph{GNU Bison}, \emph{YACC} ili \emph{BYACC}.

\subsection{YACC i BYACC}
\label{subsec:BYACC}
\emph{YACC} \cite{LexYacc} je program koji generiše \emph{LALR} \cite{DragonBook} parser na osnovu gramatike date na ulazu zajedno sa akcijama koje će se izvršiti kada se određeno pravilo prepozna u izvornom kodu. \emph{YACC} ne vrši leksičku analizu, stoga se obično koristi zajedno sa popularnim leksičkim analizatorima kao što su \emph{lex} i \emph{flex}. 

\emph{Berkeley YACC}, skraćeno \emph{BYACC} \cite{BYACC}, je generator parsera pisan po ANSI C standardu i otvorenog je koda. Posmatra se od strane mnogih kao \textit{najbolja varijanta YACC-a} \cite{LexYacc}. \emph{BYACC} dozvoljava tzv.~\emph{reentrant} k\^od --- omogućava bezbedno konkurentno izvršavanje koda na način kompatibilan sa Bison-om i to je delom razlog njegove popularnosti.

\subsection{GNU Bison}
\label{subsec:GNUBison}
\emph{GNU Bison} \cite{GNUBison} je generator parsera i deo GNU projekta \cite{GNUProject}, često referisan samo kao \emph{Bison}. \emph{Bison} generiše parser na osnovu korisnički definisane kontekstno slobodne gramatike \cite{AutomataTheory}, upozoravajući pritom na dvosmislenosti prilikom parsiranja ili nemogućnost primene gramatičkih pravila. Generisani parser je najčešće C a ređe C++ program, mada se u vreme pisanja ovog rada eksperimentiše sa Java podrškom. Generisani k\^od je u potpunosti prenosiv i ne zahteva specifične kompajlere. Bison može da, osim podrazumevanih \emph{LALR(1)} \cite{DragonBook} parsera, generiše i kanoničke \emph{LR} \cite{LR}, \emph{IELR(1)} \cite{IELR1} i \emph{GLR} \cite{GLR} parsere.

\subsection{ANTLR}
\label{subsec:ANTLR}
\emph{Another Tool for Language Recognition}, ili kraće \emph{ANTLR} \cite{ANTLR}, je generator \emph{LL(*)} \cite{LLStar} leksera i parsera pisan u programskom jeziku Java sa intuitivnim interfejsom za obilazak stabla parsiranja. Verzija $3$ podržava generisanje parsera u jezicima Ada95, ActionScript, C, C\#, Java, JavaScript, Objective-C, Perl, Python, Ruby, i Standard ML, dok verzija $4$, u daljem tekstu ANTLR4, u vreme pisanja ovog rada generiše parsere u narednim programskim jezicima: Java, C\#, C++, JavaScript, Python, Swift i Go.\footnote{ANTLR verzije $4$ je izabran u ovom radu zbog svoje popularnosti, jednostavnosti, intuitivnosti i podrške za mnoge moderne programske jezike. Verzija $4$ je izabrana po preporuci autora ANTLR-a, na osnovu eksperimentalne analize brzine i pouzdanosti te verzije u odnosu na prethodnu.}

Parseri generisani koristeći ANTLR4 koriste novu tehnologiju koja se naziva \emph{Prilagodljiv LL(*)} (engl. \emph{Adaptive LL(*)}) ili \emph{ALL(*)} \cite{ANTLRReference}, dizajniranu od strane Terensa Para, autora ANTLR-a, i Sema Harvela. \emph{ALL(*)} vrši \emph{dinamičku analizu} gramatike u fazi izvršavanja, dok su starije verzije radile analizu pre pokretanja parsera. Ovaj pristup je takođe efikasniji zbog značajno manjeg prostora ulaznih sekvenci u parser.

Najbolji aspekt ANTLR-a je lakoća definisanja gramatičkih pravila koji opisuju sintaksičke konstrukte. Primer jednostavnog pravila za definisanje aritmetičkog izraza je dat na slici \ref{fig:ANTLRExpressions}. Pošto izraz možemo definisati na više načina, pišemo alternative u definiciji pravila. Pravilo \texttt{exp} je levo rekurzivno jer barem jedna od njegovih alternativnih definicija referiše baš na pravilo \texttt{exp}. ANTLR4 automatski zamenjuje levo rekurzivna pravila u nerekurzivne ekvivalente. Jedini zahtev koji mora biti ispunjen je da leva rekurzija mora biti \emph{neposredna} --- pravila odmah moraju referisati na svoje ime u definiciji. \emph{Posredna} leva rekurzija nije dozvoljena --- pravila ne smeju referisati drugo pravilo takvo da se eventualno kroz rekurziju stigne nazad do pravila od kog se krenulo bez poklapanja sa nekim tokenom.

\begin{figure}[h!]
\begin{lstlisting}[language={}]
exp : (exp)
    | exp '*' exp
    | exp '+' exp
    | INT
    ;
\end{lstlisting}
\caption{Definicija uprošćenog aritmetičkog izraza po ANTLR4 gramatici koristeći neposrednu levu rekurziju.}
\label{fig:ANTLRExpressions}
\end{figure}
