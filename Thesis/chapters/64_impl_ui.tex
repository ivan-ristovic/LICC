\section{Implementacija korisničkog interfejsa}
\label{sec:ImplementationUI}

S obzirom da je aplikacija konzolna (osim dela komponente za vizualizaciju), korisnički interfejs se sastoji od argumenata komandne linije. Pokretanje programa bez argumenata pruža prikaz za pomoć u kome su nabrojane sve moguće opcije. Pomoć koja se pruža korisniku se može videti na slici \ref{fig:UIImpl}.

\begin{figure}[h!]
\centering
\begin{lstlisting}[language={}]
$ ./licc
ERROR(S):
No verb selected.

ast        AST generation commands
cmp        Compare source against the specification source
help       Display more information on a specific command.
version    Display version information.

$ ./licc ast
ERROR(S):
A required value not bound to option name is missing.

-v, --verbose    (Default: false) Verbose output
-t, --tree       (Default: false) Visualize AST tree
-o, --output     Output path
-c, --compact    (Default: false) Compact AST output
--help           Display this help screen.
--version        Display version information.

value pos. 0     Required. Source path

$ ./licc cmp
ERROR(S):
A required value not bound to option name is missing.

-v, --verbose    Set output to verbose messages.
--help           Display this help screen.
--version        Display version information.

value pos. 0     Required. Specification path.
value pos. 1     Required. Test source path.
\end{lstlisting}
\caption{Korisniči interfejs programa pružen kroz argumente komandne linije.}
\label{fig:UIImpl}
\end{figure}

Program zahteva da se kao prvi argument prosledi \emph{glagol} (engl. \emph{verb}) koji će odrediti operaciju koju program treba da izvrši. Od glagola zavisi broj i tip ostalih argumenata u nastavku. Mogući glagoli, sa svojim dodatnim opcijama, su:
\begin{itemize}
    \item \texttt{ast [-v -c -t] source-path [-o output-path]} \\
    Generiše opšti AST za izvorni k\^od na putanji \texttt{source-path} u JSON formatu i ispisuje isti na standardni izlaz, ili u fajl na putanji \texttt{output-path} ako je prisutna opcija \texttt{-o}. Moguće je generisati kompaktan JSON (bez poravnanja) navođenjem opcije \texttt{-c}. Vizualizacija u obliku stabla se prikazuje u novom prozoru ukoliko je prisutna opcija \texttt{-t}.
    \item \texttt{cmp [-v] specification-path test-path} \\
    Generiše opšti AST za izvorne kodove na putanjama \texttt{specification-path} i \texttt{test-path} i poredi generisana stabla. 
\end{itemize} 

Dodatno, svi glagoli podržavaju opciono detaljno logovanje operacija koje program izvršava navođenjem opcije \texttt{-v}. Takođe, navođenjem opcije \texttt{-help} nakon glagola se ispisuje uputstvo specifično za taj glagol. Verzija programa se može proveriti opcijom \texttt{--version}. Opcije \texttt{-o}, \texttt{-c}, \texttt{-t} i \texttt{-v} imaju svoje duže sinonime --- \texttt{--output}, \texttt{--compact}, \texttt{--tree} i \texttt{--verbose}, redom.

