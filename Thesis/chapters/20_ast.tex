\chapter{AST - generisanje i korišćenje}
\label{chp:RelevantTerms}

U ovom poglavlju će biti opisani koncepti i alati čije je razumevanje potrebno kako bi se razumeo opis dalje apstrakcije i implementacije samog programa. Umesto analize samog sadržaja izvornog koda analizira se \emph{apstraktno sintaksičko stablo}, opisano u odeljku \ref{sec:AST}. Kako bi se od izvornog koda došlo do stabla parsiranja a potom i do apstraktnog sintaksičkog stabla, koriste se \emph{lekseri} i \emph{parseri}. S obzirom da je cilj kreirati univerzalnu reprezentaciju, biće neophodno kreirati leksere i parsere za proizvoljne gramatike. Više reči o samom procesu dobijanja stabla parsiranja od izvornog koda i alatima koji mogu da generišu leksere i parsere biće u odeljku \ref{sec:ParsingGrammars}, sa akcentom na alat \emph{Another Tool For Language Recognition} \cite{ANTLR}, u daljem tekstu \emph{ANTLR}. Da bi se dobijena stabla koristila, neophodno je poznavati \emph{obrasce za projektovanje} opisane u odeljku \ref{sec:DesignPatterns} koji pružaju gotova rešenja za česte probleme koji se u ovom radu koriste za pružanje interfejsa obilaska stabala i izračunavanja vrednosti nad istim.

\section{Apstraktna sintaksna stabla - AST}
\label{sec:AST}

Kako bi se od datoteke na fajl sistemu koja sadrži izvorni kod programa 
došlo do izvršivog programa, potrebno je izvršiti više koraka 
\cite{CompilerConstruction}:
\begin{itemize}
    \item pretprocesiranje
    \item prevođenje
    \item asembliranje
    \item linkovanje
\end{itemize}

Ovi koraci će biti opisani na jednom primeru. Pretpostavimo da želimo 
da kompajliramo kod pisan u programskom jeziku C prikazan na slici 
\ref{fig:CompilationProcessInit}. Primetimo, da postoji greška u datom
kodu - simbol \texttt{c} koji se koristi će biti prepoznat kao 
identifikator koji ne odgovara nijednoj promenljivoj. Ovo, doduše, nije
sintaksna greška - izraz \texttt{a + c} je sasvim validan u programskom
jeziku C bez analize konteksta u kom se javlja. Problem će postati 
očigledan tek nakon parsiranja izvornog koda i provere ispunjenosti 
sintaksih pravila. Ovakve greške se nazivaju \emph{semantičke greške}.

\begin{figure}[h!]
    \begin{lstlisting}
    #include<stdio.h>

    #define T int

    int main()
    {
        T a, b;
        a = a + c;        // c nije deklarisano
        printf("%d", a);
        return 0;
    }
    \end{lstlisting}
    \caption{Primer izvornog koda pisanog u programskom jeziku C.}
    \label{fig:CompilationProcessInit}
\end{figure}

U fazi pretprocesiranja se vrše samo tekstualne operacije kao što su
brisanje komentara ili zamena makroa u jezicima kao što je C. Prvo 
mesto gde se vrši analiza sadržaja izvornog fajla je faza prevođenja.
Tu analizu vrši program koji se naziva \emph{pretprocesor}. Rezultat 
rada pretprocesora za kod sa slike \ref{fig:CompilationProcessInit} 
bi izgledao kao na slici \ref{fig:CompilationProcessPrep} \footnote{
U nekim implementacijama C standardne biblioteke, moguće je da se 
poziv funckije \texttt{printf} zameni pozivom funkcije \texttt{fprintf}
sa ispisom na \texttt{stdout}. U standardu se propisuje da funkcije 
kao što je \texttt{printf} mogu biti implementirane kao makroi. Izlaz 
na slici \ref{fig:CompilationProcessPrep} je generisan od strane 
\texttt{GCC 7.4.0} po C11 standardu.} i ovo nije slučaj u datom 
okruženju.

\begin{figure}[h!]
    \begin{lstlisting}
    int main()
    {
        int a, b;
        a = a + c;
        printf("%d", a);
        return 0;
    }
    \end{lstlisting}
    \caption{Rezultat rada pretprocesora za kod sa slike 
             \ref{fig:CompilationProcessInit}.}
    \label{fig:CompilationProcessPrep}
\end{figure}

Prilikom faze prevođenja, kako prevodilac ne bi radio nad sirovim 
karakterima izvornog koda, potrebno je izvršiti pripremu istog. 
Prevodilac ima u vidu moguće elemente programskog jezika, tzv. 
\emph{tokene}, koje treba prepoznati u datom fajlu - ključne reči, 
operatore, promenljive itd. Program koji radi \emph{tokenizaciju} -
prepoznavanje tokena u izvornom fajlu - se naziva \emph{lekser}. 
Pojednostavljen primer tokena koje lekser pokušava da prepozna 
se može videti na slici \ref{fig:CLexerExample}. Primer izlaza
leksera za izlaz pretprocesora sa slike \ref{fig:CompilationProcessPrep}
se može videti na slici \ref{fig:CompilationProcessLex}.

\begin{figure}[h!]
    \begin{lstlisting}
    Identifier : IdentifierNondigit 
                 (IdentifierNondigit | Digit)*
               ;

    IdentifierNondigit : Nondigit
                       | UniversalCharacterName
                       ;

    Nondigit : [a-zA-Z_]
             ;

    Digit : [0-9]
          ;
    \end{lstlisting}
    \caption{Primer delimične definicije tokena za ime promenljive po C11 standardu.}
    \label{fig:CLexerExample}
\end{figure}

\begin{figure}[h!]
    \begin{lstlisting}
    identifier 'main'	 [LeadingSpace]	Loc=<sample.c:3:5>
    l_paren '('		Loc=<sample.c:3:9>
    r_paren ')'		Loc=<sample.c:3:10>
    l_brace '{'	 [StartOfLine]	Loc=<sample.c:4:1>
    int 'int'	 [StartOfLine] [LeadingSpace]	Loc=<sample.c:5:5>
    identifier 'a'	 [LeadingSpace]	Loc=<sample.c:5:9>
    comma ','		Loc=<sample.c:5:10>
    identifier 'b'	 [LeadingSpace]	Loc=<sample.c:5:12>
    semi ';'		Loc=<sample.c:5:13>
    identifier 'a'	 [StartOfLine] [LeadingSpace]	Loc=<sample.c:6:5>
    equal '='	 [LeadingSpace]	Loc=<sample.c:6:7>
    identifier 'a'	 [LeadingSpace]	Loc=<sample.c:6:9>
    plus '+'	 [LeadingSpace]	Loc=<sample.c:6:11>
    identifier 'c'	 [LeadingSpace]	Loc=<sample.c:6:13>
    semi ';'		Loc=<sample.c:6:14>
    identifier 'printf'	 [StartOfLine] [LeadingSpace]	Loc=<sample.c:7:5>
    l_paren '('		Loc=<sample.c:7:11>
    string_literal '"%d"'		Loc=<sample.c:7:12>
    comma ','		Loc=<sample.c:7:16>
    identifier 'a'	 [LeadingSpace]	Loc=<sample.c:7:18>
    r_paren ')'		Loc=<sample.c:7:19>
    semi ';'		Loc=<sample.c:7:20>
    return 'return'	 [StartOfLine] [LeadingSpace]	Loc=<sample.c:8:5>
    numeric_constant '0'	 [LeadingSpace]	Loc=<sample.c:8:12>
    semi ';'		Loc=<sample.c:8:13>
    r_brace '}'	 [StartOfLine]	Loc=<sample.c:9:1>
    eof ''		Loc=<sample.c:9:2>
    \end{lstlisting}
    \caption{Primer delimične definicije tokena za ime promenljive po standardu C11.}
    \label{fig:CompilationProcessLex}
\end{figure}

Nakon završetka rada leksera potrebno je parsirati dobijene tokene.
Parsiranje vrši program koji se naziva \emph{parser}. Parser, slično
kao što lekser ima definicije tokena jezika, mora imati informacije 
o gramatici jezika. Gramatika programskog jezika se najčešće definiše
putem kontekstno-slobodnih gramatika \cite{ContextFreeGrammars}, 
čiji je primer dat na slici \ref{fig:CompilationProcessGram}.

\begin{figure}[h!]
    \begin{lstlisting}
    functionDefinition
        :   declarationSpecifiers? declarator declarationList? compoundStatement
        ;

    declarationList
        :   declaration
        |   declarationList declaration
        ;

    declaration
        :   declarationSpecifiers initDeclaratorList ';'
        | 	declarationSpecifiers ';'
        |   staticAssertDeclaration
        ;
    \end{lstlisting}
    \caption{Isečak gramatike programskog jezika C po standardu C11.}
    \label{fig:CompilationProcessLex}
\end{figure}

Izlaz rada parsera je \emph{stablo parsiranja} (eng. \emph{parse tree} 
ili \emph{derivation tree}). Takvo stablo i dalje sadrži sve relevantne
informacije o izvornom kodu. Vizuelni prikaz rada parsera za gramatiku
sa slike C11 i izvonog koda sa slike \ref{fig:CompilationProcessPrep} je
dat na slici \ref{fig:CompilationProcessPars}.

\begin{figure}[h!]
    \includegraphics{images/}
    \caption{Isečak gramatike programskog jezika C po standardu C11.}
    \label{fig:CompilationProcessLex}
\end{figure}
\section{Parsiranje gramatika programskih jezika}
\label{sec:ParsingGrammars}

Ukoliko imamo gramatiku proizvoljnog programskog jezika, postavlja se pitanje: 
\begin{quote}
    Da li je moguće definisati postupak i zatim napraviti program koji će generisati kodove leksera i parsera napisane u nekom specifičnom programskom jeziku za proizvoljnu gramatiku datu na ulazu?
\end{quote}
Odgovor je potvrdan i postoji veliki broj alata koji se mogu koristiti u ove svrhe, od kojih je navedeno par njih u odeljcima ispod.

\subsection{Lex i Flex}
\label{subsec:LexFlex}
\emph{Lex} \cite{LexYacc} je program koji generiše leksere. Danas se više koristi \emph{flex} \cite{Flex}, kreiran kao alternativa \emph{lex}-u, s obzirom da je i do dva puta brži od lex-a, koristi manje memorije nego lex, i vreme kompilacije leksera koje flex generiše je i do tri puta kraće nego kompilacija leksera koje generiše lex. Pošto flex, isto kao i lex, generiše samo leksere, najčešće se koristi u kombinaciji sa drugim alatima koji mogu da generišu parsere, kao što su npr. \emph{GNU Bison} ili \emph{BYACC}.

\subsection{GNU Bison}
\label{subsec:GNUBison}
\emph{GNU Bison} \cite{GNUBison} je generator parsera i deo GNU projekta \cite{GNUProject}, često referisan samo kao \emph{Bison}. Bison generiše parser na osnovu korisnički definisane kontekstno slobodne gramatike \cite{ContextFreeGrammars}, upozoravajući pritom na dvosmislenosti prilikom parsiranja ili nemogućnost primene gramatičkih pravila. Generisani parser je najčešće C a ređe C++ program, mada se u vreme pisanja ovog rada eksperimentiše sa Java podrškom. Generisani kodovi su u potpunosti prenosivi i ne zahtevaju specifične kompajlere. Bison može da, osim podrazumevanih \emph{LALR(1)} \cite{LALR1} parsera, generiše i kanoničke \emph{LR} \cite{LR}, \emph{IELR(1)} \cite{IELR1} i \emph{GLR} \cite{GLR} parsere.

\subsection{BYACC}
\label{subsec:BYACC}
\emph{Berkeley YACC}, skraćeno \emph{BYACC} \cite{BYACC}, je generator parsera pisan po ANSI C standardu i otvorenog je koda. Posmatra se od strane mnogih kao \textit{najbolja varijanta YACC-a} \cite{LexYacc}. BYACC dozvoljava tzv. \emph{reentrant} k\^od --- omogućava bezbedno konkurentno izvršavanje koda na način kompatibilan sa Bison-om i to je delom razlog njegove popularnosti.

\subsection{ANTLR}
\label{subsec:ANTLR}
\emph{Another Tool for Language Recognition}, ili kraće \emph{ANTLR} \cite{ANTLR}, je generator \emph{LL(*)} \cite{LLStar} leksera i parsera pisan u programskom jeziku Java sa intuitivnim interfejsom za obilazak stabla parsiranja. Verzija $3$ podržava generisanje parsera u jezicima Ada95, ActionScript, C, C\#, Java, JavaScript, Objective-C, Perl, Python, Ruby, i Standard ML, dok verzija $4$ u vreme pisanja ovog rada generiše parsere u narednim jezicima: Java, C\#, C++, JavaScript, Python, Swift i Go.

ANTLR verzije $4$ je izabran u ovom radu zbog svoje popularnosti, jednostavnosti, intuitivnosti i podrške za mnoge moderne programske jezike. Verzija $4$ je izabrana umesto verzije $3$ po preporuci autora ANTLR-a, na osnovu eksperimentalne analize brzine i pouzdanosti te verzije u odnosu na prethodnu. Lekseri i parseri za ulazne gramatike će u implementaciji biti generisani u programskom jeziku C\#.

Parseri generisani koristeći ANTLR koriste novu tehnologiju koja se naziva \emph{Prilagodljiv LL(*)} (engl. \emph{Adaptive LL(*)}) ili \emph{ALL(*)} \cite{ANTLRReference}, dizajniranu od strane Terensa Para, autora ANTLR-a, i Sema Harvela. \emph{ALL(*)} vrši \emph{dinamičku analizu} gramatike u fazi izvršavanja, dok su starije verzije radile analizu pre pokretanja parsera. Ovaj pristup je takođe efikasniji zbog značajno manjeg prostora ulaznih sekvenci u parser.

Najbolji aspekt ANTLR-a je lakoća definisanja gramatičkih pravila koji opisuju sintaksne konstrukte. Primer jednostavnog pravila za definisanje aritmetičkog izraza je dat na slici \ref{fig:ANTLRExpressions}. Pravilo \texttt{exp} je levo rekurzivno jer barem jedna od njegovih alternativnih definicija referiše baš na pravilo \texttt{exp}. ANTLR4 automatski zamenjuje levo rekurzivna pravila u nerekurzivne ekvivalente. Jedini zahtev koji mora biti ispunjen je da levo rekurzivna pravila moraju biti \emph{direktna} --- da pravila odmah referišu sama sebe. Pravila ne smeju referisati drugo pravilo sa leve strane definicije takvo da se eventualno kroz rekurziju stigne nazad do pravila od kog se krenulo bez poklapanja sa nekim tokenom.

\begin{figure}[h!]
\begin{lstlisting}[language={}]
exp : (exp)
    | exp '*' exp
    | exp '+' exp
    | INT
    ;
\end{lstlisting}
\caption{Definicija uprošćenog aritmetičkog izraza po ANTLR4 gramatici.}
\label{fig:ANTLRExpressions}
\end{figure}


\subsubsection{Preduslovi za pokretanje ANTLR4}
\label{subsubsec:ANTLRInstallation}

Kako bi se ANTLR koristio, potrebno je instalirati ANTLR i imati \emph{Java Runtime Environment} (skr. \emph{JRE}) instaliran na sistemu i dostupan globalno pokretanjem putem komande \texttt{java}. Instalacija se sastoji od preuzimanja najnovijeg \emph{.jar} fajla\footnote{Takođe je moguće prevesti izvorni k\^od dostupan na servisu GitHub \url{https://github.com/antlr/antlr4}}, sa zvanične stranice \cite{ANTLR} ili recimo korišćenjem \emph{curl} alata: 
\begin{lstlisting}[language={}]
$ curl -O http://www.antlr.org/download/antlr-4-complete.jar
\end{lstlisting}

Na UNIX sistemima moguće je kreirati alias \texttt{antlr4} ili \emph{shell} skript unutar direktorijuma \texttt{/usr/local/bin} sa imenom \texttt{antlr4} koji će pokrenuti \emph{.jar} fajl na sledeći način (pretpostavljajući da se \emph{.jar} fajl nalazi u direktorijumu \texttt{/usr/local/lib}):
\begin{lstlisting}[language={}]
#!/bin/sh
java -cp "/usr/local/lib/antlr4-complete.jar:$CLASSPATH" org.antlr.v4.Tool $*
\end{lstlisting}

Na Windows sistemima moguće je kreirati \emph{batch} skript sa imenom \texttt{antlr4.bat} koji će pokrenuti ANTLR4, na sledeći način (pretpostavljajući da se \emph{.jar} fajl nalazi u direktorijumu \texttt{C:\textbackslash{}lib}):
\begin{lstlisting}[language={}]
java -cp C:\lib\antlr-4-complete.jar;%CLASSPATH% org.antlr.v4.Tool %*
\end{lstlisting}

Ukoliko su aliasi ili skript fajlovi imenovani kao iznad, moguće je iz komandne linije pojednostavljeno pokretati ANTLR4:  
\begin{lstlisting}[language={}]
$ antlr4
ANTLR Parser Generator Version 4.0
-o ___    specify output directory where all output is generated
-lib ___  specify location of .tokens files
...
\end{lstlisting}

Dodatno, za Unix sisteme\footnote{Za Windows operativni sistem je moguće kreirati \emph{batch} skript po opisu na \url{https://github.com/antlr/antlr4/blob/master/doc/getting-started.md}.}, moguće je kreirati dodatni alias \texttt{grun} (ili alternativno, kreirati \texttt{shell script}) za biblioteku \texttt{TestRig}. Biblioteka \texttt{TestRig} se može koristiti za brzo testiranje parsera --- moguće je pokrenuti parser od bilo kog pravila i dobiti izlaz parsera u raznim formatima. \texttt{TestRig} dolazi uz ANTLR \texttt{.jar} fajl i moguće je napraviti prečicu za brzo pokretanje (nalik na ANTLR alias):
\begin{lstlisting}[language={}]
$ alias grun='java -cp "/usr/local/lib/antlr-4-complete.jar:$CLASSPATH" org.antlr.v4.gui.TestRig'
\end{lstlisting}


\subsection{Generisanje parsera koristeći ANTLR4}
\label{subsec:ANTLRParserGeneration}

U ovom odeljku će biti opisan proces kreiranja interfejsa za parsiranje programa pisanih u pseudo-programskom jeziku (u nastavku \emph{pseudo-jezik}), nalik na pseudokod. Ovako dobijeni interfejs će moći da se koristi u opšte svrhe, a za potrebe ovog rada će se koristiti za generisanje apstraktnog sintaksičkog stabla za izvorni k\^od pisan u pseudo-jeziku.

Definišimo gramatiku pseudo-jezika prateći ANTLR pravila za definisanje gramatika. Kao i za svaki drugi programski jezik, treba podržati neke osnovne koncepte: \emph{identifikatore}, \emph{izraze}, \emph{naredbe}, \emph{funkcije} i slično. Za sada se fokusirajmo na naredbe, kao samostalne izvršive jedinice k\^oda. Stoga program možemo posmatrati kao niz naredbi. U nekim slučajevima će biti potrebno definisanje kompleksnih naredbi koje se sastoje od više drugih naredbi, i ovakve složene naredbe ćemo zvati \emph{blok} ili \emph{blok naredbi}. Stoga, radi konzistentnosti, program će biti blok naredbi. Kako bismo označili da su naredbe deo bloka naredbi, koristićemo reči \texttt{begin} i \texttt{end}, osim ukoliko je reč o samo jednoj naredbi. Ovakve situacije rešavamo definisanjem \emph{alternativa} u definiciji pravila --- više definicija razdvojenih simbolom \texttt{|}. Specijalne reči kao što su \texttt{begin} i \texttt{end} će biti rezervisane reči našeg pseudo-jezika, tzv. \emph{ključne reči}. Na slici \ref{fig:PseudoDef1} se može videti definicija programa\footnote{Drugim rečima, jedan program u pseudo-jeziku je jedinica prevođenja, pa je zato pravilo nazvano \emph{unit}.} i bloka naredbi pseudo-jezika, pri čemu se ključne reči u pravilima navode između apostrofa. ANTLR dozvoljava jednostavne definicije pravila u kojima figuriše promenljiv broj drugih pravila, pri čemu se koriste simboli kao u regularnim izrazima\footnote{U regularnim izrazima, simbol \texttt{a?} označava opciono pojavljivanje simbola \texttt{a}, simbol \texttt{a+} označava jedno ili više pojavljivanja simbola \texttt{a}, a simbol \texttt{a*} označava proizvoljan broj pojavljivanja simbola \texttt{a} --- kombinacija simbola \texttt{?} i \texttt{+}.}, što je iskorišćeno za definiciju pravila bloka naredbi. \texttt{NAME} je identifikator koji predstavlja ime programa. Identifikatore ćemo definisati kasnije, za sada možemo posmatrati identifikator kao nisku karaktera s tim što će postojati restrikcije vezane za to koji karakteri se mogu naći unutar identifikatora.

\begin{figure}[h!]
\begin{lstlisting}[language={}]
unit
    : 'algorithm' NAME block EOF
    ;
block
    : 'begin' statement+ 'end'
    | statement
    ;
\end{lstlisting}
\caption{Definicija jedinice prevođenja i bloka naredbi za pseudo-jezik.}
\label{fig:PseudoDef1}
\end{figure}

Sledeći korak je definisanje naredbi pseudo-jezika. Slično kao i u drugim programskim jezicima, potrebno je podržati koncept deklaracije promenljive, dodele vrednosti izraza promenljivoj, naredbe kontrole toka --- grananje i petlje. Na slici \ref{fig:PseudoDef2} je definisano šta se sve smatra jednom naredbom. Naredbe mogu biti i prazne, što je označeno ključnom rečju \texttt{pass}. Iz definicije sa slike se jasno vidi šta sve može biti naredba (prateći redosled alternativa pravila): deklaracija, dodela, poziv funkcije (označen kao \texttt{cexp}, skraćeno od \emph{function call expression})\footnote{Funkcije mogu vratiti vrednosti pa se stoga njihovi pozivi mogu naći u izrazima --- dakle poziv funkcije je validan izraz (stoga \texttt{expression} u imenu \texttt{function call expression}). Naravno, ta vrednost se može ignorisati ili pak sama funkcija može biti takva da nema povratnu vrednost već je samo neophodno izvršiti je zbog sporednih efekata.}, vraćanje vrednosti izraza (ključna vrednost \texttt{return}) iz funkcije, prekidanje izvršavanja davanjem poruke o grešci, naredba grananja, \emph{while} petlja, \emph{repeat-until} petlja i inkrementiranje/dekrementiranje vrednosti promenljive.
    
\begin{figure}[h!]
\begin{lstlisting}[language={}]
statement
    : 'pass'
    | declaration
    | assignment
    | cexp
    | 'return' exp
    | 'error' STRING
    | 'if' exp 'then' block ('else' block)? 
    | 'while' exp 'do' block 
    | 'repeat' block 'until' exp
    | ('increment' | 'decrement') var	
    ;
\end{lstlisting}
\caption{Definicija naredbe za pseudo-jezik.}
\label{fig:PseudoDef2}
\end{figure}

Deklaracija, prikazana na slici \ref{fig:PseudoDef3}, uvodi pojavljivanje simbola sa identifikatorom \texttt{NAME} kao oznaku za promenljivu, funkciju ili proceduru --- funkciju bez povratne vrednosti. Svaka promenljiva mora biti određenog tipa, što se postiže pravilom \texttt{type}. Promenljivoj se, opciono, može pridružiti početna vrednost, drugim rečima promenljiva se može \emph{inicijalizovati} tako da joj se pridruži vrednost nekog izraza. Procedure i funkcije imaju opcione parametre, vrednosti izraza koje im se prosleđuju kasnije u pozivu kao argumenti. Lista parametara, takođe prikazana na slici \ref{fig:PseudoDef3}, se navodi kao lista proizvoljno mnogo parova \texttt{NAME : type}, što se vidi iz definicije pravila \texttt{parlist}.

\begin{figure}[h!]
\begin{lstlisting}[language={}]
declaration
    : 'declare' type NAME ('=' exp)? 
    | 'procedure' NAME '(' parlist? ')' block 
    | 'function' NAME '(' parlist? ')' 'returning' type block 
    ;
parlist
    : NAME ':' type (',' NAME ':' type)*
    ;
\end{lstlisting}
\caption{Definicija deklaracije za pseudo-jezik.}
\label{fig:PseudoDef3}
\end{figure}

Identifikatori su niske karaktera koje predstavljaju oznaku koja odgovara određenoj memorijskoj adresi. Identifikatori se koriste umesto sirovih vrednosti adresa kako bi k\^od bio čitljiviji i lakši za pisanje --- na nivou asemblera se većinom koriste adrese ili automatski generisane oznake. Na slici \ref{fig:PseudoDef4} se može videti definicija identifikatora. Identifikator se sastoji od slova, cifara i simbola \texttt{\_}, s tim što ne sme početi cifrom. Ovo je konvencija koju prati dosta jezika, uključujući programski jezik C. Primetimo da je identifikator nešto što bi lekser trebalo da prepozna tokom tokenizacije. Međutim, kada definišemo gramatiku od koje će ANTLR praviti lekser i parser, možemo i tokene definisati na isti način kao i gramatička pravila dajući regularni izraz za njihovo poklapanje. Listovi stabla parsiranja su uvek tokeni, drugim rečima se nazivaju i \emph{terminalni simboli}. Tokeni se, osim u listovima, mogu naći bilo gde u stablu parsiranja.

\begin{figure}[h!]
\begin{lstlisting}[language={}]
NAME
    : [a-zA-Z_][a-zA-Z_0-9]*
    ;
\end{lstlisting}
\caption{Definicija identifikatora za pseudo-jezik.}
\label{fig:PseudoDef4}
\end{figure}

Pošto želimo da pseudo-jezik bude strogo tipiziran, potreban je koncept tipa (videti definiciju deklaracije), čija je definicija data na slici \ref{fig:PseudoDef5}. Tip može biti \emph{primitivan} (drugim rečima \emph{prost}) ili \emph{složen}. Primitivni tipovi su podržani u samoj sintaksi jezika --- u našem slučaju brojevni tipovi i niske. Brojevi mogu biti celi ili realni. U složene tipove spadaju korisnički definisani tipovi (sa imenom \texttt{NAME}, u četvrtoj alternativi pravila \texttt{typename} sa slike \ref{fig:PseudoDef5}) i kolekcije. Od kolekcija su podržani nizovi, liste i skupovi. Prilikom definicije kolekcije mora se navesti tip elemenata kolekcije i taj tip mora biti uniforman --- isti za sve elemente kolekcije. 

\begin{figure}[h!]
\begin{lstlisting}[language={}]
type 
    : typename 'array'?
    | typename 'list'?
    | typename 'set'?
    ;
typename 
    : 'integer' 
    | 'real' 
    | 'string' 
    | NAME 
    ;
\end{lstlisting}
\caption{Definicija tipa podataka za pseudo-jezik.}
\label{fig:PseudoDef5}
\end{figure}

Izrazi, iako definisani rekurzivno, se mogu posmatrati kao kombinacija promenljivih, operatora i poziva funkcija sa odlikom da se mogu \emph{evaluirati}, tj. moguće je izračunati njihovu vrednost. Iz definicije pravila \texttt{exp} sa slike \ref{fig:PseudoDef6}, mogu se uočiti tipovi izraza, pri čemu nije vođeno računa o matematičkom prioritetu operatora, radi jednostavnosti. Izraz može biti \emph{literal}, koji predstavlja konstantu, bilo brojevnu, logičku ili nisku karaktera. Promenljive, definisane pravilom \texttt{var} su takođe izrazi, jer se trenutna vrednost promenljive posmatra kao vrednost izraza. Primetimo da promenljiva može biti kolekcijskog tipa, u kom slučaju se navodi redni broj elementa nakon identifikatora promenljive --- taj redni broj može biti rezultat evaluacije drugog izraza, ali ne bilo kakvog, stoga se u pravilu \texttt{iexp} definiše šta sve može biti korišćeno da se indeksira element kolekcije. Izrazima se može dati prioritet pomoću zagrada, što se vidi u trećoj alternativi pravila \texttt{exp}. U naredne tri alternative su opisani tipovi izraza: aritmetički, relacioni i logički. Aritmetički izrazi su vezani aritmetičkim operatorima definisanim preko pravila \texttt{aop}, slično važi i za ostala dva tipa. Svi tipovi izraza navedeni iznad su binarni, što znači da operatori zahtevaju dva argumenta. Postoje i unarni izrazi, od kojih su podržane promena znaka i logička negacija, što se vidi iz pravila \texttt{uop}.

\begin{figure}[h!]
\begin{lstlisting}[language={}]
exp
    : literal 
    | var
    | '(' exp ')'
    | exp aop exp
    | exp rop exp
    | exp lop exp
    | uop exp
    | cexp
    ;
var 
    : NAME ('[' iexp ']')?
    ;
iexp 
    : literal
    | var
    | aexp
    ;
cexp
    : 'call' NAME '(' explist? ')'
    ;
explist
    : exp (',' exp)*
    ;
aop : '+' | '-' | '*' | '/' | 'div' | 'mod' ;
rop : '>' | '>=' | '<' | '<=' | '==' | '=/=' ;
lop : 'and' | 'or' ;
uop : '-' | 'not' ;
\end{lstlisting}
\caption{Definicija izraza za pseudo-jezik.}
\label{fig:PseudoDef6}
\end{figure}

Definicija literala je prikazana na slici \ref{fig:PseudoDef7}. Literali mogu bili istinitosne konstante \texttt{True} i \texttt{False}, brojevne konstante ili niske karaktera. Brojevne konstante mogu bili celobrojni ili realni dekadni brojevi. Realne konstatne je moguće definisati u fiksnom ili pokretnom zarezu. Niske se mogu definisati između navodnika ili apostrofa. Pritom, kao i u modernim programskim jezicima, moguće je navesti sekvence koje predstavljaju specijalne karaktere kao što su novi red, tabulator itd. Oznaka \texttt{fragment} označava optimizaciju, naime nije potrebno da postoji na primer pravilo \texttt{Digit}, već samo dajemo simbol za regularni izraz koji će se koristiti u više drugih pravila i poklapati jednu dekadnu cifru.

\begin{figure}[h!]
\begin{lstlisting}[language={}]
literal : 'True' | 'False' | INT | FLOAT | STRING ;
STRING : '"' ( EscapeSequence | ~('\\'|'"') )* '"'  ;
INT : Digit+ ;
FLOAT
    : Digit+ '.' Digit* ExponentPart?
    | '.' Digit+ ExponentPart?
    | Digit+ ExponentPart
    ;

fragment
ExponentPart : [eE] [+-]? Digit+ ;
fragment
Digit : [0-9] ;
fragment
EscapeSequence : '\\' [abfnrtvz"'\\] | '\\' '\r'? '\n' ;
\end{lstlisting}
\caption{Definicija konstanti za pseudo-jezik.}
\label{fig:PseudoDef7}
\end{figure}

Poslednje što treba definisati je sve ono što lekser treba da preskoči tokom prolaska kroz izvorni k\^od programa. To su beline (nevidljivi karakteri kao što su razmaci, tabulatori i novi redovi) i komentari. Definicije ovih pravila se mogu videti na slici \ref{fig:PseudoDef8}. Vidimo da se u njima koristi posebna oznaka \texttt{-> skip}, koja predstavlja instrukcije lekseru da preskoči sve ono što ovo pravilo poklopi. Komentari su u stilu kao u programskom jeziku C (ali naravno, isti stil se koristi i u mnogim jezicima) i mogu biti jednolinijski ili višelinijski. Beline koje treba preskočiti su definisane u pravilu \texttt{WS}, skraćeno od \emph{whitespace}, što u prevodu sa engleskog znači \emph{beli prostor, belina}.

\begin{figure}[h!]
\begin{lstlisting}[language={}]
BlockComment
    :   '/*' .*? '*/'  -> skip
    ;
LineComment
    :   '//' ~[\r\n]*  -> skip
    ;
WS  
    : [ \t\u000C\r\n]+ -> skip
    ;
\end{lstlisting}
\caption{Definicija komentara i belina za pseudo-jezik.}
\label{fig:PseudoDef8}
\end{figure}

\begin{figure}[h!]
\begin{lstlisting}[language={}]
grammar Pseudo;
\end{lstlisting}
\caption{Definicija imena gramatike za pseudo-jezik.}
\label{fig:PseudoDef9}
\end{figure}

Ovako definisanu gramatiku možemo sačuvati u fajl sa imenom \texttt{Pseudo.g4}, potrebno je samo navesti ime gramatike na početku fajla, kao na slici \ref{fig:PseudoDef9}. Naredni korak je kreiranje leksera i parsera koristeći ANTLR4, predpostavljajući da je instaliran na način opisan u \ref{subsec:ANTLRInstallation}. Pokretanjem ANTLR-a generišemo lekser i parser za gramatiku pseudo-jezika:
\begin{lstlisting}[language={}]
$ antlr4 Pseudo.g4
\end{lstlisting}

ANTLR4 će generisati lekser i parser podrazumevano napisane u programskom jeziku Java. Ukoliko želimo to da promenimo, možemo koristiti opciju \texttt{-Dlanguage=...}. Kako bismo testirali generisani lekser i parser, možemo koristiti ANTLR \texttt{TestRig} da vizualno prikažemo stablo parsiranja, s tim što moramo prvo kompajlirati generisane Java k\^odove. \texttt{TestRig} pozivamo navođenjem ime gramatike (koje se poklapa sa imenom leksera i parsera) i imenom pravila od koga će parser krenuti. Opcija \texttt{-gui} pokreće vizualni prikaz stabla parsiranja pokazan na slici \ref{fig:PseudoTreeGui} (vizualni prikaz je moguće preskočiti i samo ispisati stablo u LISP formi koristeći opciju \texttt{-tree}), mada je moguće i ispisati samo tokene koristeći opciju \texttt{-tokens}. Ulaz se prosleđuje programu dok se ne naiđe na simbol \texttt{EOF}, ili alternativno se može preneti ulaz putem UNIX pipeline-a (na slici \ref{fig:PseudoTreeGui} se može videti izlaz koji se dobija korišćenjem opcije \texttt{-gui}):
\begin{lstlisting}[language={}]
$ javac *.java
$ echo "declare integer x = 5" | grun Pseudo declaration -tokens
[@0,0:6='declare',<'declare'>,1:0]
[@1,8:14='integer',<'integer'>,1:8]
[@2,16:16='x',<NAME>,1:16]
[@3,18:18='=',<'='>,1:18]
[@4,20:20='5',<INT>,1:20]
[@5,22:21='<EOF>',<EOF>,2:0]
$ echo "declare integer x = 5" | grun Pseudo declaration -tree
(declaration declare (type (typename integer)) x = (exp (literal 5)))
$ echo "declare integer x = 5" | grun Pseudo declaration -gui
\end{lstlisting}    

\begin{figure}[h!]
\centering
\includegraphics[scale=0.8]{images/pseudo_parse_tree.png}
\caption{Grafički prikaz stabla parsiranja koje generiše parser kreiran od strane \texttt{TestRig} biblioteke za k\^od pisan u pseudo-jeziku.}
\label{fig:PseudoTreeGui}
\end{figure}


\subsection{Obilazak stabla parsiranja}
\label{subsec:ANTLRParserIntegration}

ANTLR, osim leksera i parsera za datu gramatiku, može da kreira interfejse i bazne klase koji prate obrasce za projektovanje \emph{posetilac} (engl. \emph{visitor}) i osluškivač (engl. \emph{listener}\footnote{Osluškivač je varijanta obrasca \emph{posmatrač} (engl. \emph{observer})} opisane u \ref{sec:DesignPatterns}. Tako kreirani interfejsi i klase imaju metode za obilazak stabla parsiranja. ANTLR podrazumevano generiše interfejs osluškivača (slika \ref{fig:ANTLRListener}) kao i baznu klasu koja implementira generisani interfejs tako što su sve implementirane metode prazne. Stoga, ukoliko korisnik želi da definiše operaciju samo u slučaju da se prilikom obilaska stabla parsiranja naiđe na određeni tip čvora, nije potrebno implementirati ceo interfejs osluškivača već je moguće naslediti baznu klasu i predefinisati samo jedan metod. ANTLR može da generiše i posetilac (slika \ref{fig:ANTLRVisitor}), ukoliko se navede odgovarajuća opcija \texttt{-visitor} prilikom pokretanja. Slično, ukoliko nije potrebno generisati osluškivač, može se koristiti opcija \texttt{-no-listener}.

\begin{figure}[h!]
\begin{lstlisting}
public interface IPseudoListener : IParseTreeListener
{
    void EnterUnit([NotNull] PseudoParser.UnitContext context);
    void ExitUnit([NotNull] PseudoParser.UnitContext context);
    void EnterBlock([NotNull] PseudoParser.BlockContext context);
    void ExitBlock([NotNull] PseudoParser.BlockContext context);
    void EnterStatement([NotNull] PseudoParser.StatementContext context);
    void ExitStatement([NotNull] PseudoParser.StatementContext context);
    
    ...
}
\end{lstlisting}
\caption{Delimični prikaz interfejsa osluškivača generisanog od strane ANTLR4 za pseudo-jezik definisan u prethodnom odeljku (C\#).}
\label{fig:ANTLRListener}
\end{figure}

Sa slike \ref{fig:ANTLRListener} se vidi da je moguće definisati metode koje će se pozivati prilikom ulaska ali i prilikom izlaska iz čvora određenog tipa prilikom obilaska stabla parsiranja. Pritom je važno kako se stablo obilazi. U slučaju ANTLR, to je pretraga u dubinu (engl. \emph{depth-first search, DFS})\footnote{DFS je obilazak stabla takav da se obilazak duž grane stabla nastavlja sve dok je moguće ići dublje, a ako to nije moguće vratiti se unazad i obići druge grane.}, stoga će se metod \texttt{Exit} za proizvoljni čvor pozvati tek kad se obiđu sva deca tog čvora --- dakle nakon poziva njihovih \texttt{Enter} i \texttt{Exit} metoda. Pošto se DFS obično implementira putem LIFO strukture\footnote{\emph{Last In, First Out} struktura podataka je apstraktna struktura podataka sa operacijama ubacivanja i izbacivanja elemenata, pri čemu je element koji se izbacuje onaj koji je poslednji ubačen. Primer LIFO strukture je držač za CD-ove --- ne mogu se ukloniti CD-ovi ispod CD-a na vrhu (poslednji ubačen) a da se ne ukloni isti. U slučaju opisanom iznad, implementacija LIFO strukture se naziva stek \emph{stack}.}, može se reći da se \texttt{Enter} metod poziva onog trenutka kad se čvor ubaci u strukturu, a \texttt{Exit} metod onda kada se čvor ukloni iz strukture.

\begin{figure}[h!]
\begin{lstlisting}
public interface IPseudoVisitor<T> : IParseTreeVisitor<T>
{
    T VisitUnit([NotNull] PseudoParser.UnitContext context);
    T VisitBlock([NotNull] PseudoParser.BlockContext context);
    T VisitStatement([NotNull] PseudoParser.StatementContext context);
    T VisitDeclaration([NotNull] PseudoParser.DeclarationContext context);
    
    ...
}
\end{lstlisting}
\caption{Delimični prikaz interfejsa posetioca generisanog od strane ANTLR4 za pseudo-jezik definisan u prethodnom odeljku (C\#).}
\label{fig:ANTLRVisitor}
\end{figure}

Za razliku od osluškivača, posetilac je prirodnije koristiti ukoliko je potrebno izvršiti neko izračunavanje nad strukturom koja se obilazi. Interfejs posetioca (slika \ref{fig:ANTLRVisitor}) je šablonski, i metodi imaju povratnu vrednost šablonskog tipa za razliku od metoda osluškivača i, u odnosu na osluškivač, nema para metoda za svaki čvor već samo jedan metod. Dodatna razlika, ali i najveća, je ta što se metodi posetioca ne pozivaju automatski. Stoga je na programeru da nastavi obilazak i da odluči u koje čvorove želi da se spusti. Jasno je da i osluškivač i posetilac imaju svoje primene --- ukoliko je potrebno obići stablo parsiranja i dovući neke informacije može se iskoristiti osluškivač jer onda ne moramo brinuti o obilasku. S druge strane, ukoliko je potrebno izračunati neku vrednost prirodno je iskoristiti rekurziju i iskoristiti posetilac --- rekurzivni pozivi prilikom obilaska nam idu u prilog jer koristimo povratne vrednosti tih metoda da gradimo rezultat od listova ka korenu stabla parsiranja. U nastavku će se koristiti posetilac zbog kontrole obilaska ali i činjenice da se stablo parsiranja obilazi sa ciljem da se izgradi AST, koji je takođe rekurzivna struktura i gradi se inkrementalno kroz rekurziju.

Bilo da se koristi osluškivač ili posetilac, potrebno je nekako proslediti informacije o samom čvoru na koji se naišlo tokom obilaska stabla parsiranja. Te informacije se metodima osluškivača i posetioca prosleđuju putem potklasa apstrakne klase konteksta pravila \texttt{ParserRuleContext} --- u primeru iznad \texttt{UnitContext}, \texttt{BlockContext} itd. Svaki kontekst pravila po imenu odgovara pravilima definisanim u gramatici i sadrži informacije bitne za trenutni čvor u stablu parsiranja koji odgovara tipu konteksta. Takođe, u svakom kontekstu su prisutne i metode čija imena odgovaraju pravilima koja se javljaju u definiciji samog pravila koje odgovara kontekstu. Tako da, za \texttt{BlockContext}, imajući u vidu definiciju sa slike \ref{fig:PseudoDef1}, pošto se u definiciji osim tokena koristi i pravilo \texttt{statement}, u okviru \texttt{BlockContext} klase biće implementiran i metod \texttt{statement()} koji vraća kontekst pravila u ovom slučaju tipa \texttt{StatementContext[]} jer u prvoj alternativi stoji \texttt{statement+} --- dakle možemo imati više \texttt{statement} poklapanja. Sa ovim u vidu, moguće je odrediti kako će se obilazak nastaviti (u slučaju posetioca) ili dovući informacije o delovima definicije pravila. Ukoliko pravilo ima više alternativa, metodi koje vraćaju kontekst pravila koje figuriše u alternativi koja nije korišćena za poklapanje pravila će vratiti \texttt{null}. Pošto se \texttt{statement()} pravilo javlja u obe alternative pravila \texttt{block} (i nije opciono), možemo biti sigurni da povratna vrednost \texttt{statement()} metoda nikada neće biti \texttt{null}.

U poglavlju \ref{chp:MyAST} će se koristiti posetilac za obilazak stabla parsiranja i kreiranje AST apstrakcije od istog. Pritom, koristiće se implementacija posetioca u programskom jeziku C\#. 

\section{Korišćenje generisanih stabala}
\label{sec:DesignPatterns}

Kako bi se stabla parsiranja i apstraktna sintaksička stabla mogla koristiti, potrebno je pružiti i uniformni interfejs za njihov obilazak. Postoje situacije kada se stablo obilazi sa ciljem izvršavanja operacija prilikom ulaska ili izlaska iz čvorova određenog tipa, ili pak sa ciljem izračunavanja neke konkretne vrednosti. Prilikom razvoja softvera se često nailazi na ovakve probleme i stoga su kreirana ponovno upotrebljiva rešenja za te probleme. 

\emph{Obrasci za projektovanje} (engl. \emph{design patterns} \cite{DesignPatternsBook}, drugačije nazvani i \emph{projektni šabloni, uzorci}) predstavljaju opšte i ponovno upotrebljivo rešenje čestog problema, obično implementirani kroz koncepte objektno-orijentisanog programiranja. Svaki obrazac za projektovanje ima četiri osnovna elementa:
\begin{itemize}
    \item ime --- ukratko opisuje problem, rešenje i posledice,
    \item problem --- opisuje slučaj u kome se obrazac koristi,
    \item rešenje --- opisuje elemente dizajna i odnos tih elemenata,
    \item posledice --- obuhvataju rezultate i ocene primena obrasca.
\end{itemize}

Obrasce za projektovanje je moguće grupisati po situaciji u kojoj se mogu iskoristiti ili načinu na koji rešavaju zadati problem. Stoga je opšte prihvaćena podela na tri grupe:
\begin{itemize}
    \item \emph{gradivni obrasci} (engl. \emph{creational patterns}),
    \item \emph{strukturni obrasci} (engl. \emph{structural patterns}),
    \item \emph{obrasci ponašanja} (engl. \emph{behavioral patterns}).
\end{itemize}

Gradivni obrasci apstrahuju proces pravljenja objekata i važni su kada sistemi više zavise od sastavljanja objekata nego od nasleđivanja. Neki od najvažnijih gradivnih obrazaca su \emph{apstraktna fabrika} (engl. \emph{abstract factory}), \emph{graditelj} (engl. \emph{builder}), \emph{proizvodni metod} (engl. \emph{factory method}), \emph{prototip} (engl. \emph{prototype}) i \emph{unikat} (engl. \emph{singleton}). Strukturni obrasci se bave načinom na koji se klase i objekti sastavljaju u veće strukture. Neki od najvažnijih strukturnih obrazaca su \emph{adapter} (engl. \emph{adapter}), \emph{most} (engl. \emph{bridge}), \emph{sastav} (engl. \emph{composite}), \emph{dekorater} (engl. \emph{decorator}), \emph{fasada} (engl. \emph{facade}), \emph{muva} (engl. \emph{flyweight}) i \emph{proksi} (engl. \emph{proxy}). Obrasci ponašanja se bave načinom na koji se klase i objekti sastavljaju u veće strukture. Neki od najvažnijih strukturnih obrazaca su \emph{lanac odgovornosti} (engl. \emph{chain of responsibility}), \emph{komanda} (engl. \emph{command}), \emph{interpretator} (engl. \emph{interpreter}), \emph{iterator} (engl. \emph{iterator}), \emph{posmatrač} (engl. \emph{observer}), \emph{strategija} (engl. \emph{strategy}) i \emph{posetilac} (engl. \emph{visitor}).

Za potrebe ovog rada, obrasci za projektovanje će se koristiti kao opšte prihvaćeno i programerski intuitivno rešenje određenih problema. Takođe, u kontekstu stabala parsiranja i apstraktnih sintaksičkih stabala, obrasci \emph{Posmatrač} i \emph{Posetilac} su od velikog značaja jer pružaju interfejs za obilazak takvih stabala. Ovi obrasci, opisani u narednim odeljcima, se koriste od strane ANTLR alata. Takođe, s obzirom da su ovi obrasci opšte-prihvaćeno rešenje za pružanje interfejsa obilaska stabala, biće korišćeni i u implementaciji opšte apstrakcije. U nastavku će zbog opisanih razloga biti opisani samo obrasci posmatrač i posetilac, dok zainteresovani čitalac može pročitati više u \cite{DesignPatternsBook}.

\subsection{Obrazac "Posmatrač"}
\label{subsec:DesignPatternsObserver}

Obrazac za projektovanje \emph{Posmatrač} je obrazac ponašanja koji se koristi kada je potrebno definisati jedan-ka-više vezu između objekata tako da ukoliko jedan objekat promeni stanje (subjekat) svi zavisni objekti su obavešteni o izmeni i shodno ažurirani. Posmatrač predstavlja \emph{pogled} (engl. \emph{View}) u MVC (engl. \emph{Model-View-Controller}) arhitekturi. Na slici \ref{fig:UMLObserver} se može videti UML dijagram \cite{UML} ovog obrasca. 

\begin{figure}[h!]
\centering
\includegraphics[scale=0.8]{images/observer.jpg}
\caption{UML dijagram obrasca za projektovanje "Posmatrač". Preuzeto sa \url{https://sourcemaking.com/design_patterns/observer}.} 
\label{fig:UMLObserver}
\end{figure}

Primer upotrebe ovog obrasca može biti aukcija gde je aukcionar subjekat i započinje aukciju, dok učesnici aukcije (objekti) posmatraju aukcionera i reaguju na podizanje cene. Prihvatanje promene cene menja trenutnu cenu i aukcioner oglašava promenu iste, a svi učesnici aukcije dobijaju informaciju da se izmena izvršila. Za potrebe ovog rada, primer upotrebe može biti obilazak stablolike kolekcije (recimo stabla parsiranja) i obaveštavanje o nailasku na čvorove određenih tipova. Te informacije se dalje mogu iskoristiti za izračunavanja nad pomenutom strukturom ili generisanje novih struktura (recimo AST). 

\subsection{Obrazac "Posetilac"}
\label{subsec:DesignPatternsListener}

Obrazac za projektovanje \emph{Posetilac} je obrazac ponašanja koji predstavlja operaciju koju je potrebno izvesti nad elementima objektne strukture. Posetilac omogućava definisanje nove operacije bez izmena klasa elemenata nad kojima operiše. Operacija koja će se izvesti zavisi od imena zahteva, tipa posetioca i tipa elementa kog posećuje. Na slici \ref{fig:UMLVisitor} se može videti UML dijagram \cite{UML} ovog obrasca. 

\begin{figure}[h!]
    \centering
    \includegraphics[scale=0.8]{images/visitor.jpg}
    \caption{UML dijagram obrasca za projektovanje "Posetilac". Preuzeto sa \url{https://sourcemaking.com/design_patterns/visitor}} 
    \label{fig:UMLVisitor}
\end{figure}

Primer upotrebe ovog obrasca može biti operisanje taksi kompanija. Kada osoba pozove taksi kompaniju (prihvatanje posetioca), kompanija šalje vozilo osobi koja je pozvala kompaniju. Nakon ulaska u vozilo (posetilac), mušterija ne kontroliše svoj transport već je to u rukama taksiste (posetioca). Za potrebe ovog rada, primer upotrebe može biti prikupljanje informacija o kolekciji stablolike strukture (recimo stablo parsiranja) i korišćenje istih za neko izračunavanje ili generisanje novih struktura (recimo AST). 

\section{Programske paradigme i gramatičke razlike programskih jezika}
\label{sec:Paradigms}

Iako se u suštini svode na mašinski jezik ili asembler, viši programski jezici mogu imati velike razlike međusobno --- kako u načinu pisanja koda, tako i u efikasnosti izvršavanja. Način, ili stil programiranja se naziva \emph{programska paradigma} \cite{ProgrammingParadigms}. Može se pokazati da sve što je rešivo putem jedne, može i da se reši i putem ostalih; međutim neki problemi se prirodnije rešavaju koristeći specifične paradigme. Neke poznatije programske paradigme su navedene u nastavku zajedno sa njihovim odlikama i primerima upotrebe.


\subsection{Imperativna paradigma}
\label{subsec:ParadigmImperative}

\emph{Imperativna paradigma} pretpostavlja da se promene u trenutnom stanju izvršavanja mogu sačuvati kroz promenljive. Izračunavanja se vrše putem niza koraka, u svakom koraku se te promenljive referišu ili se menjaju njihove trenutne vrednosti. Raspored koraka je bitan, jer svaki korak može imati različite posledice s obzirom na trenutne vrednosti promenljivih na početku tog koraka. Primer koda pisanog u imperativnoj paradigmi se može videti na slici \ref{fig:ParadigmImperative}.

\begin{figure}[h!]
\begin{lstlisting}
    result = []
    i = 0
start:
    numPeople = length(people)
    if i >= numPeople goto finished
    p = people[i]
    nameLength = length(p.name)
    if nameLength <= 5 goto nextOne
    upperName = toUpper(p.name)
    addToList(result, upperName)
nextOne:
    i = i + 1
    goto start
finished:
    return sort(result)
\end{lstlisting}
\caption{Primer koda pisanog u imperativnoj paradigmi.}
\label{fig:ParadigmImperative}
\end{figure}

Stariji programski jezici najčešće prate ovu paradigmu više nego bilo koju drugu iz par razloga. Prvi je taj što imperativna paradigma najbliže oslikava samu mašinu na kojoj se program izvršava, pa je programer mnogo "bliži" mašini. Ova paradigma je bila veoma popularna zbog ranih ograničenja u hardveru i potrebe za efikasnim programima. Danas, zbog mnogo bržeg razvoja i mnogo jačih računara, efikasnost se sve manje uzima u obzir.

Naravno, imperativna paradigma ima i svoje nedostatke. Naime, najveći problem je razumevanje i verifikovanje semantike programa zbog postojanja sporednih efekata\footnote{Sporedni efekti (promena stanja mašine) ne poštuju \emph{referencijalnu transparentnost} koja se definiše na sledeći način: \emph{Ako važi $P(x)$ i $x = y$ u nekom trenutku, onda $P(x) = P(y)$ važi tokom čitavog vremena izvršavanja programa}.}. Stoga je i pronalaženje grešaka u programima pisanim u imperativnoj paradigmi znatno komplikovanije. Pošto je k\^od veoma niskog nivoa, apstrakcija takvog koda je više ograničena nego u ostalim paradigmama. Na kraju, redosled izvršavanja je vrlo bitan, što neke probleme čini težim ukoliko se pokušaju rešiti pomoću imperativne paradigme.


\subsection{Strukturna paradigma}
\label{subsec:ParadigmImperativeStructural}

\emph{Strukturna paradigma} je vrsta imperativne paradigme gde se kontrola toka vrši putem ugnježdenih petlji, uslovnih grananja i podrutina. Promenljive su obično lokalne za blok u kome su definisane, što određuje i njihov životni vek i vidljivost. Primer koda pisanog u strukturnoj paradigmi se može videti na slici \ref{fig:ParadigmStructural}. Danas je najpopularnija kombinacija strukturne paradigme sa \emph{proceduralnom paradigmom}, baziranom na konceptu poziva \emph{procedure} --- podrutine ili funkcije koja sadrži seriju koraka koje je potrebno izvršiti redom.

\begin{figure}[h!]
\begin{lstlisting}
result = [];
for (i = 0; i < length(people); i++) {
    p = people[i];
    if (length(p.name)) > 5 {
        addToList(result, toUpper(p.name));
    }
}
return sort(result);
\end{lstlisting}
\caption{Primer koda pisanog po strukturnoj paradigmi.}
\label{fig:ParadigmStructural}
\end{figure}


\subsection{Skript paradigma i njen odnos sa proceduralnom paradigmom}
\label{subsec:Languages}

Čak i unutar jedne paradigme kao što je proceduralna, mogu se naći veoma velike varijacije u izgledu koda pisanog u različitim programskim jezicima koji prate proceduralnu paradigmu. Kako hardver postaje moćniji, više se ceni vreme koje programer provede u procesu pisanja koda nego koliko je taj kod efikasan. Štaviše, u nekim slučajevima je dobitak u efikasnosti veoma mali u poređenju sa vremenom koje je potrebno utrošiti da bi se ta efikasnost postigla. Ukoliko se program pokreće veoma retko, možda nije ni bitno da li se on izvršava sekundu sporije od efikasnog programa, ako je za njegovo pisanje utrošeno znatno manje vremena. Ovo je pristup koji prate \emph{skript} jezici kao što su \texttt{Python, Perl, bash} itd. Iako proceduralni, oni se razlikuju od klasičnih predstavnika proceduralne paradigme i njihove razlike su vremenom postale tolike da nije neuobičajeno da se skript jezici svrstaju u zasebnu, \emph{skript paradigmu}. Stoga će se u nastavku, pod terminom \emph{proceduralni jezik} smatrati tradicionalni proceduralni jezik, ukoliko nije naznačeno drugačije. Na slici \ref{fig:LanguagesDiff} se mogu uočiti navedene razlike.

\begin{figure}[h!]
\begin{lstlisting}
int main() {
    int k = 0;
    for (int i = 0; i < 1000000; i++)
        k++;
    return 0;
}
\end{lstlisting}
\begin{lstlisting}[language={}]
$ time: 0.03s user 0.00s system 70% cpu 0.044 total
\end{lstlisting}
\begin{lstlisting}
k = 0
for i in range(1000000):
    k += 1
\end{lstlisting}
\begin{lstlisting}[language={}]
$ time: 0.16s user 0.03s system 93% cpu 0.200 total
\end{lstlisting}
\caption{Primer koda pisanog po tradicionalnoj proceduralnoj paradigmi (gore, \texttt{C}) i po modernoj skript paradigmi (gore, \texttt{Python 3}) kao i odgovarajuća vremena izvršavanja dobijena komandom \texttt{time}.}
\label{fig:LanguagesDiff}
\end{figure}

Promenljive predstavljaju jedan od osnovnih koncepata na kojem se zasnivaju i proceduralni i skript jezici. Promenljivu odlikuje, između ostalog, i njen \emph{tip} koji određuje količinu memorije potrebnu za njeno skladištenje. Proceduralni programski jezici zahtevaju definisanje tipa promenljive i obično su i \emph{statički}, što znači da promenljive ne mogu menjati svoj tip tokom izvršavanja programa. Proces uvođenja imena promenljive se u naziva \emph{deklaracija promenljive}. Slično kao i za promenljive, potrebno je deklarisati i funkcije pre trenutka njihovog korišćenja kako bi prevodilac znao broj i tipove parametara funkcije kao i njihove povratne vrednosti. Skript jezici žrtvuju strogu tipiziranost kako bi proces pisanja koda bio brži. Stoga su oni obično \emph{dinamički} --- promenljive mogu menjati tip tokom izvršavanja programa. Pošto promenljive mogu menjati svoj tip, definisanje tipa prilikom uvođenja imena promenljive postaje redundantno jer prevodilac može to sam da zaključi. Stoga i sam proces uvođenja imena promenljive postaje redundantan. Slično, parametri funkcija takođe nisu fiksnog tipa. Slično važi i za povratnu vrednost funkcije.

Kod proceduralnih jezika, pošto su obično strogo tipizirani, mogu se iskoristiti strukture podataka koje omogućavaju brz pristup svojim elementiram. To su obično nizovi koji predstavljaju kontinualni blok memorije u kom su elementi niza smešteni jedan do drugog. Pristup se vrši na osnovu indeksa i, pošto su svi elementi istog tipa (zauzimaju jednaku količinu memorije), može se u konstantnom vremenu izračunati memorijska lokacija na kojoj se nalazi element niza sa datim indeksom. Kompleksnije strukture podataka obično nisu podržane u samom jeziku. Neki proceduralni jezici dozvoljavaju veoma niski pristup kroz \emph{pokazivače} ili \emph{reference} na memorijske adrese (\texttt{C} i \texttt{C++}). Većina modernih proceduralnih jezika ne dozvoljava rad sa pokazivačima, ne brinući puno o efikasnosti, dok neki dozvoljavaju korišćenje pokazivača u specijalnim situacijama sa eksplicitnom naznakom (\texttt{C\#}).

Pored dinamičnosti kad je u pitanju tip promenljivih, skript jezici često imaju neke specifične strukture podataka ugrađene u sam jezik kao olakšice prilikom programiranja. Primarna struktura podataka je \emph{jednostruko ulančana lista}\footnote{Lista je rekurzivna kolekcija podataka koja se sastoji od glave koja sadrži vrednost određenog tipa, i pokazivača na rep --- drugu listu. Specijalno, praznim pokazivačima se označava kraj liste (prazna lista).}, za razliku od niza kod proceduralnih jezika. Razlog zašto se koriste liste je delimično zbog toga što, kao i ostale promenljive, liste nisu strogo tipizirane. Moguće je u listu ubacivati elemente različitih tipova --- što onemogućava skladištenje u kontinualnom bloku memorije (osim ukoliko je lista imutabilna, što nije obično slučaj). Skript jezici uglavnom omogućavaju indeksni pristup elementima liste pa programeru izgleda kao da radi nad običnim nizom. Neki skript jezici omogućavaju kreiranje \emph{asocijativnih nizova}, gde indeks niza ne mora biti ceo broj već može uzimati vrednost iz domena bilo kog tipa. Osim listi, obično su podržane i torke, i za njih važe iste slobode kao i za liste. Kompleksnije strukture podataka uključuju skupove i rečnike (drugačije nazivane i \emph{heš mape}, engl. \emph{hash map}) koji su kolekcija ključ-vrednost parova gde je dozvoljen indeksni pristup po vrednosti ključa. Skript programski jezici su skoro uvek interpretirani, iako se neki jezici mogu kompajlirati po potrebi za efikasnije ponovno izvršavanje. S obzirom da efikasnost nije u glavnom planu, u skript jezicima nije dozvoljen direktan pristup memoriji putem pokazivača ili referenci. 


\subsection{Ostale popularne programske paradigme}
\label{subsec:ParadigmsOther}

\emph{Objektno-orijentisana paradigma} (kraće \emph{OOP}) je paradigma u kojoj se objekti stvarnog sveta posmatraju kao zasebni entiteti koji imaju sopstveno stanje koje se modifikuje samo pomoću procedura ugrađenih u same objekte --- tzv. \emph{metode}. Posledica zasebnog operisanja objekata omogućava njihovu enkapsulaciju u module koji sadrže lokalnu sredinu i metode. Komunikacija sa objektom se vrši prosleđivanjem poruka. Objekti su organizovani u klase, od kojih nasleđuju atribute i metode. OOP omogućava ponovnu iskorišćenost koda i ekstenzibilnost koda.

\emph{Logička paradigma} koristi deklarativni pristup rešavanju problema. Umesto zadavanja instrukcija koje treba da dovedu do rezultata, opisuje se sam rezultat kroz činjenice --- skup logičkih pretpostavki koji se zatim prevodi u upit koji se dalje koristi. Uloga računara je održavanje i logička dedukcija.

\emph{Funkcionalna paradigma} posmatra sve potprograme kao funkcije u matematičkom smislu --- uzimaju argumente i vraćaju jedinstven rezultat. Povratna vrednost zavisi isključivo od argumenata, što znači da je nebitan trenutak u kom je funkcija pozvana. Izračunavanja se vrše primenom i kompozicijom funkcija. 

