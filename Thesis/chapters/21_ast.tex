\section{Apstraktna sintaksna stabla - AST}
\label{sec:AST}

Kako bi se od datoteke na fajl sistemu koja sadrži izvorni kod programa 
došlo do izvršivog programa, potrebno je izvršiti više koraka 
\cite{CompilerConstruction}:
\begin{itemize}
    \item pretprocesiranje
    \item prevođenje
    \item asembliranje
    \item linkovanje
\end{itemize}

Ovi koraci će biti opisani na jednom primeru. Pretpostavimo da želimo 
da kompajliramo kod pisan u programskom jeziku C prikazan na slici 
\ref{fig:CompilationProcessInit}. Primetimo, da postoji greška u datom
kodu - simbol \texttt{c} koji se koristi će biti prepoznat kao 
identifikator koji ne odgovara nijednoj promenljivoj. Ovo, doduše, nije
sintaksna greška - izraz \texttt{a + c} je sasvim validan u programskom
jeziku C bez analize konteksta u kom se javlja. Problem će postati 
očigledan tek nakon parsiranja izvornog koda i provere ispunjenosti 
sintaksih pravila. Ovakve greške se nazivaju \emph{semantičke greške}.

\begin{figure}[h!]
    \begin{lstlisting}
    #include<stdio.h>

    #define T int

    int main()
    {
        T a, b;
        a = a + c;        // c nije deklarisano
        printf("%d", a);
        return 0;
    }
    \end{lstlisting}
    \caption{Primer izvornog koda pisanog u programskom jeziku C.}
    \label{fig:CompilationProcessInit}
\end{figure}

U fazi pretprocesiranja se vrše samo tekstualne operacije kao što su
brisanje komentara ili zamena makroa u jezicima kao što je C. Prvo 
mesto gde se vrši analiza sadržaja izvornog fajla je faza prevođenja.
Tu analizu vrši program koji se naziva \emph{pretprocesor}. Rezultat 
rada pretprocesora za kod sa slike \ref{fig:CompilationProcessInit} 
bi izgledao kao na slici \ref{fig:CompilationProcessPrep} \footnote{
U nekim implementacijama C standardne biblioteke, moguće je da se 
poziv funckije \texttt{printf} zameni pozivom funkcije \texttt{fprintf}
sa ispisom na \texttt{stdout}. U standardu se propisuje da funkcije 
kao što je \texttt{printf} mogu biti implementirane kao makroi. Izlaz 
na slici \ref{fig:CompilationProcessPrep} je generisan od strane 
\texttt{GCC 7.4.0} po C11 standardu.} i ovo nije slučaj u datom 
okruženju.

\begin{figure}[h!]
    \begin{lstlisting}
    int main()
    {
        int a, b;
        a = a + c;
        printf("%d", a);
        return 0;
    }
    \end{lstlisting}
    \caption{Rezultat rada pretprocesora za kod sa slike 
             \ref{fig:CompilationProcessInit}.}
    \label{fig:CompilationProcessPrep}
\end{figure}

Prilikom faze prevođenja, kako prevodilac ne bi radio nad sirovim 
karakterima izvornog koda, potrebno je izvršiti pripremu istog. 
Prevodilac ima u vidu moguće elemente programskog jezika, tzv. 
\emph{tokene}, koje treba prepoznati u datom fajlu - ključne reči, 
operatore, promenljive itd. Program koji radi \emph{tokenizaciju} -
prepoznavanje tokena u izvornom fajlu - se naziva \emph{lekser}. 
Pojednostavljen primer tokena koje lekser pokušava da prepozna 
se može videti na slici \ref{fig:CLexerExample}. Primer izlaza
leksera za izlaz pretprocesora sa slike \ref{fig:CompilationProcessPrep}
se može videti na slici \ref{fig:CompilationProcessLex}.

\begin{figure}[h!]
    \begin{lstlisting}
    Identifier : IdentifierNondigit 
                 (IdentifierNondigit | Digit)*
               ;

    IdentifierNondigit : Nondigit
                       | UniversalCharacterName
                       ;

    Nondigit : [a-zA-Z_]
             ;

    Digit : [0-9]
          ;
    \end{lstlisting}
    \caption{Primer delimične definicije tokena za ime promenljive po C11 standardu.}
    \label{fig:CLexerExample}
\end{figure}

\begin{figure}[h!]
    \begin{lstlisting}
    identifier 'main'	 [LeadingSpace]	Loc=<sample.c:3:5>
    l_paren '('		Loc=<sample.c:3:9>
    r_paren ')'		Loc=<sample.c:3:10>
    l_brace '{'	 [StartOfLine]	Loc=<sample.c:4:1>
    int 'int'	 [StartOfLine] [LeadingSpace]	Loc=<sample.c:5:5>
    identifier 'a'	 [LeadingSpace]	Loc=<sample.c:5:9>
    comma ','		Loc=<sample.c:5:10>
    identifier 'b'	 [LeadingSpace]	Loc=<sample.c:5:12>
    semi ';'		Loc=<sample.c:5:13>
    identifier 'a'	 [StartOfLine] [LeadingSpace]	Loc=<sample.c:6:5>
    equal '='	 [LeadingSpace]	Loc=<sample.c:6:7>
    identifier 'a'	 [LeadingSpace]	Loc=<sample.c:6:9>
    plus '+'	 [LeadingSpace]	Loc=<sample.c:6:11>
    identifier 'c'	 [LeadingSpace]	Loc=<sample.c:6:13>
    semi ';'		Loc=<sample.c:6:14>
    identifier 'printf'	 [StartOfLine] [LeadingSpace]	Loc=<sample.c:7:5>
    l_paren '('		Loc=<sample.c:7:11>
    string_literal '"%d"'		Loc=<sample.c:7:12>
    comma ','		Loc=<sample.c:7:16>
    identifier 'a'	 [LeadingSpace]	Loc=<sample.c:7:18>
    r_paren ')'		Loc=<sample.c:7:19>
    semi ';'		Loc=<sample.c:7:20>
    return 'return'	 [StartOfLine] [LeadingSpace]	Loc=<sample.c:8:5>
    numeric_constant '0'	 [LeadingSpace]	Loc=<sample.c:8:12>
    semi ';'		Loc=<sample.c:8:13>
    r_brace '}'	 [StartOfLine]	Loc=<sample.c:9:1>
    eof ''		Loc=<sample.c:9:2>
    \end{lstlisting}
    \caption{Primer delimične definicije tokena za ime promenljive po standardu C11.}
    \label{fig:CompilationProcessLex}
\end{figure}

Nakon završetka rada leksera potrebno je parsirati dobijene tokene.
Parsiranje vrši program koji se naziva \emph{parser}. Parser, slično
kao što lekser ima definicije tokena jezika, mora imati informacije 
o gramatici jezika. Gramatika programskog jezika se najčešće definiše
putem kontekstno-slobodnih gramatika \cite{ContextFreeGrammars}, 
čiji je primer dat na slici \ref{fig:CompilationProcessGram}.

\begin{figure}[h!]
    \begin{lstlisting}
    functionDefinition
        :   declarationSpecifiers? declarator declarationList? compoundStatement
        ;

    declarationList
        :   declaration
        |   declarationList declaration
        ;

    declaration
        :   declarationSpecifiers initDeclaratorList ';'
        | 	declarationSpecifiers ';'
        |   staticAssertDeclaration
        ;
    \end{lstlisting}
    \caption{Isečak gramatike programskog jezika C po standardu C11.}
    \label{fig:CompilationProcessLex}
\end{figure}

Izlaz rada parsera je \emph{stablo parsiranja} (eng. \emph{parse tree} 
ili \emph{derivation tree}). Takvo stablo i dalje sadrži sve relevantne
informacije o izvornom kodu. Vizuelni prikaz rada parsera za gramatiku
sa slike C11 i izvonog koda sa slike \ref{fig:CompilationProcessPrep} je
dat na slici \ref{fig:CompilationProcessPars}.

\begin{figure}[h!]
    \includegraphics{images/}
    \caption{Isečak gramatike programskog jezika C po standardu C11.}
    \label{fig:CompilationProcessLex}
\end{figure}