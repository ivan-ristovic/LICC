\section{Apstraktna sintaksna stabla - AST}
\label{sec:AST}

Kako bi se kod pisan u nekom programskom jeziku (\emph{izvorni fajl}) preveo u kod koji će se izvršavati na nekoj mašini (\emph{izvršivi fajl}), prevodilac prolazi kroz određene korake. Da bi se izvorni kod preveo, prevodilac mora da zna njegovu formu, ili \emph{sintaksu}, i njegovo značenje, ili \emph{semantiku}. Deo prevodioca koji određuje da li je izvorni kod ispravno formiran u terminima sintakse i semantike se naziva \emph{prednji deo} (engl. \emph{front end}). Ukoliko je izvorni kod ispravan, prednji deo kreira \emph{međureprezentaciju} koda (engl. \emph{intermediate representation}, u daljem tekstu \emph{IR}). Ukoliko to nije slučaj, prevođenje ne uspeva i programeru se daje poruka o detaljima zašto prevođenje nije uspelo. \cite{EngineeringCompilers}

Postupak rada prednjeg dela će biti opisan kroz konkretan primer. Pretpostavimo da želimo da prevedemo kod pisan u programskom jeziku C prikazan na slici \ref{fig:CompilationProcessInit}. Primetimo da postoji greška u datom kodu - simbol \texttt{c} koji se koristi u dodeli u liniji $8$ će biti prepoznat kao identifikator koji ne odgovara nijednoj deklarisanoj promenljivoj - stoga ne možemo prevesti ovaj kod. Ovo, doduše, nije sintaksna greška - izraz \texttt{a+c} je sasvim validan u programskom jeziku C bez analize konteksta u kom se javlja. Problem će postati očigledan tek nakon parsiranja izvornog koda i provere ispunjenosti sintaksih pravila, tačnije u fazi semantičke provere. Stoga se ovakve greške nazivaju \emph{semantičke greške}, dok se greške u sintaksi nazivaju \emph{sintaksne greške}.

\begin{figure}[h!]
    \begin{lstlisting}
    #include<stdio.h>

    #define T int

    int main()
    {
        T a, b;
        a = a + c;        // c nije deklarisano
        printf("%d", a);
        return 0;
    }
    \end{lstlisting}
    \caption{Primer izvornog koda pisanog u programskom jeziku C.}
    \label{fig:CompilationProcessInit}
\end{figure}

Pre nego što prednji kraj prevodioca uopšte dobije kod koji treba prevesti, vrši se \emph{pretprocesiranje} od strane programa koji se naziva \emph{pretprocesor}. U fazi pretprocesiranja se izvode samo tekstualne operacije kao što su brisanje komentara ili zamena makroa u jezicima kao što je C. Rezultat rada pretprocesora za kod sa slike \ref{fig:CompilationProcessInit} bi izgledao kao na slici \ref{fig:CompilationProcessPrep} 
\footnote{U nekim implementacijama C standardne biblioteke, moguće je da se poziv funckije \texttt{printf} zameni pozivom funkcije \texttt{fprintf}
sa ispisom na \texttt{stdout}. U standardu se propisuje da funkcije kao što je \texttt{printf} mogu biti implementirane kao makroi. Izlaz 
na slici \ref{fig:CompilationProcessPrep} je generisan od strane \texttt{GCC 7.4.0} po C11 standardu i ovo nije slučaj u datom okruženju.}.

\begin{figure}[h!]
    \begin{lstlisting}
    # 1 "<stdin>"
    # 1 "<built-in>"
    # 1 "<command-line>"
    # 31 "<command-line>"
    # 1 "/usr/include/stdc-predef.h" 1 3 4

    ...

    extern char *ctermid (char *__s) __attribute__ ((__nothrow__ , __leaf__));
    # 840 "/usr/include/stdio.h" 3 4
    extern void flockfile (FILE *__stream) __attribute__ ((__nothrow__ , __leaf__));
    extern int ftrylockfile (FILE *__stream) __attribute__ ((__nothrow__ , __leaf__)) ;
    extern void funlockfile (FILE *__stream) __attribute__ ((__nothrow__ , __leaf__));
    # 868 "/usr/include/stdio.h" 3 4
    # 2 "<stdin>" 2
    # 2 "<stdin>"

    int main()
    {
        int a, b;
        a = a + c;
        printf("%d", a);
        return 0;
    }
    \end{lstlisting}
    \caption{Prikaz rezultata rada pretprocesora za izvorni kod sa slike \ref{fig:CompilationProcessInit}. Pritom, prikazano je samo par linija sa početka i kraja izlaza pretprocesora - kod iznad \texttt{main} funkcije je uključen iz \texttt{stdio.h} zaglavlja.}
    \label{fig:CompilationProcessPrep}
\end{figure}

Da bi proverio sintaksu izvornog koda, prevodilac mora da uporedi strukturu istog sa unapred definisanom strukturom za određeni programski jezik. Ovo zahteva formalnu definiciju sintakse jezika. Programski jezik možemo posmatrati kao skup \emph{pravila} koji se naziva \emph{gramatika} \cite{ContextFreeGrammars}, prikazana na slici \ref{fig:CompilationProcessGram}. U prednjem delu se izvode dva procesa koji određuju da li ulaz zaista zadovoljava gramatiku određenog programskog jezika. Ova dva procesa se nazivaju \emph{skeniranje} i \emph{parsiranje}, a komponente prednjeg dela koje vrše te procese se nazivaju \emph{skener} (takođe se naziva i \emph{lekser}) i \emph{parser}, redom.

\begin{figure}[h!]
    \begin{lstlisting}[language={}]
    functionDefinition
        :   declarationSpecifiers? declarator 
            declarationList? compoundStatement
        ;

    declarationList
        :   declaration
        |   declarationList declaration
        ;

    declaration
        :   declarationSpecifiers initDeclaratorList ';'
        | 	declarationSpecifiers ';'
        |   staticAssertDeclaration
        ;
    \end{lstlisting}
    \caption{Isečak gramatike programskog jezika C po standardu C11.}
    \label{fig:CompilationProcessGram}
\end{figure}

Prilikom faze prevođenja, kako prevodilac ne bi radio nad sirovim karakterima izvornog koda, potrebno je izvršiti pripremu istog - skeniranje. Prevodilac ima u vidu moguće elemente programskog jezika, tzv. \emph{tokene}, koje treba prepoznati u datom fajlu - ključne reči, operatore, promenljive itd. Proces prepoznavanja tokena u izvornom fajlu se naziva \emph{tokenizacija}. Pojednostavljen primer tokena koje lekser pokušava da prepozna se može videti na slici \ref{fig:CLexerExample}. Primer izlaza leksera za izlaz pretprocesora sa slike \ref{fig:CompilationProcessPrep} se može videti na slici \ref{fig:CompilationProcessLex}.
\footnote{Moderni kompajleri često nemaju odvojene faze u kojima se pozivaju skeniranja i parsiranja, već se skeniranje odvija paralelno sa fazom parsiranja. Međutim, to nas ne sprečava da ispišemo tokene onda kada se oni prepoznaju, i to je demonstrirano na slici \ref{fig:CompilationProcessLex}.}

\begin{figure}[h!]
    \begin{lstlisting}[language={}]
    Identifier : IdentifierNondigit 
                 (IdentifierNondigit | Digit)*
               ;

    IdentifierNondigit : Nondigit
                       | UniversalCharacterName
                       ;

    Nondigit : [a-zA-Z_]
             ;

    Digit : [0-9]
          ;
    \end{lstlisting}
    \caption{Primer delimične definicije tokena za ime promenljive po C11 standardu.}
    \label{fig:CLexerExample}
\end{figure}

\begin{figure}[h!]
    \begin{lstlisting}[language={}]
    identifier 'main'	 [LeadingSpace]	Loc=<sample.c:3:5>
    l_paren '('		Loc=<sample.c:3:9>
    r_paren ')'		Loc=<sample.c:3:10>
    l_brace '{'	 [StartOfLine]	Loc=<sample.c:4:1>
    int 'int'	 [StartOfLine] [LeadingSpace]	Loc=<sample.c:5:5>
    identifier 'a'	 [LeadingSpace]	Loc=<sample.c:5:9>
    comma ','		Loc=<sample.c:5:10>
    identifier 'b'	 [LeadingSpace]	Loc=<sample.c:5:12>
    semi ';'		Loc=<sample.c:5:13>
    identifier 'a'	 [StartOfLine] [LeadingSpace]	Loc=<sample.c:6:5>
    equal '='	 [LeadingSpace]	Loc=<sample.c:6:7>
    identifier 'a'	 [LeadingSpace]	Loc=<sample.c:6:9>
    plus '+'	 [LeadingSpace]	Loc=<sample.c:6:11>
    identifier 'c'	 [LeadingSpace]	Loc=<sample.c:6:13>
    semi ';'		Loc=<sample.c:6:14>
    identifier 'printf'	 [StartOfLine] [LeadingSpace]	Loc=<sample.c:7:5>
    l_paren '('		Loc=<sample.c:7:11>
    string_literal '"%d"'		Loc=<sample.c:7:12>
    comma ','		Loc=<sample.c:7:16>
    identifier 'a'	 [LeadingSpace]	Loc=<sample.c:7:18>
    r_paren ')'		Loc=<sample.c:7:19>
    semi ';'		Loc=<sample.c:7:20>
    return 'return'	 [StartOfLine] [LeadingSpace]	Loc=<sample.c:8:5>
    numeric_constant '0'	 [LeadingSpace]	Loc=<sample.c:8:12>
    semi ';'		Loc=<sample.c:8:13>
    r_brace '}'	 [StartOfLine]	Loc=<sample.c:9:1>
    eof ''		Loc=<sample.c:9:2>
    \end{lstlisting}
    \caption{Proces tokenizacije koda sa slike \ref{fig:CompilationProcessPrep}. Generisano uz pomoć \texttt{clang} \cite{Clang} kompajlera.}
    \label{fig:CompilationProcessLex}
\end{figure}

Nakon faze skeniranja potrebno je parsirati dobijene tokene. Parser, imajući u vidu gramatiku jezika, pokušava da kreira \emph{stablo parsiranja} (eng. \emph{parse tree} ili \emph{derivation tree}). Takvo stablo i dalje sadrži sve relevantne informacije o izvornom kodu. Vizuelni prikaz rada parsera za gramatiku sa slike C11 i izvonog koda sa slike \ref{fig:CompilationProcessPrep} je dat na slici \ref{fig:CompilationProcessPars}. Stablo parsiranja se koristi u narednim fazama prevođenja.

\begin{figure}[h!]
    \centering
    \scalebox{0.95}[1.3] {
        \includegraphics[width=\textwidth]{images/parse_tree.png}
    }
    \caption{Prikaz dela stabla parsiranja koje generiše parser kreiran od strane alata ANTLR4 za kod sa slike \ref{fig:CompilationProcessPrep}.}
    \label{fig:CompilationProcessPars}
\end{figure}

Za potrebe ovog rada, što se procesa prevođenja tiče, dovoljno je pozvnavanje prednjeg dela, stoga neće biti reči o daljim koracima u fazi prevođenja (semantička provera, optimizacije, generisanje IR). Zainteresovani čitalac može više detalja pronaći u \cite{EngineeringCompilers} i \cite{CompilerConstruction}. 

Stablo parsiranja sadrži sve informacije potrebne u fazi parsiranja uključujući detalje korisne samo za parser prilikom provere ispunjenosti gramatičkih pravila. Sa druge strane, \emph{apstraktno sintaksno stablo} sadrži samo sintaksnu strukturu u jednostavnijoj formi. Na slici \ref{fig:CompilationProcessPars1} se može videti koliko je stablo parsiranja komplikovano čak i za naizgled jednostavne aritmetičke izraze. Razlog ovolike komplikovanosti dolazi iz rekurzivnih pravila definisanih u C11 gramatici. Parseru su sve ove informacije neophodne ali na apstratknijem nivou nisu potrebne. Jedina važna semantička odlika izraza \texttt{a+c} je ta da je to zbir vrednosti nekih promenljivih. Sve ostale informacije su nepotrebne. Na slici \ref{fig:ASTSimple} možemo videti različita apstraktna sintaksna stabla za pomenuti izraz, ali takođe i za malo komplikovanije izraze. Podrazumeva se, naravno, da je ulaz već tokenizovan. 

\begin{figure}[h!]
    \centering
    \includegraphics[scale=0.6]{images/parse_tree_expr.png}
    \caption{Prikaz kompleksnosti stabla parsiranja za izraz 
    \texttt{a+c} u C11 gramatici.} 
    \label{fig:CompilationProcessPars1}
\end{figure}

\begin{figure}[h!]
    \centering
    \includegraphics[scale=0.8]{images/ast.png}
    \caption{Varijante apstraktnih sintaksnih stabala bez regularnosti(levo) i sa regularnošću (desno) za izraze \texttt{a+c} (gore) i \texttt{a + (3 - c)} (dole).} 
    \label{fig:ASTSimple}
\end{figure}

Uloga apstraktnih sintaksnih stabala je da pokažu semantiku strukture koda preko stabala. Kao što se vidi na slici \ref{fig:ASTSimple}, postoji određeni nivo slobode u dizajniranju ovih stabala. Generalno, \emph{terminalni simboli}, simboli koji predstavljaju listove stabla parsera, koji odgovaraju operatorima i naredbama se podižu naviše i postaju koreni podstabala, dok se njihovi operandi ostavljaju kao njihovi potomci u stablu. Desna stabla sa slike ne prate u potpunosti ovaj princip, ali se takođe koriste zbog regularnosti izraza - recimo ukoliko binarni izraz posmatramo kao koncept, mnogo je lakše raditi sa ovakvom strukturom. Ovakva struktura će biti korišćena kasnije u implementaciji programa. Primetimo takođe da se u stablima za izraz \texttt{a + (3 - c)} (dole) implicitno sačuvala informacija o prioritetu operacije oduzimanja u izrazu. Jasno je, dakle, da se računanje vrednosti aritmetičkih izraza onda vrši kretanjem od listova stabla ka korenu. Takođe, pošto su apstraktna sintaksna stabla apstrakcija stabla parsiranja, više istih izraza jezika može imati isto apstraktno sintaksno stablo ali različito stablo parsiranja; na primer, ako razmatramo izraz \texttt{(a + 5) - x / 2} i izraz \texttt{a + 5 - (x / 2)}.

Apstrakna sintaksna stabla će u daljem tekstu biti referisana skraćenicom \emph{AST}, koja dolazi od engleskog naziva \emph{Abstract Syntax Trees}. Takođe, reči samom dizajnu i tipovima čvorova AST-ova korišćenih u implementaciji će biti u poglavlju \ref{TODO}. U narednom poglavlju će više biti reči o alatu koji je korišćen za kreiranje parsera za C11 gramatiku (ali i za druge, proizvoljne gramatike), koji daje mogućnost jednostavnog obilaska stabla parsiranja i pruža intuitivan način za izvršavanje logike nad tim stablom, što uključuje i izradu AST-a od stabla parsiranja.
