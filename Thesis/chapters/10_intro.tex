\chapter{Uvod}
\label{chp:Intro}

Apstraktna sintaksna stabla (engl. \emph{abstract syntax tree}, skr. \emph{AST}) imaju veliku ulogu u procesu analize programa. Na osnovu toga da li se analiza vrši pre ili u toku izvršavanja programa, te analize mogu biti \emph{statičke} i \emph{dinamičke}, pri čemu se AST najviše koristi za statičku analizu. Statičke analiza omogućava otkivanje raznih grešaka veoma rano, u nekim slučajevima čak i tokom pisanja problematičnog dela programa. 

AST nastaje parsiranjem izvornog koda, kao rezultat apstrahovanja stabla parsiranja koje generiše parser - program koji parsira izvorni kod i identifikuje sintaksne konstrukte jezika. Parser pokušava da u izvornom kodu pronađe određena gramatička pravila jezika koji je dizajniran da prepozna. Svaki programski jezik ima specifična sintaksna pravila pa stoga su i gramatike programskih jezika raznorodne, što se stoga prenosi i na stabla parsiranja. Za statičku analizu često nije neophodno posmatrati kod na nivou parsera (kroz stablo parsiranja), pošto informacije koje su neophodne parseru često nisu potrebne za neku konkretnu analizu. Stoga se stablo parsiranja često apstrahuje tako da što iz njega izvuku bitne sintaksne informacije - što za rezultat daje AST.

Posmatranje programa na apstraktnom nivou kroz AST pruža mogućnost za poređenje dva programa na apstraktnom nivou. Jedna od primena može biti semantička ekvivalentnost, u okviru statičke analize. Semantička ekvivalentnost je neodlučiv problem u opštem slučaju, međutim pod određenim pretpostavkama je moguće kreirati algoritme koji daju smislene rezultate. Te pretpostavke mogu uključiti i sličnost u strukturi programa, što se takođe može odraziti i na izgled i sličnosti u strukturi njihovih AST-ova, problem koji se može rešiti primenom algoritama za rad sa stablima (ali i grafovima uopšte, jer su stablo specijalizacija grafa). 

Iako veoma konceptualno moćan alat, AST je previše specifičan za konkretni programski jezik, s obzirom da nastaje od stabla parsiranja koje je usko vezano za gramatiku konkretnog programskog jezika. Motivacija za ovaj rad počiva u nepostojanju opštih apstrakcija za više programskih jezika. Danas postoji dosta programskih jezika ali u okviru iste programske paradigme ne postoji puno mesta za inovacije, s obzirom da su koncepti koje programski jezici te paradigme pružaju univerzalni. U ovom radu će biti predstavljena opšta AST apstrakcija za proceduralne programske jezike, sa ciljem da je moguće što više proceduralnih jezika dovesti na istu apstraktnu reprezentaciju, čak i programske jezike koji pripadaju skript paradigmi. Takođe, na apstraktnom nivou nije važno od kog se programskog jezika dobio AST, što ima veliku važnost u procesu migracije na nove tehnologije, s obzirom da je prepisivanje programa sa jednog jezika na drugi danas veoma česta operacija.

Naravno, semantička ekvivalentnost se ne mora zasnivati na apstrahovanju programa, već se takođe često zasniva na spuštanju na nivo međukoda između visokog programskog jezika i asemblera, u nekim slučajevima i do mašinskog jezika. U ovom radu je odabran apstraktni pristup s obzirom na pomenutu važnost AST-ova i nedostatkom opštih apstrakcija. 

U poglavlju \ref{chp:RelevantTerms} će biti opisani relevantni pojmovi potrebni za razumevanje rada uz akcenat na apstraktnim sintaksnim stablima i procesu dobijanja istih. Opšta AST apstrakcija za imperativne jezike biće opisana u poglavlju \ref{chp:MyAST}, a njena upotreba u problemu odlučivanja semantičke ekvivalentnosti kao i sam algoritam za poređenje opštih apstrakcija biće opisani u poglavlju \ref{chp:ASTComparing}. Implementacija apstrakcije i algoritama semantičkog poređenja će biti opisana u poglavlju \ref{chp:Implementation}. Na kraju, biće dati glavni zaključci ovog rada kao i moguća unapređenja i budući koraci. 

