\chapter{Uvod}
\label{chp:Intro}

Apstraktno sintaksičko stablo (engl. \emph{abstract syntax tree}, skr. \emph{AST}) programa ima značajnu ulogu u procesu kreiranja izvršivog programa od izvornog koda. AST nastaje parsiranjem izvornog koda, kao rezultat apstrahovanja stabla parsiranja koje generiše \emph{parser}. Parser čita izvorni k\^od i pokušava da u njemu pronađe primene određenih pravila jezika čiji je k\^od dizajniran da parsira. Svaki programski jezik ima specifična sintaksna pravila pa su stoga i skupovi pravila (tzv. \emph{gramatike}) programskih jezika raznorodni, što se zatim prenosi i na generisana stabla parsiranja. Stablo parsiranja se apstrahuje tako što se iz njega izvuku samo bitne sintaksne a uklone neke tehničke informacije.

Ovakva apstrakcija se najpre koristi u semantičkoj analizi programa koju vrši prevodilac nakon faze parsiranja i provere sintaksne ispravnosti koda. Ukoliko program prođe semantičke provere, prelazi se na prevođenje istog u međureprezentaciju i fazu optimizacije. Nakon faze optimizacije sledi generisanje asemblerskog koda koji se zatim prevodi u mašinski k\^od.

AST, zbog svoje uloge u semantičkoj analizi, može poslužiti i za analizu programa pre samog prevođenja, kroz proces poznat pod nazivom \emph{statička analiza}. Posmatranje programa kroz AST pruža mogućnost za poređenje dva programa na apstraktnom nivou. Jedna primena ove ideje u okviru statičke analize može biti provera semantičke ekvivalentnosti. Provera semantičke ekvivalentnosti dva programa je neodlučiv problem u opštem slučaju, međutim pod određenim pretpostavkama koje pojednostavljuju problem moguće je dizajnirati algoritme koji daju smislene rezultate u praksi. Jedna od često korišćenih pretpostavki je pretpostavka sličnosti strukture dva programa. Interesantno, provera da li dva programa zadovoljavaju ovu premisu se može proveriti posmatranjem izgleda i sličnosti u strukturi na apstraktnom nivou --- problem koji se može rešiti primenom algoritama za rad sa stablima jer je u pitanju AST (ali i grafovima uopšte, jer je stablo specijalizacija grafa).

Iako veoma konceptualno moćan alat, AST je ipak specifičan za konkretni programski jezik s obzirom da nastaje od stabla parsiranja koje je usko vezano za gramatiku konkretnog programskog jezika. Motivacija za ovaj rad dolazi od nepostojanja opštih apstrakcija sintaksičkih stabala koje bi se mogle koristiti za analizu programa napisanih u različitim programskim jezicima. Iako je broj programskih jezika danas veoma veliki, u okviru iste programske paradigme jezici moraju implementirati koncepte koji su potrebni da bi se programiralo u toj paradigmi i ta zajednička svojstva se mogu iskoristiti za formiranje opšte apstrakcije na AST nivou. 

U ovom radu će biti predstavljena opšta AST apstrakcija za imperativne programske jezike, sa ciljem da se omogući zajednička apstraktna reprezentacija velikog broja imperativnih jezika, pa čak i onih koji pripadaju skript paradigmi. Njena upotreba će biti demonstrirana na problemu semantičke ekvivalentnosti dobijenih apstrakcija kroz naivni algoritam poređenja simboličkih promenljivih. Štaviše, na apstraktnom nivou nije važno od kog se programskog jezika dobio AST, što može imati primenu u procesu migracije na nove tehnologije.

Naravno, semantička ekvivalentnost se ne mora zasnivati na apstrahovanju programa, već se takođe često rešava spuštanjem na nivo međukoda između višeg programskog jezika i asemblera. U nekim slučajevima se može iči i do asemblera pa i mašinskog jezika. Ukoliko bi se posmatrali asemblerski ili mašinski kod, vršilo bi se poređenje kodova prilagođenih određenoj arhitekturi procesora. U ovom radu je odabran AST-zasnovan pristup, s obzirom na važnosti i značaj apstraktnih sintaksičkih stabala, ali i zbog nedostatka opštih apstrakcija.

U poglavlju \ref{chp:RelevantTerms} će biti opisani relevantni pojmovi potrebni za razumevanje rada uz akcenat na apstraktnim sintaksičkim stablima i procesu njihovog dobijanja. U poglavlju \ref{sec:Symbolics} će biti opisano simboličko izračuvanje koje se koristi prilikom provere sematničke ekvivalentnosti apstraktnih sintaksičkih stabala. Opšta AST apstrakcija za imperativne jezike biće opisana u poglavlju \ref{chp:MyAST}, a njena upotreba u problemu odlučivanja semantičke ekvivalentnosti kao i sam algoritam za poređenje opštih apstrakcija biće opisani u poglavlju \ref{chp:ASTComparing}. Implementacija apstrakcije i algoritma semantičkog poređenja će biti opisana u poglavlju \ref{chp:Implementation}. Na kraju, biće dati glavni zaključci ovog rada kao i moguća unapređenja i budući koraci. 

