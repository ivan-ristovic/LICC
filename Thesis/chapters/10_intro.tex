\chapter{Uvod}
\label{chp:Intro}

Apstraktna sintaksička stabla (engl. \emph{abstract syntax trees}, skr. \emph{AST}) imaju veliku ulogu u procesu analize izvršavanja programa. Na osnovu toga da li se analiza vrši pre ili u toku izvršavanja programa, te analize mogu biti \emph{statičke} i \emph{dinamičke}, pri čemu se AST najviše koristi za statičku analizu. Statička analiza omogućava otkivanje grešaka veoma rano, u nekim slučajevima čak i u toku pisanja programa. 

AST nastaje parsiranjem izvornog koda, kao rezultat apstrahovanja stabla parsiranja koje generiše \emph{parser}. Parser čita izvorni kod i pokušava da u njemu pronađe određena gramatička pravila jezika čije je kodove dizajniran da parsira. Svaki programski jezik ima specifična sintaksna pravila pa stoga su i skupovi pravila (tzv. \emph{gramatike}) programskih jezika raznorodne, što se stoga prenosi i na generisana stabla parsiranja. Za statičku analizu često nije neophodno posmatrati kod na nivou parsera kroz stablo parsiranja pošto informacije koje su neophodne parseru često nisu od preteranog značaja za neku konkretnu analizu. Stoga se stablo parsiranja često apstrahuje tako da što iz njega izvuku same bitne sintaksne informacije a uklone neke tehničke informacije - što za rezultat daje AST.

Posmatranje programa na apstraktnom nivou kroz AST pruža mogućnost za poređenje dva programa na apstraktnom nivou. Jedna primena ove ideje u okviru statičke analize može biti provera semantičke ekvivalentnosti. Provera semantičke ekvivalentnosti dva programa je neodlučiv problem u opštem slučaju, međutim pod određenim pretpostavkama koje pojednostavljuju problem moguće je dizajnirati algoritme koji daju smislene rezultate u praksi. Jedna od često korišćenih pretpostavki je pretpostavka sličnosti strukture dva programa. Interesantno, provera da li dva programa zadovoljavaju ovu premisu se može proveriti posmatranjem izgleda i sličnosti u strukturi njihovih AST-ova - problem koji se može rešiti primenom algoritama za rad sa stablima (ali i grafovima uopšte, jer su stablo specijalizacija grafa).

Iako veoma konceptualno moćan alat, AST je previše specifičan za konkretni programski jezik s obzirom da nastaje od stabla parsiranja koje je usko vezano za gramatiku konkretnog programskog jezika. Motivacija za ovaj rad dolazi od nepostojanja opštih apstrakcija za više programskih jezika koji pripadaju istoj programskoj paradigmi. Iako je broj programskih jezika danas veoma veliki, u okviru iste programske paradigme ne postoji puno prostora za inovacije s obzirom da su koncepti koje ta paradigma pruža univerzalni. U ovom radu će biti predstavljena opšta AST apstrakcija za imperativne programske jezike, sa ciljem da je moguće što više imperativnih jezika dovesti na istu apstraktnu reprezentaciju, čak i programske jezike koji pripadaju skript paradigmi. Njena upotreba će biti demonstrirana na problemu semantičke ekvivalentnosti dobijenih apstrakcija kroz naivni algoritam poređenja simboličkih promenljivih. Štaviše, na apstraktnom nivou nije važno od kog se programskog jezika dobio AST, što može imati veliku primenu u procesu migracije na nove tehnologije, s obzirom da je prepisivanje programa sa jednog jezika na drugi danas veoma česta operacija.

Naravno, semantička ekvivalentnost se ne mora zasnivati na apstrahovanju programa, već se takođe često rešava spuštanjem na nivo međukoda između višeg programskog jezika i asemblera, u nekim slučajevima se može iči i do asemblera pa i mašinskog jezika. Na ovom nivou se vrši poređenje kodova bliskih računaru, što obično zahteva i prilagođavanje određenoj arhitekturi procesora. U ovom radu je odabran apstraktni pristup s obzirom na pomenutu primenu AST-ova i nedostatkom opštih apstrakcija, ali takođe zbog nezavisnosti od specifične arhitekture procesora. 

U poglavlju \ref{chp:RelevantTerms} će biti opisani relevantni pojmovi potrebni za razumevanje rada uz akcenat na AST-ovima i procesu dobijanja istih. Opšta AST apstrakcija za imperativne jezike biće opisana u poglavlju \ref{chp:MyAST}, a njena upotreba u problemu odlučivanja semantičke ekvivalentnosti kao i sam algoritam za poređenje opštih apstrakcija biće opisani u poglavlju \ref{chp:ASTComparing}. Implementacija apstrakcije i algoritma semantičkog poređenja će biti opisana u poglavlju \ref{chp:Implementation}. Na kraju, biće dati glavni zaključci ovog rada kao i moguća unapređenja i budući koraci. 

