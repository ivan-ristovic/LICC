\chapter{Implementacija}
\label{chp:Implementation}

U ovom poglavlju će biti opisana implementacija pratećeg projekta pisanog u programskom jeziku C\# 8.0, koristeći \emph{.NET Core 3.1} radni okvir. C\# je izabran zbog lakoće implementacije velikih projekata i velike podrške paketa koji se mogu preuzeti, od kojih su korišćeni \emph{ANTLR Runtime} paket koji daje potrebne biblioteke za rad sa ANTLR generisanim parserima i \emph{Math.NET Symbolics} paket za rad sa simboličkim vrednostima. Rezultat je konzolna aplikacija koja može da generiše, serijalizuje ili prikaže opšti AST za dati izvorni kod, ali i da poredi takav AST sa drugim. Čitav projekat je dostupan u potpunosti na servisu GitHub na adresi \url{https://github.com/ivan-ristovic/LICC}.

Jedan od glavnih ciljeva aplikacije je modularnost i jednostavna proširivost. U tom duhu se, pored implementacije klasa potrebnih za predstavljanje opšte AST apstrakcije, pruža i interfejs za kreiranje adaptera koji će od proizvoljnog stabla parsiranja kreirati opšti AST. Kao primer, adapteri su kreirani za programske jezike C i Lua, a za primer potpune slobode u izboru gramatike je kreirana gramatika za pseudo-jezik i adapter za istu, što dozvoljava poređenje kodova sa specifikacijom datom u obliku pseudo-koda. Čitav projekat se sastoji od više komponenti, od kojih su značajnije:
\begin{itemize}
    \item Biblioteka koja pruža klase za rad sa opštom AST apstrakcijom
    \item Komponenta za kreiranje AST od proizvoljne gramatike putem adaptera --- moguća serijalizacija u JSON
    \item Komponenta za semantičko poređenje opštih AST --- konzolni izlaz
    \item Komponenta za AST vizualizaciju --- grafički prikaz AST
    \item Korisnički interfejs --- komandna linija
\end{itemize}

Čitava arhitektura data putem UML dijagrama komponenti se može videti na slici \ref{fig:ImplementationComponents}. Osim implementacije same aplikacije, svaki funkcionalni deo projekta prate i testovi jedinica koda, koji su povezani sa \emph{GitHub Actions} porškom za neprekidnu integraciju.

\begin{figure}[h!]
\centering
\includegraphics[scale=0.8]{images/uml/ComponentDiagram.png}
\caption{UML dijagram komponenti implementacije.}
\label{fig:ImplementationComponents}
\end{figure}


\section{Implementacija apstrakcije}
\label{sec:ImplementationMyAST}

Implementacija prati hijerarhije opisane u poglavlju \ref{chp:MyAST} kroz mehanizam nasleđivanja. Svaki tip čvora će biti zasebna klasa koja nasleđuje apstraktnu klasu \texttt{ASTNode}. Dijagram klasa koje nasleđuju klasu \texttt{ASTNode} se može videti na slikama \ref{fig:UMLASTNode1} i \ref{fig:UMLASTNode2}. Pored implementacije klasa koje predstavljaju AST čvorove, kreiran je i interfejs za obilazak AST-a putem obrazca posetilac.

\begin{figure}[h!]
\centering
\includegraphics[scale=0.7]{images/uml/ASTNode.png}
\line(1,0){450}\\
\includegraphics[scale=0.7]{images/uml/OperatorNode.png}
\caption{UML klasni dijagram (deo 1).}
\label{fig:UMLASTNode1}
\end{figure}

\begin{figure}[h!]
\centering
\includegraphics[scale=0.65]{images/uml/DeclarationNode.png}
\includegraphics[scale=0.55]{images/uml/ExpressionNode.png}
\includegraphics[scale=0.55]{images/uml/StatementNode.png}
\caption{UML klasni dijagram (deo 2).}
\label{fig:UMLASTNode2}
\end{figure}

Ovako kreirana AST struktura je \emph{imutabilna} - ne mogu se dinamički dodavati ili uklanjati deca čvorovima. Moguće je klonirati AST čvorove ili vršiti zamenu određenog podstabla drugim podstablom unutar AST-a, ne menjajući original već kopiju originala. Svaki AST čvor se može porediti po jednakosti sa drugim AST čvorom sa intuitivnom logikom poređenja pruženom kroz predefinisane operatore poređenja po jednakosti.


\section{Implementacija upoređivača}
\label{sec:ImplementationComparer}

\pangrami
\section{Implementacija vizualnog prikaza AST}
\label{sec:ImplementationVisualizer}


\section{Implementacija korisničkog interfejsa}
\label{sec:ImplementationUI}


