\section{ANTLR}
\label{sec:ANTLR}

Pretpostavljajući da imamo gramatiku proizvoljnog programskog jezika, postavlja se pitanje: \emph{Da li je moguće definisati postupak i zatim napraviti program koji će generisati kodove leksera i parsera napisane u određenom programskom jeziku za proizvoljnu gramatiku datu na ulazu?}. Odgovor je potvrdan i postoji veliki broj alata koji se mogu koristiti u ove svrhe, od kojih je navedeno par njih: 
\begin{itemize}
    \item \emph{GNU Bison} \cite{GNUBison}\\
        GNU Bison je generator parsera i deo GNU projekta \cite{GNUProject}, često referisan samo kao \emph{Bison}. Bison generiše parser na osnovu korisnički definisanih kontekstno slobodnih gramatika \cite{ContextFreeGrammars}, upozoravajući pritom na dvosmislenosti prilikom parsiranja ili nemogućnost primena gramatičkih pravila. Generisani parser je najčešće C a ređe C++ kod, mada se u vreme pisanja ovog rada eksperimentiše sa Java podrškom. Generisani kodovi su u potpunosti prenosivi i ne zahtevaju specifične kompajlere. Bison može da, osim podrazumevanih \emph{LALR(1)} \cite{LALR1} parsera, generiše i kanoničke \emph{LR} \cite{LR}, \emph{IELR(1)} \cite{IELR1} i \emph{GLR} \cite{GLR} parsere.
    \item \emph{Flex} \cite{Flex}\\
        Kreiran kao alternativa \emph{lex}-u \cite{LexYacc}, Flex generiše samo leksere pa se stoga najčešće koristi u kombinaciji sa drugim alatima koji mogu da generišu parsere, kao što je \emph{BYACC}, opisan u nastavku.
    \item \emph{BYACC} \cite{BYACC}\\
        \emph{Berkeley YACC}, skraćeno \emph{BYACC}, pisan po ANSI C standardu i otvorenog koda, se smatra od strane mnogih kao \textit{najbolja varijanta YACC-a} \cite{LexYacc}. BYACC dozvoljava tzv. \emph{reentrant} kod - memorija je deljenja između poziva pa je bezbedno konkurentno izvršavanje koda - na način kompatibilan sa Bison-om i to je delom razlog njegove popularnosti.
    \item \emph{ANTLR} \cite{ANTLR}\\
        \emph{Another Tool for Language Recognition}, ili kraće \emph{ANTLR}, je generator \emph{LL(*)} \cite{LLStar} leksera i parsera pisan u programskom jeziku Java sa intuitivnim interfejsom za obilazak stabla parsiranja. Verzija $3$ podržava generisanje parsera u jezicima Ada95, ActionScript, C, C\#, Java, JavaScript, Objective-C, Perl, Python, Ruby, i Standard ML, dok verzija $4$ u vreme pisanja ovog rada samo generiše parsere u narednim jezicima: Java, C\#, C++, JavaScript, Python, Swift i Go.
\end{itemize}
ANTLR, verzije $4$, je izabran u ovom radu zbog svoje jednostavnosti, intuitivnosti i podrške za mnoge moderne programske jezike. Verzija $4$ je izabrana umesto verzije $3$ po preporuci autora, na osnovu eksperimentalne analize brzine i pouzdanosti verzije $4$ u odnosu na verziju $3$. Lekseri i parseri za ulazne gramatike će u implementaciji biti generisani u programskom jeziku C\#.


\subsection{Preduslovi za pokretanje ANTLR4}
\label{subsec:ANTLRInstallation}

Kako bi ANTLR generisao parser u proizvoljnom programskom jeziku, potrebno je instalirati ANTLR i imati \emph{Java Runtime Environment} (skr. \emph{JRE}) instaliran na sistemu i dostupan globalno pokretanjem putem komande \texttt{java}. Instalacija se sastoji od preuzimanja najnovijeg \emph{.jar} fajla
\footnote{Takođe je moguće kompajlirati izvorni kod dostupan na servisu GitHub \url{https://github.com/antlr/antlr4}}, sa zvanične stranice \cite{ANTLR} ili recimo korišćenjem \emph{curl} alata: 
\begin{lstlisting}[language={}]
$ curl -O http://www.antlr.org/download/antlr-4-complete.jar
\end{lstlisting}

Na UNIX sistemima moguće je kreirati alias \texttt{antlr4} ili \emph{shell} skript unutar direktorijuma \texttt{/usr/local/bin} sa imenom \texttt{antlr4} koji će pokrenuti \emph{.jar} fajl na sledeći način (pretpostavljajući da se \emph{.jar} fajl nalazi u direktorijumu \texttt{/usr/local/lib}):
\begin{lstlisting}[language={}]
#!/bin/sh
java -cp "/usr/local/lib/antlr4-complete.jar:$CLASSPATH" org.antlr.v4.Tool $*
\end{lstlisting}

Na Windows sistemima moguće je kreirati \emph{batch} skript sa imenom \texttt{antlr4.bat} koji će pokrenuti ANTLR4, na sledeći način (pretpostavljajući da se \emph{.jar} fajl nalazi u direktorijumu \texttt{C:\textbackslash{}lib}):
\begin{lstlisting}[language={}]
java -cp C:\lib\antlr-4-complete.jar;%CLASSPATH% org.antlr.v4.Tool %*
\end{lstlisting}

Ukoliko su aliasi ili skript fajlovi imenovani kao iznad, moguće je iz komandne linije pojednostavljeno pokretati ANTLR4:  
\begin{lstlisting}[language={}]
$ antlr4
ANTLR Parser Generator Version 4.0
-o ___    specify output directory where all output is generated
-lib ___  specify location of .tokens files
...
\end{lstlisting}


\subsection{Generisanje parsera koristeći ANTLR4}
\label{subsec:ANTLRParserGeneration}

