\subsection{Čvorovi izraza}
\label{subsec:MyASTExpressionNodes}

Izraz, kao što se može videti na primeru gramatike sa slike \ref{fig:ANTLRExpressions}, se definiše rekurzivno i izraze mogu proširiti razni operatori. Na slici \ref{fig:ExpressionNodes} se mogu videti tipovi apstraktnih konstrukcija koje će se koristiti da bi se predstavili izrazi. Dodatno, za vezivanje izraza će se koristiti apstrakcije operatora definisane u prethodnom odeljku.

\begin{figure}[h!]
\centering
\includegraphics[scale=0.5]{images/expression_nodes.png}
\caption{Vrste čvorova izraza.}
\label{fig:ExpressionNodes}
\end{figure}

Najjednostavniji izraz predstavljaju konstante ili \emph{literali}. Literali mogu biti brojevne konstante, karakterske konstante ili konstantne niske. Literali često mogu imati i sufiks (najčešće za brojevne literale), koji određuje tip literala u slučajevima gde postoji dvosmislenost. Na primer, literal $5$ možemo posmatrati kao 32-bitni ceo broj ili kao 64-bitni ceo broj (ali i kao realan broj, ako ne zahtevamo da realne brojeve moramo pisati u nepokretnom ili pokretnom zarezu). Da bi se ova dvosmislenost uklonila, možemo eksplicitno naznačiti da se govori o 64-bitnom celom broju dodavanjem sufiksa \texttt{L}, ako je u pitanju programski jezik C ili njemu slični jezici. Takođe, pošto nezanemarljiv broj programskih jezika dozvoljava rad sa pokazivačima ili neposredno koristi alokaciju memorije za kreiranje objekata, uobičajeno je korišćenje prazne adrese kao specijalne vrednosti (\texttt{null} ili \texttt{nil}). Za ovakve vrednosti, ali i potencijalno druge vrednosti koje označavaju praznu vrednost, može se kreirati poseban tip literala, na slici \ref{fig:ExpressionNodes} nazvan \texttt{NULL literal} (ime je pozajmljeno od praznih pokazivača u programskim jezicima kao što je npr.~C, nije podržan pokazivački tip čvora u okviru apstrakcije).

Osim literala, samostalne promenljive mogu predstavljati validan izraz, u kom slučaju je vrednost izraza trenutna vrednost te promenljive. Slično važi i za indeksni pristup nizu\footnote{Isto važi i za bilo koju drugu kolekciju, ukoliko je nad njom definisan operator indeksnog pristupa. Predefinisanje ovog operatora nije razmatrano u ovom radu.}. U slučaju indeksnog pristupa, potrebno je navesti izraz čija vrednost označava indeks (to ne mora biti jednostavni literal). Postoje smislena ograničenja šta sve sme da se nađe unutar izraza koji predstavlja indeks elementa niza tako da to semantički ima smisla, ali se na ovom nivou ne bavimo semantičkom analizom. 

Unutar izraza se mogu naći i pozivi funkcija. Naravno, pretpostavljamo da funkcija ima povratnu vrednost, koja će se iskoristiti nakon poziva funkcije u kontekstu iz kojeg je ona pozvana. Iako je u ovom radu akcenat na imperativnoj paradigmi, neki funkcionalni koncepti su implicitno podržani zbog načina na koji je implementiran opšti AST --- operatori kompozicije funkcija (na način opisan u \ref{subsec:MyASTOperatorNodes}) i anonimne funkcije (koje se mogu smatrati validnim izrazima). Sa druge strane, poređenje funkcionalnog koda sa imperativnim kodom nije razmatrano. 

Operatori opisani u \ref{subsec:MyASTOperatorNodes} mogu vezati sve tipove iznad i formirati složenije izraze. U zavisnosti od broja izraza koje operator vezuje, izraze možemo podeliti na unarne i binarne. Unarne izraze nadograđuju unarni operatori dok su binarni izrazi dobijeni primenom binarnog operatora na dva izraza. U zavisnosti od tipa binarnog operatora (videti sliku \ref{fig:OperatorNodes}), binarne izraze delimo na sličan način. Naravno, svaki od tipova binarnog izraza zahteva odgovarajući tip binarnog operatora. Slično se može uraditi i za unarne izraze, ali takođe i napraviti podela na prefiksne i postfiksne unarne izraze. S obzirom da je cilj napraviti opšti AST, činjenica da li je unarni operator prefiksni ili postfiksni nije od suštinskog značaja, pogotovo ukoliko se uzme u obzir da dva programska jezika mogu imati unarne operatore sa istom semantikom ali različitom pozicijom u odnosu na operand --- u jednom jeziku taj operator može biti prefiksni a u drugom postfiksni. Kako bi poređenje ovakvih operatora funkcionisalo bez obzira na njihovu poziciju u odnosu na operand, u ovom radu nije pravljena podela na prefiksne i postfiksne unarne operatore.

Na slici \ref{fig:MyASTExampleExpressions} se mogu videti kreirani AST za izraz \texttt{(3 + 5) << f(4)}. Ovaj izraz poprima isti oblik bez obzira na to koji je programski jezik u pitanju, ali iako se sintaksa bude razlikovala ili operatori budu imali drugi simbol, logika operatora opisana putem funkcije će ostati ista.

\begin{figure}[h!]
\centering
\includegraphics[scale=0.6]{images/ast_expr.png}
\caption{AST generisan od izraza \texttt{(3 + 5) << f(4)}.}
\label{fig:MyASTExampleExpressions}
\end{figure}
