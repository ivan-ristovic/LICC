% Format teze zasnovan je na paketu memoir
% http://tug.ctan.org/macros/latex/contrib/memoir/memman.pdf ili
% http://texdoc.net/texmf-dist/doc/latex/memoir/memman.pdf
% 
% Prilikom zadavanja klase memoir, navedenim opcijama se podešava 
% veličina slova (12pt) i jednostrano štampanje (oneside).
% Ove parametre možete menjati samo ako pravite nezvanične verzije
% mastera za privatnu upotrebu (na primer, u b5 varijanti ima smisla 
% smanjiti 
\documentclass[12pt,oneside]{memoir} 

% Paket koji definiše sve specifičnosti master rada Matematičkog fakulteta
\usepackage[latinica]{matfmaster} 
%
% Podrazumevano pismo je ćirilica.
%   Ako koristite pdflatex, a ne xetex, sav latinički tekst na srpskom jeziku
%   treba biti okružen sa \lat{...} ili \begin{latinica}...\end{latinica}.
%
% Opcija [latinica]:
%   ako želite da pišete latiniciom, dodajte opciju "latinica" tj.
%   prethodni paket uključite pomoću: \usepackage[latinica]{matfmaster}.
%   Ako koristite pdflatex, a ne xetex, sav ćirilički tekst treba biti
%   okružen sa \cir{...} ili \begin{cirilica}...\end{cirilica}.
%
% Opcija [biblatex]:
%   ako želite da koristite reference na više jezika i umesto paketa
%   bibtex da koristite BibLaTeX/Biber, dodajte opciju "biblatex" tj.
%   prethodni paket uključite pomoću: \usepackage[biblatex]{matfmaster}
%
% Opcija [b5paper]:
%   ako želite da napravite verziju teze u manjem (b5) formatu, navedite
%   opciju "b5paper", tj. prethodni paket uključite pomoću: 
%   \usepackage[b5paper]{matfmaster}. Tada ima smisla razmisliti o promeni
%   veličine slova (izmenom opcije 12pt na 11pt u \documentclass{memoir}).
%
% Naravno, opcije je moguće kombinovati.
% Npr. \usepackage[b5paper,biblatex]{matfmaster}

% Pomoćni paket koji generiše nasumičan tekst u kojem se javljaju sva slova
% azbuke (nema potrebe koristiti ovo u pravim disertacijama)
\usepackage[latinica]{pangrami}

% Datoteka sa potrebnim paketima
\usepackage{listings}
\usepackage{textcomp}
\usepackage{xcolor}

% \setmonofont{Consolas}
\definecolor{bluekeywords}{rgb}{0,0,1}
\definecolor{greencomments}{rgb}{0,0.5,0}
\definecolor{redstrings}{rgb}{0.64,0.08,0.08}
\definecolor{xmlcomments}{rgb}{0.5,0.5,0.5}
\definecolor{types}{rgb}{0.17,0.57,0.68}

\usepackage{listings}
\lstset{
    language=csh,
    captionpos=b,
    numbers=left, 
    numberstyle=\tiny,
    frame=single,
    framesep=10pt,
    showspaces=false,
    showtabs=false,
    breaklines=true,
    showstringspaces=false,
    breakatwhitespace=true,
    escapeinside={(*@}{@*)},
    commentstyle=\color{greencomments},
    morekeywords={partial, var, value, get, set},
    keywordstyle=\color{bluekeywords},
    stringstyle=\color{redstrings},
    basicstyle=\ttfamily\small,
    tabsize=4,
    framexleftmargin=1.5em,
    xleftmargin=2em,
    % escapechar=\&,
    classoffset=1, 
    morekeywords={ abstract, event, new, struct,
    as, explicit, null, switch,
    base, extern, object, this,
    bool, false, operator, throw,
    break, finally, out, true,
    byte, fixed, override, try,
    case, float, params, typeof,
    catch, for, private, uint,
    char, foreach, protected, ulong,
    checked, goto, public, unchecked,
    class, if, readonly, unsafe,
    const, implicit, ref, ushort,
    continue, in, return, using,
    decimal, int, sbyte, virtual,
    default, interface, sealed, volatile,
    delegate, internal, short, void,
    do, is, sizeof, while,
    double, lock, stackalloc,
    else, long, static,
    enum, namespace, string, 
    >,<,.,;,,,-,!,=,~},
    classoffset=0,
}

% \lstset {
%     frame=single,
%     framesep=10pt,
%     xrightmargin=-25pt,
%     showstringspaces=false,
%     upquote=true,
%     commentstyle=\color{commentgreen},
%     keywordstyle=\color{blue},
%     stringstyle=\color{red},
%     basicstyle=\footnotesize\ttfamily,
%     emphstyle={\color{blue}},
%     escapechar=\&,
%     % keyword highlighting
%     classoffset=1, % starting new class
%     morekeywords={ abstract, event, new, struct,
%     as, explicit, null, switch,
%     base, extern, object, this,
%     bool, false, operator, throw,
%     break, finally, out, true,
%     byte, fixed, override, try,
%     case, float, params, typeof,
%     catch, for, private, uint,
%     char, foreach, protected, ulong,
%     checked, goto, public, unchecked,
%     class, if, readonly, unsafe,
%     const, implicit, ref, ushort,
%     continue, in, return, using,
%     decimal, int, sbyte, virtual,
%     default, interface, sealed, volatile,
%     delegate, internal, short, void,
%     do, is, sizeof, while,
%     double, lock, stackalloc,
%     else, long, static,
%     enum, namespace, string, 
%     >,<,.,;,,,-,!,=,~},
%     keywordstyle=\color{blue},
%     classoffset=0,
% }

% Datoteka sa literaturom u BibTex tj. BibLaTeX/Biber formatu
\bib{references}

% Ime kandidata na srpskom jeziku (u odabranom pismu)
\autor{Ivan Ristović}
% Naslov teze na srpskom jeziku (u odabranom pismu)
\naslov{Kreiranje zajedničke AST apstrakcije za različite programske jezike}
% Godina u kojoj je teza predana komisiji
\godina{2020}
% Ime i afilijacija mentora (u odabranom pismu)
\mentor{doc. dr Milena \textsc{Vujošević-Janičić}\\ Univerzitet u Beogradu, Matematički fakultet}
% Ime i afilijacija prvog člana komisije (u odabranom pismu)
\komisijaA{dr Ana \textsc{Anić}\\ University of Disneyland, Nedođija}
% Ime i afilijacija drugog člana komisije (u odabranom pismu)
\komisijaB{dr Laza \textsc{Lazić}\\ Univerzitet u Beogradu, Matematički fakultet}
% Ime i afilijacija trećeg člana komisije (opciono)
% \komisijaC{}
% Ime i afilijacija četvrtog člana komisije (opciono)
% \komisijaD{}
% Datum odbrane (odkomentarisati narednu liniju i upisati datum odbrane ako je poznat)
% \datumodbrane{}

% Apstrakt na srpskom jeziku (u odabranom pismu)
\apstr{%
\pangrami
}

% Ključne reči na srpskom jeziku (u odabranom pismu)
\kljucnereci{TODO}

\begin{document}
% ==============================================================================
% Uvodni deo teze
\frontmatter
% ==============================================================================
% Naslovna strana
\naslovna
% Strana sa podacima o mentoru i članovima komisije
\komisija
% Strana sa posvetom (u odabranom pismu)
\posveta{TODO zahvalnica}
% Strana sa podacima o disertaciji na srpskom jeziku
\apstrakt
% Sadržaj teze
\tableofcontents*

% ==============================================================================
% Glavni deo teze
\mainmatter
% ==============================================================================

\chapter{Uvod}
\label{chp:Intro}

Apstraktno sintaksičko stablo (engl. \emph{abstract syntax tree}, skr. \emph{AST}) programa ima značajnu ulogu u procesu kreiranja izvršivog programa od izvornog koda. AST nastaje u fazi sintaksičke analize (ili \emph{fazi rasčlanjavanja}) kao rezultat apstrahovanja stabla sintaksičke analize dobijenog od strane sintaksičkog analizatora (skr.~\emph{parsera}). Parser čita izvorni k\^od i pokušava da u njemu pronađe primene određenih pravila jezika čiji je k\^od dizajniran da raščlani. Svaki programski jezik ima specifična sintaksna pravila pa su stoga i skupovi pravila (tzv. \emph{gramatike}) programskih jezika raznorodni, što se zatim prenosi i na generisana stabla sintaksičke analize. Stablo sintaksičke analize se apstrahuje tako što se iz njega izvuku samo bitne sintaksičke a uklone neke tehničke informacije.

Ovakva apstrakcija se najpre koristi u semantičkoj analizi programa koju vrši prevodilac nakon faze sintaksičke analize i provere sintaksičke ispravnosti koda. Ukoliko program prođe semantičke provere, prelazi se na prevođenje u međureprezentaciju i fazu optimizacije. Nakon faze optimizacije sledi generisanje asemblerskog koda koji se zatim prevodi u mašinski k\^od.

AST, zbog svoje uloge u semantičkoj analizi, može poslužiti i za analizu programa pre samog prevođenja, kroz proces poznat pod nazivom \emph{statička analiza}. Posmatranje programa kroz AST pruža mogućnost za poređenje dva programa na apstraktnom nivou. Jedna primena ove ideje u okviru statičke analize može biti provera semantičke ekvivalentnosti. Provera semantičke ekvivalentnosti dva programa je neodlučiv problem u opštem slučaju, međutim pod određenim pretpostavkama koje pojednostavljuju problem moguće je dizajnirati algoritme koji daju smislene rezultate u praksi. Jedna od često korišćenih pretpostavki je pretpostavka sličnosti strukture dva programa. Interesantno, provera da li dva programa zadovoljavaju ovu premisu se može proveriti posmatranjem izgleda i sličnosti u strukturi na apstraktnom nivou --- problem koji se može rešiti primenom algoritama za rad sa stablima jer je u pitanju AST (ali i grafovima uopšte, jer je stablo specijalizacija grafa).

Iako je AST reprezentacija neophodna za kompilaciju i primenjiva za neke druge vrste problema, ipak je specifična za konkretni programski jezik s obzirom da nastaje od stabla sintaksičke analize koje je usko vezano za gramatiku konkretnog programskog jezika. Motivacija za ovaj rad dolazi od nepostojanja opštih apstrakcija sintaksičkih stabala koje bi se mogle koristiti za analizu programa napisanih u različitim programskim jezicima. Iako je broj programskih jezika danas veoma veliki, u okviru iste programske paradigme jezici moraju implementirati koncepte koji su potrebni da bi se programiralo u toj paradigmi i ta zajednička svojstva se mogu iskoristiti za formiranje zajedničke apstrakcije. 

U ovom radu će biti predstavljena opšta AST apstrakcija za imperativne programske jezike, sa ciljem da se omogući zajednička apstraktna reprezentacija velikog broja imperativnih jezika, pa čak i onih koji pripadaju skript paradigmi. Njena upotreba će biti demonstrirana na problemu semantičke ekvivalentnosti dobijenih apstrakcija kroz naivni algoritam poređenja simboličkih promenljivih. Štaviše, na apstraktnom nivou nije važno od kog se programskog jezika dobio AST, što može imati primenu u procesu migracije na nove tehnologije. Na slici \ref{fig:IntroExample} se mogu videti primeri dve funkcije pisane u različitim programskim jezicima koje se mogu apstrahovati tako da imaju skoro identičan AST. Razlike koje moraju postojati u ovom slučaju su tipovi argumenata --- u drugoj funkciji nije zagarantovano da argumenti (ali i povratna vrednost) moraju biti tipa \texttt{int}. Treba napomenuti da ovakvo apstrahovanje često dovodi do gubitka informacija --- u dobijenim apstrakcijama funkcija sa slike \ref{fig:IntroExample} nisu poznati konkretne vrste petlji od kojih se dobila apstraktna petlja (u opštem slučaju je moguće sve vrste petlji svesti na jednu).

\begin{figure}[h!]
\begin{lstlisting}
void array_sum(int[] arr, int n) {
    int sum = 0, i = 0;
    while (i < n) {
        sum += arr[i];
        i++;
    }
    return sum;
}
\end{lstlisting}
\begin{lstlisting}
function array_sum(arr, n)
    local sum = 0
    for i,v in ipairs(arr) do
        sum = sum + v
    end
    return sum
end
\end{lstlisting}
\caption{Segmenti koda pisani u različitim programskim jezicima (C gore, i Lua dole) koji se mogu apstrahovati tako da imaju identični AST.}
\label{fig:IntroExample}
\end{figure}

Naravno, semantička ekvivalentnost se ne mora zasnivati na apstrahovanju programa, već se takođe često rešava spuštanjem na nivo međukoda između višeg programskog jezika i asemblera. U nekim slučajevima se može ići i do asemblera pa i mašinskog jezika. Ukoliko bi se posmatrali asemblerski ili mašinski k\^od, vršilo bi se poređenje kodova prilagođenih određenoj arhitekturi procesora. Neki moderni radni okviri kao što je \emph{Microsoft .NET} radni okvir, imaju kao svoju komponentu i virtualnu mašinu na kojoj se izvršavaju programi koji koriste taj radni okvir, bez obzira na programski jezik u kojem su ti programi napisani. Virtualna mašina prevodi međureprezentacije  programa dobijene od prevodioca u mašinski jezik i izvršava ih. Međutim, iako je međukod isti, AST programa pisanih u različitim jezicima i dalje nije. U ovom radu je odabran pristup zasnovan na AST, s obzirom na važnosti i značaj apstraktnih sintaksičkih stabala, ali i zbog nedostatka opštih apstrakcija.

U poglavlju \ref{chp:RelevantTerms} će biti opisani relevantni pojmovi potrebni za razumevanje rada uz akcenat na apstraktnim sintaksičkim stablima i procesu njihovog dobijanja. Opšta AST apstrakcija za imperativne jezike biće opisana u poglavlju \ref{chp:MyAST}, a njena upotreba u problemu odlučivanja semantičke ekvivalentnosti kao i sam algoritam za poređenje opštih apstrakcija biće opisani u poglavlju \ref{chp:ASTComparing}. Implementacija apstrakcije i algoritma semantičkog poređenja će biti opisana u poglavlju \ref{chp:Implementation}. Na kraju, biće dati glavni zaključci ovog rada kao i moguća unapređenja i budući koraci. 

\chapter{Pregled relevantnih pojmova}
\label{chp:RelevantTerms}

U ovom poglavlju će biti opisani koncepti i alati čije je razumevanje potrebno kako bi se razumeo opis dalje apstrakcije i implementacije samog programa. Umesto analize samog sadržaja izvornog koda analizira se \emph{apstraktno sintaksičko stablo}, opisano u odeljku \ref{sec:AST}. Kako bi se od izvornog koda došlo do stabla parsiranja a potom i do apstraktnog sintaksičkog stabla, koriste se \emph{lekseri} i \emph{parseri}. S obzirom da je cilj kreirati univerzalnu reprezentaciju, biće neophodno kreirati leksere i parsere za proizvoljne gramatike. Više reči o samom procesu dobijanja stabla parsiranja od izvornog koda i alatima koji mogu generišu leksere i parsere biće u odeljku \ref{sec:ParsingGrammars}, sa akcentom na alat \emph{Another Tool For Language Recognition} \cite{ANTLR}, u daljem tekstu \emph{ANTLR}, opisan u odeljku \ref{subsec:ANTLR}. Osim pojmova usko povezanih sa apstraktnim sintaksičkim stablima, u odeljku \ref{sec:Symbolics} će takođe biti opisani i koncepti \emph{simboličkog izračunavanja}, zbog njihove primene u procesu testiranja semantičke ekvivalentnosti, kao i \emph{obrasci za projektovanje} u odeljku \ref{sec:DesignPatterns} koji pružaju gotova rešenja za česte probleme koji se u ovom radu koriste za pružanje interfejsa obilaska stabala i izračunavanja vrednosti nad istim.

\section{Apstraktna sintaksna stabla - AST}
\label{sec:AST}

Kako bi se od datoteke na fajl sistemu koja sadrži izvorni kod programa 
došlo do izvršivog programa, potrebno je izvršiti više koraka 
\cite{CompilerConstruction}:
\begin{itemize}
    \item pretprocesiranje
    \item prevođenje
    \item asembliranje
    \item linkovanje
\end{itemize}

Ovi koraci će biti opisani na jednom primeru. Pretpostavimo da želimo 
da kompajliramo kod pisan u programskom jeziku C prikazan na slici 
\ref{fig:CompilationProcessInit}. Primetimo, da postoji greška u datom
kodu - simbol \texttt{c} koji se koristi će biti prepoznat kao 
identifikator koji ne odgovara nijednoj promenljivoj. Ovo, doduše, nije
sintaksna greška - izraz \texttt{a + c} je sasvim validan u programskom
jeziku C bez analize konteksta u kom se javlja. Problem će postati 
očigledan tek nakon parsiranja izvornog koda i provere ispunjenosti 
sintaksih pravila. Ovakve greške se nazivaju \emph{semantičke greške}.

\begin{figure}[h!]
    \begin{lstlisting}
    #include<stdio.h>

    #define T int

    int main()
    {
        T a, b;
        a = a + c;        // c nije deklarisano
        printf("%d", a);
        return 0;
    }
    \end{lstlisting}
    \caption{Primer izvornog koda pisanog u programskom jeziku C.}
    \label{fig:CompilationProcessInit}
\end{figure}

U fazi pretprocesiranja se vrše samo tekstualne operacije kao što su
brisanje komentara ili zamena makroa u jezicima kao što je C. Prvo 
mesto gde se vrši analiza sadržaja izvornog fajla je faza prevođenja.
Tu analizu vrši program koji se naziva \emph{pretprocesor}. Rezultat 
rada pretprocesora za kod sa slike \ref{fig:CompilationProcessInit} 
bi izgledao kao na slici \ref{fig:CompilationProcessPrep} \footnote{
U nekim implementacijama C standardne biblioteke, moguće je da se 
poziv funckije \texttt{printf} zameni pozivom funkcije \texttt{fprintf}
sa ispisom na \texttt{stdout}. U standardu se propisuje da funkcije 
kao što je \texttt{printf} mogu biti implementirane kao makroi. Izlaz 
na slici \ref{fig:CompilationProcessPrep} je generisan od strane 
\texttt{GCC 7.4.0} po C11 standardu.} i ovo nije slučaj u datom 
okruženju.

\begin{figure}[h!]
    \begin{lstlisting}
    int main()
    {
        int a, b;
        a = a + c;
        printf("%d", a);
        return 0;
    }
    \end{lstlisting}
    \caption{Rezultat rada pretprocesora za kod sa slike 
             \ref{fig:CompilationProcessInit}.}
    \label{fig:CompilationProcessPrep}
\end{figure}

Prilikom faze prevođenja, kako prevodilac ne bi radio nad sirovim 
karakterima izvornog koda, potrebno je izvršiti pripremu istog. 
Prevodilac ima u vidu moguće elemente programskog jezika, tzv. 
\emph{tokene}, koje treba prepoznati u datom fajlu - ključne reči, 
operatore, promenljive itd. Program koji radi \emph{tokenizaciju} -
prepoznavanje tokena u izvornom fajlu - se naziva \emph{lekser}. 
Pojednostavljen primer tokena koje lekser pokušava da prepozna 
se može videti na slici \ref{fig:CLexerExample}. Primer izlaza
leksera za izlaz pretprocesora sa slike \ref{fig:CompilationProcessPrep}
se može videti na slici \ref{fig:CompilationProcessLex}.

\begin{figure}[h!]
    \begin{lstlisting}
    Identifier : IdentifierNondigit 
                 (IdentifierNondigit | Digit)*
               ;

    IdentifierNondigit : Nondigit
                       | UniversalCharacterName
                       ;

    Nondigit : [a-zA-Z_]
             ;

    Digit : [0-9]
          ;
    \end{lstlisting}
    \caption{Primer delimične definicije tokena za ime promenljive po C11 standardu.}
    \label{fig:CLexerExample}
\end{figure}

\begin{figure}[h!]
    \begin{lstlisting}
    identifier 'main'	 [LeadingSpace]	Loc=<sample.c:3:5>
    l_paren '('		Loc=<sample.c:3:9>
    r_paren ')'		Loc=<sample.c:3:10>
    l_brace '{'	 [StartOfLine]	Loc=<sample.c:4:1>
    int 'int'	 [StartOfLine] [LeadingSpace]	Loc=<sample.c:5:5>
    identifier 'a'	 [LeadingSpace]	Loc=<sample.c:5:9>
    comma ','		Loc=<sample.c:5:10>
    identifier 'b'	 [LeadingSpace]	Loc=<sample.c:5:12>
    semi ';'		Loc=<sample.c:5:13>
    identifier 'a'	 [StartOfLine] [LeadingSpace]	Loc=<sample.c:6:5>
    equal '='	 [LeadingSpace]	Loc=<sample.c:6:7>
    identifier 'a'	 [LeadingSpace]	Loc=<sample.c:6:9>
    plus '+'	 [LeadingSpace]	Loc=<sample.c:6:11>
    identifier 'c'	 [LeadingSpace]	Loc=<sample.c:6:13>
    semi ';'		Loc=<sample.c:6:14>
    identifier 'printf'	 [StartOfLine] [LeadingSpace]	Loc=<sample.c:7:5>
    l_paren '('		Loc=<sample.c:7:11>
    string_literal '"%d"'		Loc=<sample.c:7:12>
    comma ','		Loc=<sample.c:7:16>
    identifier 'a'	 [LeadingSpace]	Loc=<sample.c:7:18>
    r_paren ')'		Loc=<sample.c:7:19>
    semi ';'		Loc=<sample.c:7:20>
    return 'return'	 [StartOfLine] [LeadingSpace]	Loc=<sample.c:8:5>
    numeric_constant '0'	 [LeadingSpace]	Loc=<sample.c:8:12>
    semi ';'		Loc=<sample.c:8:13>
    r_brace '}'	 [StartOfLine]	Loc=<sample.c:9:1>
    eof ''		Loc=<sample.c:9:2>
    \end{lstlisting}
    \caption{Primer delimične definicije tokena za ime promenljive po standardu C11.}
    \label{fig:CompilationProcessLex}
\end{figure}

Nakon završetka rada leksera potrebno je parsirati dobijene tokene.
Parsiranje vrši program koji se naziva \emph{parser}. Parser, slično
kao što lekser ima definicije tokena jezika, mora imati informacije 
o gramatici jezika. Gramatika programskog jezika se najčešće definiše
putem kontekstno-slobodnih gramatika \cite{ContextFreeGrammars}, 
čiji je primer dat na slici \ref{fig:CompilationProcessGram}.

\begin{figure}[h!]
    \begin{lstlisting}
    functionDefinition
        :   declarationSpecifiers? declarator declarationList? compoundStatement
        ;

    declarationList
        :   declaration
        |   declarationList declaration
        ;

    declaration
        :   declarationSpecifiers initDeclaratorList ';'
        | 	declarationSpecifiers ';'
        |   staticAssertDeclaration
        ;
    \end{lstlisting}
    \caption{Isečak gramatike programskog jezika C po standardu C11.}
    \label{fig:CompilationProcessLex}
\end{figure}

Izlaz rada parsera je \emph{stablo parsiranja} (eng. \emph{parse tree} 
ili \emph{derivation tree}). Takvo stablo i dalje sadrži sve relevantne
informacije o izvornom kodu. Vizuelni prikaz rada parsera za gramatiku
sa slike C11 i izvonog koda sa slike \ref{fig:CompilationProcessPrep} je
dat na slici \ref{fig:CompilationProcessPars}.

\begin{figure}[h!]
    \includegraphics{images/}
    \caption{Isečak gramatike programskog jezika C po standardu C11.}
    \label{fig:CompilationProcessLex}
\end{figure}
\input{chapters/24_symbolics.tex}
\section{Obrasci za projektovanje}
\label{sec:DesignPatterns}

\emph{Obrasci za projektovanje} (engl. \emph{design patterns} \cite{DesignPatterns}, drugačije nazvani i \emph{projektni šabloni, uzorci}) predstavljaju opšte i ponovno upotrebljivo rešenje čestog problema, obi;no implementirani kroz koncepte objektno-orijentisanog programiranja. Svaki obrazac za projektovanje ima četiri osnovna elementa:
\begin{itemize}
    \item ime - ukratko opisuje problem, rešenje i posledice
    \item problem - opisuje slučaj u kome se obrazac koristi
    \item rešenje - opisuje elemente dizajna i odnos tih elemenata
    \item posledice - obuhvataju rezultate i ocene primena obrasca
\end{itemize}

Obrasce za projektovanje je moguće grupisati po situaciji u kojoj se mogu iskoristiti ili načinu na koji rešavaju zadati problem. Stoga je opšte prihvaćena podela na tri grupe:
\begin{itemize}
    \item \emph{gradivni obrasci} (engl. \emph{creational patterns})
    \item \emph{strukturni obrasci} (engl. \emph{structural patterns})
    \item \emph{obrasci ponašanja} (engl. \emph{behavioral patterns})
\end{itemize}

Gradivni obrasci apstrahuju proces pravljenja objekata i važni su kada sistemi više zavise od sastavljanja objekata nego od nasleđivanja. Neki od najvažnijih gradivnih obrazaca su \emph{apstraktna fabrika} (engl. \emph{abstract factory}), \emph{graditelj} (engl. \emph{builder}), \emph{proizvodni metod} (engl. \emph{factory method}), \emph{prototip} (engl. \emph{prototype}) i \emph{unikat} (engl. \emph{singleton}). Strukturni obrasci se bave načinom na koji se klase i objekti sastavljaju u veće strukture. Neki od najvažnijih strukturnih obrazaca su \emph{adapter} (engl. \emph{adapter}), \emph{most} (engl. \emph{bridge}), \emph{sastav} (engl. \emph{composite}), \emph{dekorater} (engl. \emph{decorator}), \emph{fasada} (engl. \emph{facade}), \emph{muva} (engl. \emph{flyweight}) i \emph{proksi} (engl. \emph{proxy}). Obrasci ponašanja se bave načinom na koji se klase i objekti sastavljaju u veće strukture. Neki od najvažnijih strukturnih obrazaca su \emph{lanac odgovornosti} (engl. \emph{chain of responsibility}), \emph{komanda} (engl. \emph{command}), \emph{interpretator} (engl. \emph{interpreter}), \emph{iterator} (engl. \emph{iterator}), \emph{posmatrač} (engl. \emph{observer}), \emph{strategija} (engl. \emph{strategy}) i \emph{posetilac} (engl. \emph{visitor}).

Za potrebe ovog rada, obrasci za projektovanje će se koristiti kao opšte prihvaćeno i programerski intuitivno rešenje određenih problema. Takođe, u kontekstu stabala parsiranja i AST-ova opisanih u poglavlju \ref{sec:AST}, obrasci \emph{Posmatrač} i \emph{Posetilac} su od velikog značaja jer pružaju interfejs za obilazak takvih stabala. Ovi obrasci se koriste od strane alata korišćenih u ovom radu kao što je ANTLR, opisan u poglavlju \ref{subsec:ANTLR}. Takođe, s obzirom da su ovi obrasci opšte-prihvaćeno rešenje za pružanje interfejsa obilaska stabala, biće korišćeni i u implementaciji opšte apstrakcije. U nastavku će zbog opisanih razloga biti opisani samo obrasci posmatrač i posetilac, dok zainteresovani čitalac može pročitati više u \cite{DesignPatternsBook}.

\subsection{Obrazac "Posmatrač"}
\label{subsec:DesignPatternsObserver}

Obrazac za projektovanje \emph{Posmatrač} je strukturni obrazac za projektovanje koji se koristi kada je potrebno definisati jedan-ka-više vezu između objekata tako da ukoliko jedan objekat promeni stanje (subjekat) svi zavisni objekti su obavešteni o izmeni i shodno ažurirani. Posmatrač predstavlja \emph{pogled} (engl. \emph{View}) u MVC (engl. \emph{Model-View-Controller}) arhitekturi. Na slici \ref{fig:UMLObserver} se može videti UML dijagram \cite{UML} ovog obrasca. 

\begin{figure}[h!]
\centering
\includegraphics[scale=0.7]{images/observer.jpg}
\caption{UML dijagram obrasca za projektovanje "Posmatrač".} 
\label{fig:UMLObserver}
\end{figure}

Primer upotrebe ovog obrasca može biti aukcija gde je aukcionar subjekat i započinje aukciju, dok učesnici aukcije (objekti) posmatraju aukcionera i reaguju na podizanje cene. Prihvatanje promene cene menja trenutnu cenu i aukcioner oglašava promenu iste, a svi učesnici aukcije dobijaju informaciju da se izmena izvršila. Za potrebe ovog rada, primer upotrebe može biti obilazak stablolike kolekcije (recimo stabla parsiranja) i obaveštavanje o nailasku na čvorove određenih tipova. Te informacije se dalje mogu iskoristiti za izračunavanja nad pomenutom strukturom ili generisanje novih struktura (recimo AST). 

\subsection{Obrazac "Posetilac"}
\label{subsec:DesignPatternsListener}

Obrazac za projektovanje \emph{Posetilac} je strukturni obrazac za projektovanje koji predstavlja operaciju koju je potrebno izvesti nad elementima objektne strukture. Posetilac omogućava definisanje nove operacije bez izmena klasa elemenata nad kojima operiše. Operacija koja će se izvesti zavisi od imena zahteva, tipa posetioca i tipa elementa kog posećuje. Na slici \ref{fig:UMLVisitor} se može videti UML dijagram \cite{UML} ovog obrasca. 

\begin{figure}[h!]
    \centering
    \includegraphics[scale=0.6]{images/visitor.jpg}
    \caption{UML dijagram obrasca za projektovanje "Posetilac".} 
    \label{fig:UMLVisitor}
\end{figure}

Primer upotrebe ovog obrasca može biti operisanje taksi kompanija. Kada osoba pozove taksi kompaniju (prihvatanje posetioca), kompanija šalje vozilo osobi koja je pozvala kompaniju. Nakon ulaska u vozilo (posetilac), mušterija ne kontroliše svoj transport već je to u rukama taksiste (posetioca). Za potrebe ovog rada, primer upotrebe može biti prikupljanje informacija o kolekciji stablolike strukture (recimo stablo parsiranja) i korišćenje istih za neko izračunavanje ili generisanje novih struktura (recimo AST). 

\input{chapters/23_parsing.tex}
\section{Programske paradigme i gramatičke razlike programskih jezika}
\label{sec:Paradigms}

Iako se u suštini svode na mašinski jezik ili asembler, viši programski jezici mogu imati velike razlike međusobno - kako u načinu pisanja koda, tako i u efikasnosti izvršavanja. Način, ili stil programiranja se naziva \emph{programska paradigma} \cite{ProgrammingParadigms}. Može se pokazati da sve što je rešivo putem jedne, može i da se reši i putem ostalih; međutim neki problemi se prirodnije rešavaju koristeći specifične paradigme. Neke poznatije programske paradigme su navedene u nastavku zajedno sa njihovim odlikama i primerima upotrebe.

\subsection{Imperativna paradigma}
\label{subsec:ParadigmImperative}

\emph{Imperativna paradigma} pretpostavlja da se promene u trenutnom stanju izvršavanja mogu sačuvati kroz promenljive. Izračunavanja se vrše putem niza koraka, u svakom koraku se te promenljive referišu ili se menjaju njihove trenutne vrednosti. Raspored koraka je bitan, jer svaki korak može imati različite posledice s obzirom na trenutne vrednosti promenljivih na početku tog koraka. Primer koda pisanog po imperativnoj paradigmi se može videti na slici \ref{fig:ParadigmImperative}.

\begin{figure}[h!]
\begin{lstlisting}
    result = []
    i = 0
start:
    numPeople = length(people)
    if i >= numPeople goto finished
    p = people[i]
    nameLength = length(p.name)
    if nameLength <= 5 goto nextOne
    upperName = toUpper(p.name)
    addToList(result, upperName)
nextOne:
    i = i + 1
    goto start
finished:
    return sort(result)
\end{lstlisting}
\caption{Primer koda pisanog po imperativnoj paradigmi.}
\label{fig:ParadigmImperative}
\end{figure}

Stariji programski jezici najčešće prate ovu paradigmu više nego bilo koju drugu iz par razloga. Prvi je taj što imperativna paradigma najbliže oslikava samu mašinu na kojoj se program izvršava, pa je programer mnogo "bliži" istoj. Posledica ovog pristupa, a to je i drugi razlog za popularnost imperativne paradigme, je omogućila da je ova paradigma bila najefikasnija zbog ograničenja u hardveru. Danas, zbog mnogo bržeg razvoja i mnogo jačih računara, efikasnost se sve manje i manje uzima u obzir.

Naravno, imperativna paradigma ima i svoje nedostatke. Naime, najveći problem je razumevanje i verifikovanje semantike programa zbog postojanja sporednih efekata \footnote{Sporedni efekti (promena stanja mašine) ne poštuju \emph{referencijalnu transparentnost} koja se definiše na sledeći način: \emph{Ako važi $P(x)$ i $x = y$ u nekom trenutku, onda $P(x) = P(y)$ važi tokom čitavog vremena izvršavanja programa}.}. Stoga je i pronalaženje grešaka u kodovima koji prate imperativnu paradigmu znatno komplikovanije. Apstrakcija je takođe više ograničena u imperativnoj nego u ostalim paradigmama. Na kraju, redosled izvršavanja je vrlo bitan, što neke probleme čini težim ukoliko se pokušaju rešiti pomoću imperativne paradigme.

\subsection{Strukturna paradigma}
\label{subsec:ParadigmImperativeStructural}

\emph{Strukturna paradigma} je vrsta imperativne paradigme gde se kontrola toka vrši putem ugnježdenih petlji, uslovnih grananja i podrutina. Promenljive su obično lokalne za blok u kome su definisane, što određuje i njihov životni vek i vidljivost. Primer koda pisanog po imperativnoj paradigmi se može videti na slici \ref{fig:ParadigmStructural}. Najpopularniji derivat strukturne paradigme je \emph{proceduralna paradigma}, bazirana na konceptu poziva \emph{procedure} - podrutine ili funkcije koja sadrži seriju koraka koje je potrebno izvršiti redom.

\begin{figure}[h!]
\begin{lstlisting}
result = [];
for i = 0; i < length(people); i++ {
    p = people[i];
    if length(p.name)) > 5 {
        addToList(result, toUpper(p.name));
    }
}
return sort(result);
\end{lstlisting}
\caption{Primer koda pisanog po strukturnoj paradigmi.}
\label{fig:ParadigmStructural}
\end{figure}


\subsection{Logička paradigma}
\label{subsec:ParadigmLogical}

\emph{Logička paradigma} koristi deklarativni pristup rešavanju problema. Umesto zadavanja instrukcija koje treba da dovedu do rezultata, opisuje se sam rezultat kroz činjenice - skup logičkih pretpostavki. Taj opis se zatim prevodi u upit koji se dalje koristi. Uloga računara je održavanje i logička dedukcija. Logička paradigma se deli u tri sekcije:
\begin{itemize}
    \item niz deklaracija i definicija koje opisuju problem iz nekog domena,
    \item relevantne činjenice i
    \item relevantni ciljevi u formi upita.
\end{itemize}

Bilo koji rezultat dedukcije rešenja upita predstavlja rezultat izvršavanja. Deklaracije i definicije se konstruišu iz relacija, npr. $X \in Y$ ili $X \in [a,b]$. Prednost ovog pristupa je mala količina programiranja, pošto dedukcioni sistem traži rešenje problema. Takođe, verifikovanje validnosti je stoga trivijalno. Primer koda pisanog po logičkoj paradigmi se može videti na slici \ref{fig:ParadigmLogical}.

\begin{figure}[h!]
\begin{lstlisting}
domains
    being = symbol 
predicates
    animal(being)        % sve životinje su ziva bica
    dog(being)           % svi psi su ziva bica
    die(being)           % svi ziva bica umiru 
clauses
    animal(X) :- dog(X)  % svi psi su zivotinje
    dog(fido).           % fido je pas
    die(X) :- animal(X)  % sve životinje umiru 
\end{lstlisting}
\caption{Primer koda pisanog po logičkoj paradigmi.}
\label{fig:ParadigmLogical}
\end{figure}


\subsection{Funkcionalna paradigma}
\label{subsec:ParadigmFunctional}

\emph{Funkcionalna paradigma} posmatra sve potprograme kao funkcije u matematičkom smislu - uzimaju argumente i vraćaju jedinstven rezultat. Povratna vrednost zavisi isključivo od argumenata, što znači da je nebitan trenutak u kom je funkcija pozvana. Izračunvanja se vrše primenom aplikacije funkcija, kompozicijom funkcija i redukcijom. Primer koda pisanog po funkcionalnoj paradigmi se može videti na slici \ref{fig:ParadigmFunctional}.

\begin{figure}[h!]
\begin{lstlisting}
people 
    |> map    (extract_name . to_upper) 
    |> filter (\name -> length name > 5) 
    |> sort
    |> take 5
    |> join ", "
\end{lstlisting}
\caption{Primer koda pisanog po funkcionalnoj paradigmi.}
\label{fig:ParadigmFunctional}
\end{figure}
    
Funkcionalni programski jezici se baziraju na funkcionalnoj paradigmi. Takvi jezici dozvoljavaju tretiranje funkcija kao \emph{građana prvog reda} - mogu biti tretirane kao podaci pa se mogu proslediti drugim funkcijama ili vratiti kao rezultat izračunavanja drugih funkcija. Prednosti funkcionalnih jezika su visok nivo apstrakcije, što prevazilazi mnogo detalja programiranja i stoga eliminiše pojavu velikog broja grešaka, nezavisnost od redosleda izračunavanja, što omogućava paralelizam, i formalnu matematičku verifikaciju. Mane su potencijalna manja efikasnost, što danas predstavlja manji problem, kao i teškoća implementacije specifične sekvencijalne aktivnosti ili potreba za stanjem, što bi se lako implementiralo imperativno ili preko OO paradigme.


\subsection{Objektno-orijentisana paradigma}
\label{subsec:ParadigmOOP}

\emph{Objektno-orijentisana paradigma} (kraće \emph{OOP}) je paradigma u kojoj se objekti stvarnog sveta posmatraju kao zasebni entiteti koji imaju sopstveno stanje koje se modifikuje samo pomoću procedura ugrađenih u same objekte - tzv. \emph{metode}. Posledica zasebnog operisanja objekata omogućava njihovu enkapsulaciju u module koji sadrže lokalnu sredinu i metode. Komunikacija sa objektom se vrši prosleđivanjem poruka. Objekti su organizovani u klase, od kojih nasleđuju metode i ekvivalentne promenljive. OOP omogućava ponovnu iskorišćenost koda i ekstenzibilnost koda. Primer koda po strukturnoj paradigmi je dat na slici \ref{fig:ParadigmOOP}. 

\begin{figure}[h!]
\begin{lstlisting}
abstract class Employee
{
    private String name;

    Employee(String name) {
        this.name = name;
    }

    abstract void work();
};

class WageEmployee extends Employee
{
    private double wage;
    private double hours;

    WageEmployee(String name) {
        super(name);
        this.name = name;
    }

    void work() {
        wage += 200;
        hours += 8;
    }
};

var bill = new WageEmployee("Bill Gates");
bill.work();
\end{lstlisting}
\caption{Primer koda pisanog po OO paradigmi u programskom jeziku Java.}
\label{fig:ParadigmOOP}
\end{figure}

Nova klasa ($A$) može \emph{naslediti} ili \emph{konkretizovati} drugu klasu ($B$). $B$ se zove \emph{bazna klasa} ili \emph{natklasa}, dok se $A$ naziva \emph{izvedena klasa} ili \emph{potklasa}. Izvedena klasa nasleđuje sve odlike bazne klase - strukturu i ponašanje - postaje specijalizacija bazne klase dok bazna klasa postaje generalizacija svoje potklase. Osim nasleđenih osobina, potklasa može imati dodatna stanja (instancne promenljive) ili dodatna ponašanja (metode). Dozvoljeno je i predefinisanje ponašanja bazne klase. Mehanizam nasleđivanja je dozvoljen i ukoliko nije dozvoljen pristup izvornom kodu bazne klase.

Idealno, stanju objekta može da pristupi i modifikuje samo pomoću metoda tog objekta. Većina OO jezika dozvoljava direktnu manipulaciju stanja ali taj pristup nije preporučen. Kako bi se enkapsulacija i skrivanje informacija kao najveće prednosti OO paradigme mogle iskoristiti, interfejs klase (kako se pristupa objektima) bi trebalo da bude odvojen od implementacije klase (izvornog koda metoda klasa).


\subsection{Skript paradigma i njen odnos sa proceduralnom paradigmom}
\label{subsec:Languages}

Čak i unutar jedne paradigme kao što je proceduralna, mogu se naći veoma velike varijacije u izgledu koda pisanog u različitim programskim jezicima koji prate proceduralnu paradigmu. Kako hardver postaje moćniji, više se ceni vreme koje programer provede u procesu pisanja koda nego koliko je taj kod efikasan. Štaviše, u nekim slučajevima je dobitak u efikasnosti veoma mali u poređenju sa vremenom koje je potrebno utrošiti da bi se ta efikasnost postigla. Ukoliko se program pokreće veoma retko, možda nije ni bitno da li se on izvršava sekundu sporije od efikasnog programa, ako je za njegovo pisanje utrošeno znatno manje vremena. Ovo je pristup koji prate \emph{skript} jezici kao što su \texttt{Python, Perl, bash} itd. Iako proceduralni, oni se razlikuju od klasičnih predstavnika proceduralne paradigme i njihove razlike su vremenom postale tolike da nije neuobičajeno da se skript jezici svrstaju u zasebnu, \emph{skript paradigmu}. Stoga će se u nastavku, pod terminom \emph{proceduralni jezik} smatrati tradicionalni proceduralni jezik, ukoliko nije naznačeno drugačije. Na slici \ref{fig:LanguagesDiff} se mogu uočiti navedene razlike.

\begin{figure}[h!]
\begin{lstlisting}
int main() {
    int k = 0;
    for (int i = 0; i < 1000000; i++)
        k++;
    return 0;
}
\end{lstlisting}
\begin{lstlisting}[language={}]
$ time: 0.03s user 0.00s system 70% cpu 0.044 total
\end{lstlisting}
\begin{lstlisting}
k = 0
for i in range(1000000):
    k += 1
\end{lstlisting}
\begin{lstlisting}[language={}]
$ time: 0.16s user 0.03s system 93% cpu 0.200 total
\end{lstlisting}
\caption{Primer koda pisanog po tradicionalnoj proceduralnoj paradigmi (gore, \texttt{C}) i po modernoj skript paradigmi (gore, \texttt{Python 3}) kao i odgovarajuća vremena izvršavanja dobijena komandom \texttt{time}.}
\label{fig:LanguagesDiff}
\end{figure}

Promenljive predstavljaju jedan od osnovnih koncepata na kojem se zasnivaju i proceduralni i skript jezici. Promenljivu odlikuje, između ostalog, i njen \emph{tip} koji određuje količinu memorije potrebnu za njeno skladištenje. Proceduralni programski jezici zahtevaju definisanje tipa promenljive i obično su i \emph{statički}, što znači da promenljive ne mogu menjati svoj tip tokom izvršavanja programa. Proces uvođenja imena promenljive se u naziva \emph{deklaracija promenljive}. Slično kao i za promenljive, potrebno je deklarisati i funkcije pre trenutka njihovog korišćenja kako bi prevodilac znao broj i tipove parametara funkcije kao i njihove povratne vrednosti. Skript jezici žrtvuju strogu tipiziranost kako bi proces pisanja koda bio brži. Stoga su oni obično \emph{dinamički} - promenljive mogu menjati tip tokom izvršavanja programa. Pošto promenljive mogu menjati svoj tip, definisanje tipa prilikom uvođenja imena promenljive postaje redundantno jer prevodilac može to sam da zaključi. Stoga i sam proces uvođenja imena promenljive postaje redundantan. Slično, parametri funkcija takođe nisu fiksnog tipa. Slično važi i za povratnu vrednost funkcije.

Kod proceduralnih jezika, pošto su obično strogo tipizirani, mogu se iskoristiti strukture podataka koje omogućavaju brz pristup svojim elementiram. To su obično nizovi koji predstavljaju kontinualni blok memorije u kom su elementi niza smešteni jedan do drugog. Pristup se vrši na osnovu indeksa i, pošto su svi elementi istog tipa (zauzimaju jednaku količinu memorije), može se u konstantnom vremenu izračunati memorijska lokacija na kojoj se nalazi element niza sa datim indeksom. Kompleksnije strukture podataka obično nisu podržane u samom jeziku. Neki proceduralni jezici dozvoljavaju veoma niski pristup kroz \emph{pokazivače} ili \emph{reference} na memorijske adrese (\texttt{C} i \texttt{C++}). Većina modernih proceduralnih jezika ne dozvoljava rad sa pokazivačima, ne brinući puno o efikasnosti, dok neki dozvoljavaju korišćenje pokazivača u specijalnim situacijama sa eksplicitnom naznakom (\texttt{C\#}).

Pored dinamičnosti kad je u pitanju tip promenljivih, skript jezici često imaju neke specifične strukture podataka ugrađene u sam jezik kao olakšice prilikom programiranja. Primarna struktura podataka je \emph{jednostruko ulančana lista} \footnote{Lista je rekurzivna kolekcija podataka koja se sastoji od glave koja sadrži vrednost određenog tipa, i pokazivača na rep - drugu listu. Specijalno, praznim pokazivačima se označava kraj liste (prazna lista).}, za razliku od niza kod proceduralnih jezika. Razlog zašto se koriste liste je delimično zbog toga što, kao i ostale promenljive, liste nisu strogo tipizirane. Moguće je u listu ubacivati elemente različitih tipova - što onemogućava skladištenje u kontinualnom bloku memorije (osim ukoliko je lista imutabilna, što nije obično slučaj). Skript jezici uglavnom omogućavaju indeksni pristup elementima liste pa programeru izgleda kao da radi nad običnim nizom. Neki skript jezici omogućavaju kreiranje \emph{asocijativnih nizova}, gde indeks niza ne mora biti ceo broj već može uzimati vrednost iz domena bilo kog tipa. Osim listi, obično su podržane i torke, i za njih važe iste slobode kao i za liste. Kompleksnije strukture podataka uključuju skupove i rečnike (drugačije nazivane i \emph{heš mape}, engl. \emph{hash map}) koji su kolekcija ključ-vrednost parova gde je dozvoljen indeksni pristup po vrednosti ključa. Skript programski jezici su skoro uvek interpretirani, iako se neki jezici mogu kompajlirati po potrebi za efikasnije ponovno izvršavanje. S obzirom da efikasnost nije u glavnom planu, u skript jezicima nije dozvoljen direktan pristup memoriji putem pokazivača ili referenci. 



\chapter{Opis opštih AST apstrakcija za imperativne jezike}
\label{chp:MyAST}

Kao što je opisano u odeljku \ref{sec:Paradigms}, dosta različitih "pod-paradigmi" potiče iz imperativne paradigme. Strukturna, proceduralna i skript paradigma, iako naizgled različite, poseduju veliki broj sličnih osobina i koncepata. Moderni programski jezici uzimaju korisne koncepte iz svakakvih paradigmi pa je teško vezati jezik za jednu konkretnu paradigmu. Ovo je motivacija za apstrahovanje koncepata različitih paradigmi, ali pre svega imperativne i njenih "derivata" --- proceduralne, skript i objektno orijentisane. U ovom poglavlju će biti opisana opšta apstrakcija za imperativnu paradigmu i njene derivate. To uključuje i skript jezike koji, kako će biti pokazano u ovom radu, mogu da se posmatraju na istom nivou kao i svoji proceduralni "rođaci".

Svaki programski jezik ima svoju gramatiku i na osnovu toga ima svoja gramatička pravila koja se oslikavaju u apstraktnim sintaksnim stablima tih jezika. Na slikama \ref{fig:ASTLua} i \ref{fig:ASTGo} se mogu videti razlike jezika \texttt{Lua} i \texttt{Go}, kao primere skript odnosno proceduralne paradigme, kad se posmatra njihov AST.

\begin{figure}[h!]
\centering
\includegraphics[scale=0.6]{images/ast_lua.png}
\caption{AST isečka koda pisanog u programskom jeziku Lua. Prikazano putem \url{https://astexplorer.net/}}
\label{fig:ASTLua}
\end{figure}

\begin{figure}[h!]
\centering
\includegraphics[scale=0.7]{images/ast_go.png}
\caption{AST isečka koda pisanog u programskom jeziku Go. Prikazano putem \url{https://astexplorer.net/}}
\label{fig:ASTGo}
\end{figure}

Kako bi se kreirala smislena apstrakcija stabla parsiranja, potrebno je identifikovati bitne informacije u stablu parsiranja ali i koncepte same gramatike koji su ponovno upotrebljivi. Najjednostavnije rešenje je mimikovati čvorove stabla parsiranja, ukoliko su gramatička pravila kreirana tako da oslikaju koncepte jezika koji gramatika definiše. Na primer, ukoliko u gramatici imamo pravilo \texttt{deklaracija} sa alternativama \texttt{deklaracijaPromenljive} i \texttt{deklaracijaFunkcije}, možemo kreirati apstraktni koncept \texttt{Deklaracija} sa konkretizacijama \texttt{DeklaracijaPromenljive} i \texttt{DeklaracijaFunkcije}. Kako se definišu deklaracije promenljivih i funkcija zavisi dalje od definicija pravila \texttt{deklaracijaPromenljive} i \texttt{deklaracijaFunkcije}. Naravno, nije uvek moguće primeniti ovakav postupak. Takođe, nekada u gramatici definišemo pomoćna pravila kako bismo se izborili sa rekurzijom ili izbegli neke tipove rekurzije --- ta pravila ne bi trebalo da imaju odgovarajuće tipove u opštoj apstrakciji. 

Pošto su u pitanju gramatike programskih jezika, onda je jasno da dosta različitih gramatika dele slične koncepte i da je moguće definisati tipove čvorova koji odgovaraju tim konceptima. Neki od njih mogu biti: naredba, izraz, deklaracija, poziv funkcije, dodela itd. Može se uočiti i hijerarhija između navedenih koncepata, međutim poziv funkcije se može smatrati kao samostalna naredba ali može biti i deo izraza. Dakle, prilikom definisanja hijerarhije ne treba dozvoliti nešto što nema smisla (npr. ako je dozvoljeno višestruko nasleđivanje i poziv funkcije je i naredba ali i izraz, onda se izrazi u kojima figurišu pozivi funkcija sastoje od više naredbi.).

Osim naredbi i izraza (koje vezuju operatori), kao osnovnih koncepata imperativnih jezika, deklaracije se ne pojavljuju u skript jezicima zbog slabe tipiziranosti. Moguće je, međutim, posmatrati i promenljive u kodovima skript jezika kao promenljive deklarisane neposredno pre trenutka njihove upotrebe --- detaljnije opisano u \ref{sec:MyASTDeclarationNodes}. Što se tiče njihovog tipa, može biti dozvoljena promena istog, ili, kako je izabrano u ovom radu, biće iskorišćen specijalni tip od kog potiču svi ostali tipovi.

\begin{figure}[h!]
\centering
\includegraphics[scale=0.6]{images/nodes.png}
\caption{Prikaz osnovnih vrsta AST čvorova.}
\label{fig:ASTNode}
\end{figure}

Na slici \ref{fig:ASTNode} se mogu videti osnovni tipovi AST čvorova zasnovani na konceptima opisanim iznad. U nastavku će po odeljcima biti detaljnije opisan svaki od prikazanih tipova.

\section{Čvorovi deklaracija}
\label{sec:MyASTDeclarationNodes}

\pangrami

\section{Čvorovi operatora}
\label{sec:MyASTOperatorNodes}

\pangrami

\section{Čvorovi izraza}
\label{sec:MyASTExpressionNodes}

\pangrami

\section{Čvorovi naredbi}
\label{sec:MyASTStatementNodes}

Naredbe su najkomplikovanije za apstrahovanje zbog njihove raznovrsnosti. Programski jezici često uvode nove sintaksne strukture i naredbe koje nisu do tada viđene u ostalim jezicima. Uprkos svemu tome, ipak je moguće uočiti neke sličnosti sa već postojećim konceptima i svesti ih na isti nivo. Na slici \ref{fig:StatementNodes} se mogu videti tipovi apstraktnih konstrukcija koje će se koristiti da bi se predstavile naredbe.

\begin{figure}[h!]
    \centering
        \includegraphics[scale=0.7]{images/statement_nodes.png}
    \caption{Vrste čvorova naredbi.}
    \label{fig:StatementNodes}
\end{figure}


\chapter{Semantičko poređenje opštih AST}
\label{chp:ASTComparing}

Jedna od motivacija svođenja imperativnih jezika na isti nivo apstrakcije (opisane u poglavlju \ref{chp:MyAST}) može biti poređenje kodova pisanih u različitim programskim jezicima. Pritom, s obzirom da su u pitanju stabla, moguće je koristiti razne algoritme za poređenje stabala (ali i grafova uopšte) nad ovakvim apstrakcijama. Pritom, potrebno je i definisati kriterijum poređenja --- moguće je porediti kodove \emph{strukturno}, \emph{semantički} itd. U ovom radu je od značaja semantička ekvivalentnost, koja je u opštem slučaju neodlučiv problem. Međutim, ukoliko se ograničimo samo na strukturno slične kodove, moguće je dobiti smislene rezultate u praksi, nalik na one dobijene u ovom radu. Precizna definicija strukturne sličnosti se obično iskazuje u terminima sličnosti strukture njihovih apstrakcija. Naravno, postoje kodovi koji nisu strukturno ekvivalentni ali su semantički ekvivalentni --- takvi slučajevi se onda neće razmatrati zbog neispunjenosti pretpostavke o strukturnoj sličnosti. 

Pretpostavka strukturne sličnosti je velika i znatno smanjuje broj slučajeva upotrebe takvog upoređivača. Međutim, danas je od velikog značaja verifikacija programa dobijenih sitnim refaktorisanjem već postojećh i verifikovanih programa, često ne menjajući strukturu uopšte (a ako se struktura koda menja onda se to obično radi u mmanjim koracima između kojih i dalje važi pretpostavka strukturne sličnosti kodova u susednim koracima). Slično, prepisivanja programa sa jednog programskog jezika na drugi su jako uobičajena, pa je i u takvim situacijama implicitno prisutna strukturna slučnost, što zavisi od konkretnih programskih jezika ali se u praksi često smanjuje napor tako što se održava struktura koda, barem u inicijalnim verzijama. 

Definicija strukturne sličnosti za potrebe ovog rada će se odnositi na sličnosti u rasporedu blokova naredbi dok će raspored naredbi u blokovima biti nebitan. Na primer, ukoliko se prvi program sastoji od bloka naredbi u kome se nalaze dva druga bloka, očekuje se da i drugi program ima istu organizaciju blokova, pri čemu se pod terminom blok podrazumevaju i složene naredbe koje se sastoje od bloka naredbi u sebi, kao što su definicije funkcija, uslovna grananja, petlje itd. Pod ovim uslovima, moguće je porediti blokove prvog programa sa odgovarajućim blokovima drugog programa.

U ovom radu je poređenje vršeno pomoću algoritma pisanog specifično za rad sa opštim apstrakcijama opisanim u ovom radu. Grubi opis algoritma za poređenje, u daljem tekstu \emph{upoređivač}, je opisan na slici \ref{fig:ComparisonAlgorithmPseudo}. Upoređivač se sastoji od više upoređivača koje porede specifične tipove čvorova. Za početak, potreban je jedan adapter koji će dobiti pokazivače na korene stabala koje je potrebno uporediti. S obzirom da tipovi čvorova mogu biti različiti, potrebno je proveriti da li su tipovi isti. Ukoliko to nije slučaj, prijavljuje se greška i rad se prekida. U protivnom, potrebno je odrediti tip čvorova i pozvati konkretni algoritam za poređenje. 

\begin{figure}[!h]
\begin{algorithmic}[1]
\Procedure{Uporedi}{$n_1$, $n_2$}
\If{\emph{$n_1$ i $n_2$ su istog tipa}}
    \State $t \gets$ \emph{tip čvora $n_1$}
    \If{\emph{postoji definisan upoređivač za čvorove tipa} $t$}
        \State $U \gets$ \emph{upoređivač čvorova tipa $t$}
        \State \textbf{return} U$(n_1, n_2)$
    \Else
        \If{$\text{BrojDece}(n_1) \neq \text{BrojDece}(n_2)$}d
            \State \textbf{return} \texttt{False}
        \Else        
            \If{$\text{Atributi}(n_1) \neq \text{Atributi}(n_2)$}
                \State \textbf{return} \texttt{False}
            \EndIf
            \For{$i \gets 0$ \textbf{to} $\text{BrojDece}(n_1)$}
                \State $d_1 \gets $ \emph{dete $i$ čvora $n_1$}
                \State $d_2 \gets $ \emph{dete $i$ čvora $n_2$}
                \If{\textbf{not} $\text{Uporedi}(d_1, d_2)$}
                    \State \textbf{return} \texttt{False}
                \EndIf
            \EndFor
            \State \textbf{return} \texttt{True}
        \EndIf
    \EndIf
\Else
    \State \textbf{return} \texttt{False}
\EndIf
\EndProcedure
\end{algorithmic}
\caption{Osnovni AST upoređivač.}
\label{fig:ComparisonAlgorithmPseudo}
\end{figure}

Podrazumevana implementacija poređenja može biti takva da se uporede atributi svih čvorova a zatim se svako dete prvog čvora rekurzivno uporedi sa odgovarajućim detetom drugog čvora (ukoliko imaju isti broj dece). Ako neki par dece nije ekvivalentan, onda to ne važi ni za njihove roditelje. Za većinu tipova čvorova ovakvo poređenje je dovoljno. Međutim, poređenje blokova naredbi je fundamentalno drugačije i za njega će biti definisane posebna procedura poređenja opisana u nastavku.


\section{Upoređivač blokova naredbi}
\label{sec:ASTComparingBlocks}

Podrazumevani način poređenja dece svakog čvora nije dobar u opštem slučaju za blokove naredbi jer je osetljiv na izmene redosleda naredbi --- na primer promena redosleda deklaracija. Stoga je upoređivač blokova potrebno napisati tako da može da uoči semantičku ekvivalentnost iako naredbe nisu nužno jednake, a možda ih čak ima i različit broj.

Upoređivač se zasniva na poređenju vrednosti promenljivih na kraju svakog bloka naredbi. Apstrakcije dva programa će se porediti paralelno --- \emph{blok-po-blok}. Naredbe svakog bloka će se izvršavati i pratiće se izmene vrednosti promenljivih deklarisanih do sada (bilo u trenutnom bloku, ili u roditeljskim blokovima). Na kraju svakog bloka će se izvršiti provera vrednosti promenljivih --- svaka razlika će se prijaviti kao potencijalna greška a finalnu presudu o jednakosti će dati analiza jednakosti promenljivih. Ceo algoritam je prikazan na slici \ref{fig:ComparisonAlgorithmBlocksPseudo}.

\begin{figure}[!h]
\begin{algorithmic}[1]
\Procedure{UporediBlokove}{$b_1$, $b_2$}
\State $gds_1 \gets $ \emph{simboli iz svih predaka bloka $b_1$}
\State $gds_2 \gets $ \emph{simboli iz svih predaka bloka $b_2$}
\State $lds_1 \gets $ \emph{lokalni simboli za blok $b_1$}
\State $lds_2 \gets $ \emph{lokalni simboli za blok $b_2$}
\State $\text{UporediSimbole}(lds_1, lds_2)$
\State $\text{IzvrsiNaredbe}(b_1, b_2, lds_1, lds_2, gds_1, gds_2)$
\State \textbf{return} $\text{UporediSimbole}(lds_1, lds_2) \wedge \text{UporediSimbole}(gds_1, gds_2)$
\EndProcedure
\end{algorithmic}
\caption{Upoređivač blokova naredbi.}
\label{fig:ComparisonAlgorithmBlocksPseudo}
\end{figure}

U opisu algoritma se koristi termin \emph{simbol} koji se sastoji od identifikatora i simboličke vrednosti promenljive. Lokalni simboli su deklarisani unutar bloka a globalni su svi simboli koji su deklarisani van trenutnog bloka a koji se mogu referisati iz njega. Pronalaženje deklarisanih simbola u bloku podrazumeva prolaz kroz naredbe bloka i registrovanje svih naredbi deklaracije, izvlačenje deklaratora iz njih i, uzimajući u obzir opcione inicijalizatore, kreiranje simboličke vrednosti za upravo deklarisani identifikator. Identifikator i opcioni simbolički inicijalizator čine \emph{simbol}. Isto se ponavlja za sve naredbe deklaracije u bloku i rezultat je skup deklarisanih simbola.

Nakon registrovanja svih lokalnih simbola proverava se njihova ekvivalentnost u funkciji \texttt{UporediSimbole}. Ova funkcija proverava da li se svi simboli iz prvog bloka nalaze u drugom i prijavljuje ukoliko neki simboli fale ili ukoliko postoje simboli koji su višak. Zatim, za simbole koji se nalaze u oba skupa, proverava njihove simboličke vrednosti. Ukoliko su te vrednosti različite, prijavljuje se potencijalna greška i na osnovu toga da li je bilo konflikata vraća se istinitosna vrednost. Razlog zašto se ta vrednost ne koristi dalje nakon prvog poziva ove funkcije je ta što različiti inicijalizatori ne znače nužno da postoji problem. Problem postoji ukoliko se nakon izvršavanja svih naredbi i dalje dešavaju konflikti u simboličkim vrednostima za neke promenljive. 

Procedura \texttt{IzvrsiNaredbe} izvršava paralelno naredbe iz oba bloka i na osnovu toga koje su naredbe u pitanju može i da ažurira simboličke vrednosti unutar skupova deklarisanih simbola. Pseudokod ove procedure je dat na slici \ref{fig:ComparisonAlgorithmBlocksPseudo1}. Naredbe se za svaki blok izvršavaju dok se ne naiđe do naredbe iz koje se može izvući novi blok --- to mogu biti naredbe grananja, iteracije, definicije funkcija i slično. Sve naredbe do pronađene naredbe se izvršavaju. Procedura \texttt{IzvrsiNaredbu} će proveriti tip naredbe i, u zavisnosti od toga da li je to naredba dodele, eventualno promeniti vrednosti u skupovima prosleđenih simbola. Nakon izvršavanja svih naredbi do pronađene naredbe koja sadrži blok, izvlači se blok iz nje (to isto se radi i za drugi program). Kad se blokovi izvuku, rekurzivno se poziva upoređivač blokova za pronađene parnjake. Po povratku iz rekurzivnog poziva nastavlja se isti postupak sve dok se ne izvrše sve naredbe. Pritom, algoritam se oslanja na strukturnu sličnost --- ukoliko jedan AST ima više blokova na istoj dubini u odnosu na drugi, poređenje možda neće uočiti neke razlike.

\begin{figure}[!h]
\begin{algorithmic}[1]
\Procedure{IzvrsiNaredbe}{$b_1$, $b_2, lds_1, lds_2, gds_1, gds_2$}
\State $n_1 \gets $ \emph{niz naredbi bloka $b_1$} 
\State $n_2 \gets $ \emph{niz naredbi bloka $b_2$}
\State $i \gets j \gets 0$
\State $ni \gets nj \gets 0$
\State $eq \gets $ \texttt{True}
\While{\texttt{True}}
    \State $ni \gets $ \emph{indeks prve naredbe koja sadrži blok u $n_1$ počev od indeksa $ni$}
    \State $nj \gets $ \emph{indeks prve naredbe koja sadrži blok u $n_2$ počev od indeksa $nj$}
    \For{$naredba \in \{n_1[x] \mid x \in [i..ni]\}$}
        \State $\text{IzvrsiNaredbu}(naredba, lds_1, gds_1)$
    \EndFor
    \State $i \gets i + ni$
    \For{$naredba \in \{n_2[x] \mid x \in [j..nj]\}$}
        \State $\text{IzvrsiNaredbu}(naredba, lds_2, gds_2)$
    \EndFor
    \State $j \gets j + nj$
    \If{$i > \text{Duzina}(n_1) \vee j > \text{Duzina}(n_2)$}
        \State \textbf{prekini petlju}
    \EndIf
    \State $nb_1 \gets $ \emph{izvuci blok iz naredbe $n_1[i]$}
    \State $nb_2 \gets $ \emph{izvuci blok iz naredbe $n_2[j]$}
    \State $eq \gets eq \wedge \text{UporediBlokove}(nb_1, nb_2)$
    \State $i \gets i + 1$
    \State $j \gets j + 1$
\EndWhile
\State \textbf{return} $eq$
\EndProcedure
\end{algorithmic}
\caption{Upoređivač blokova naredbi.}
\label{fig:ComparisonAlgorithmBlocksPseudo1}
\end{figure}

\input{chapters/50_conclusion.tex}

% ------------------------------------------------------------------------------
% Literatura
% ------------------------------------------------------------------------------
\literatura

% ==============================================================================
% Završni deo teze i prilozi
\backmatter
% ==============================================================================

% ------------------------------------------------------------------------------
% Biografija kandidata
\begin{biografija}
  \textbf{Ivan Ž. Ristović} rođen je 17.01.1995. godine u Užicu. Osnovnu školu, kao i 
  prirodno-matematički smer Užičke gimnazije, završio je kao nosilac Vukove diplome. 
  Tokom navedenog perioda školovanja isticao se u oblastima matematike, informatike, 
  fizike, hemije i engleskog jezika, što potvrđuje veći broj nagrada na Državnim 
  takmičenjima.

  Smer Informatika na Matematičkom fakultetu Univerziteta u Beogradu upisuje 2014. 
  godine. Na navedenom smeru je diplomirao 2018. godine, posle tri godine studija 
  sa prosečnom ocenom 9,17. Master studije upisuje na istom fakultetu odmah nakon 
  diplomiranja.
  
  U avgustu 2018. biva izabran u zvanje „Saradnik u nastavi“ na Matematičkom 
  fakultetu paralelno sa master studijama. Drži vežbe iz kurseva "Računarske mreže",
  "Funkcionalno programiranje", "Programske paradigme" i "Objektno orijentisano 
  programiranje" na kasnijim godinama osnovnih studija.
  
  Oblasti interesovanja uključuju pre svega razvoj i verifikaciju softvera, mikroservise 
  i računarske mreže.
\end{biografija}
% ------------------------------------------------------------------------------

\end{document}
